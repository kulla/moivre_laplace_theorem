\chapter{The De Moivre-Laplace theorem}

\section{Formulation of De Moivre-Laplace theorem}

In the last chapter we have seen with the application of the binomial distribution in the investigation of the sex ratio that calculations with the binomial distribution are arduous. In order to compute $\binom nk p^kq^{n-k}$ there are $n+2\min\{k,n-k\}-1$ operations necessary.

This motivates approximations to the binomial distribution to minimize the computational efforts. With those approximations also equations like $\P{\Bs \le z}=\alpha$ are easier to solver for $z$ or $n$ ($\Bs$ shall be binomially distributed). Explicit solutions for those equations are hard to calculate~\cite[p. 469]{hald1}.

The first simple approximation to the binomial distribution was found by Abraham de Moivre with the help of James Stirling~\cite[p. 469]{hald1}. His proof was later extended by Pierre Simon Laplace~\cite[pp. 495 ff.]{hald1}. Today this approximation is known as the ``De Moivre-Laplace theorem''~\cite[pp. 64-67]{irle}. \todo{besseres Zitat}

\section{The historical proof of the De Moivre-Laplace theorem}

\section{Applications of De Moivre-Laplace theorem}

Graunt + sex ratio, Computerprogramme
