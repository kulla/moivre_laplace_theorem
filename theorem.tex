\chapter{The De Moivre-Laplace theorem}

\section{Formulation of De Moivre-Laplace theorem}

\includefig[0.25\textwidth]{moivre}{Abraham de Moivre}{File \href{https://commons.wikimedia.org/wiki/File:Abraham_de_moivre.jpg}{``Abraham de moivre.jpg''} from Wikimedia Commons. Uploaded by user \href{https://commons.wikimedia.org/wiki/User:\%E7\%AB\%B9\%E9\%BA\%A6\%E9\%AD\%9A(Searobin)} Searobin and licensed under Public Domain. Original source: \url{https://www.york.ac.uk/depts/maths/histstat/people/} .}

\includefig[0.25\textwidth]{laplace}{Pierre Simon Laplace}{File \href{http://flickr.com/photos/37667416@N04/4840056407}{``Laplace''} from Flickr. Uploaded by \href{http://www.flickr.com/people/37667416@N04}{Biblioteca de la Facultad de Derecho y Ciencias del Trabajo Universidad de Sevilla} and licensed under \href{https://creativecommons.org/licenses/by/2.0/deed.en}{CC-BY 2.0}.}

In the last chapter we have seen with the application of the binomial distribution in the investigation of the sex ratio that calculations with the binomial distribution are arduous. In order to compute $\binom nk p^kq^{n-k}$ there are $n+2\min\{k,n-k\}-1$ operations necessary.

This motivates approximations to the binomial distribution to minimize the computational efforts. With those approximations also equations like $\P{\Bs \le z}=\alpha$ are easier to solver for $z$ or $n$ ($\Bs$ shall be binomially distributed). Explicit solutions for those equations are hard to calculate~\cite[p. 469]{hald1}.

The first simple approximation to the binomial distribution was found by Abraham de Moivre with the help of James Stirling~\cite[p. 469]{hald1}. His proof was later extended by Pierre Simon Laplace~\cite[pp. 495 ff.]{hald1}. Today this approximation is known as the ``De Moivre-Laplace theorem''~\cite[pp. 64-67]{irle}. \todo{besseres Zitat}

I want to split this theorem in two parts. With the first part one get an approximation for the probability mass function of a binomial distribution:

\begin{theorem}[Local version of De Moivre-Laplace theorem]
  Let $p\in(0,1)$. ...
\end{theorem}

In the second part one gets an approximation to the culmulative distribution function which is the actual theorem by De Moivre and Laplace:

\begin{theorem}[De Moivre-Laplace theorem]
  ...
\end{theorem}

\section{The journey to De Moivre-Laplace theorem}

In \emph{A history of propability and statistics and their applications before 1750} Anders Hald gave a good summarize how de Moivre and Laplace found their theorem. In \cite[pp. 469-470]{hald1} he wrote:

\begin{quotation}
  In 1721 de Moivre began his investigations od the binomial distribution for $p=\tfrac 12$. He first found an approximation to the maximum term and next an approximation to the ratio of the maximum to the term at a distance of $d$ from the maximum. [...] From 1725 onward James Stirling (1692-1770) worked on the same problem and found that the constant entering de Moivre's formula equals $\sqrt{2\pi}$. After having obtained these results they realized that it would be simpler to begin with an approximation to $\ln n!$, and they both proved Stirling's formula, $n! \sim \sqrt{2\pi n}n^ne^{-n}$. De Moivre's proofs are given in the \emph{Miscellanea Analytica} (MA)(1730), Stirling's in his \emph{Methodus Differentialis} (1730).

  [...] Three years later, de Moivre (1733) simplified his result for $p=\tfrac 12$ and showed that the normal density function may be used as an approcimation to the binomial. The generalization to an arbitrary value of $p$ is of course very easy, so without proof de Moivre stated that

  \begin{align*}
      b(np+d,n,p) \sim (2\pi npq)^{-\tfrac 12} \exp\left( -\frac{d^2}{2npq} \right), d = O(\sqrt n),
  \end{align*}

  and that $P_d$ [with $P_d=\P{|\Bs -np|\le d}$] may be obtained by integration. He also showd how to calculate the standardized normal probability integral [numerically] and gave the result for one, two, and three times the standard deviation.
\end{quotation}

In his book \emph{A History of Mathematical Statistics From 1750 to 1930} Hald wrote about the proof by Laplace~\cite[p. 24]{hald2}:

\begin{quotation}
  De Moivre's theorem is mentioned only occasionally in the probabilistic literature between 1740 and 1812, when Laplace (TAP, II, \$16) \todo{cite} filled the gaps in the proof. Laplace assumes that p is a real number in the unit interval \todo{foonote} and shows that the mode of the binomial distribution equals $m=[(n+1)p]=np+z$, say, $-q < z \le p$. By means of Stirling's formula he obtains an expansion of $\ln(m+d)$ in powers of $d$ including terms of order $n^{-1/2}$, so that he gets a correction to the main term depending on $(q-p)d/npq$, a correction of skewness as it is called today. He notes, that the approximation may be improved by taking more terms of the expansion into account.
\end{quotation}

\todo{ist $p$ und $q$ definiert?}

\section{Proof of the De Moivre-Laplace theorem}

In this section I want to prove the theorem by de Moivre and Laplace whereby I want to follow the basic ideas of their original proof. So we will first derive Stirling's formula of the factorial with which we will find an approximation of the binomial's probability mass function. This approximation will be used to deduce de Moivre's and Laplace's theorem.

\subsection{Stirling's formula}

In the following I want to write $a_n \asim b_n$ for sequences $\seq{a_n}$ and $\seq{b_n}$ with nonzero elements, whenever $\lim_{n\to\infty} \tfrac{a_n}{b_n}=1$. Thus

\begin{align}
  a_n \asim b_n \iff \lim_{n\to\infty} \frac{a_n}{b_n} = 1
\end{align}

\noindent Without proof the following theorem\footnote{See~\cite[p. 505]{heuser} and \cite[p. 63]{heuser}} will be used in this section:

\begin{theorem}[Wallis' product]
  It is:

  \begin{align}
    \lim_{n\to\infty} \frac{2^2\cdot4^2\cdot6^2\dots(2n)^2}{1^2\cdot3^2\cdot5^2\dots(2n-1)^2}\cdot \frac{1}{2n} = \lim_{n\to\infty} \frac{1}{2n} \cdot \frac{2^{4n}}{\binom{2n}{n}^2} = \frac{\pi}{2}
  \end{align}
\end{theorem}

\noindent A proof of this equality can be found in~\cite[pp. 504-505]{heuser}. From this equation directly follows: 

\begin{align} \label{wallis}
  \lim_{n\to\infty} \frac{2^{2n}}{\sqrt n \binom{2n}{n}} = \lim_{n\to\infty} \sqrt{\frac 1n \cdot \frac{2^{4n}}{\binom{2n}{n}^2}} = \sqrt \pi
\end{align}

\noindent Besides we will also need the Euler–Maclaurin formula~\cite[p. 226]{koenigsberger}:

\begin{theorem}[Euler–Maclaurin formula]
  Let $f:[1,n]\to\R$ be a $2n+1$-times continuously differentiable function. Let $\bn$ be the $n$th Bernoulli number and let $\bp{x}$ be the $n$th periodic Bernoulli polynomial. It is

  \begin{multline} \label{euler-maclaurin-formula}
    \sum_{k=1}^n f(k) = \int_1^n f(x)\d{x} + \frac{f(1)+f(n)}{2} + \sum_{k=1}^n \frac{\bn[2k]}{(2k)!} \Big[f^{(2k-1)}(n) - f^{(2k-1)}(1)\Big] \nl
     + \int_1^n \frac{\bp[2n+1]{x}}{(2n+1)!} f^{(2n+1)}(x) \d{x}
  \end{multline}
\end{theorem}

\noindent The periodic Bernoulli polynomials $\bps$ are defined on $[0,1)$ by~\cite[p. 291]{koenigsberger}:

\begin{enumerate}
  \item $\bp[0]{x} = 1$
  \item $\bps[n+1]'(x) = (n+1)\cdot \bp{x}$
  \item $\int_0^1 \bp{x} \d{x} = 0$
\end{enumerate}

The definition of $\bps$ on $[0,1)$ is then periodically continued to the whole domain $\R$. The Bernoulli numbers $\bn$ fulfill $\bn=\bp{0}$. All odd Bernoulli numbers $\bn[2n-1]$ for $n>1$ are zero and for the even Bernoulli numbers we get~\cite[p.~289]{koenigsberger}:

\begin{align}
  \bn[2] = \frac 16;\quad \bn[4]=-\frac{1}{30};\quad \bn[6]=\frac{1}{42};\quad \ldots
\end{align}
  
You can find a proof of the Euler-Maclaurin formula in~\cite[pp.~225-226]{koenigsberger} and~\cite[pp.~506-509]{heuser}. With the above two theorems we can derive Stirling's formula: \todo{Wer?}

\begin{theorem}[Stirling's formula]
  The factorial $n!$ fulfills:
  
  \begin{align}
    n! \asim \sqrt{2\pi n} \left(\frac ne\right)^n
  \end{align}
\end{theorem}

\begin{proof}
  In the proof we will follow~\cite[pp. 227-228]{koenigsberger}. First we apply the Euler-Maclaurin formula~\eqref{euler-maclaurin-formula} to $\ln(n!)=\sum_{k=1}^n \ln(k)$:

  \begin{align}
    \ln(n!) & = \sum_{k=1}^n \ln(k) \nl
    & = \int_1^n \ln(x) \d{x} + \frac{\ln(1)+\ln(n)}{2} + \frac{\bn[2]}{2!} \Big[\ln'(n)-\ln'(1)\Big] + \int_1^n \frac{\bp[3]{x}}{3!} \ln^{(3)}(x) \d{x} \nl
    &= \int_1^n \ln(x) \d{x} + \frac 12 \ln(n) + \frac{1}{12} \left(\frac 1n - 1\right) + \frac 13 \int_1^n \frac{\bp[3]{x}}{x^3} \d{x} \nl
    & \begin{comment}
      \int_1^n \ln(x) \d{x} = \Big[x\ln(x)-x\Big]_1^n = n\ln(n)-n+1
    \end{comment} \nonumber \nl
    &= n\ln(n)-n+\frac 12 \ln(n) + \frac{1}{12n} + \frac{11}{12} + \frac{1}{3} \int_1^n \frac{\bp[3]{x}}{x^3}\d{x} 
  \end{align}

  \noindent Thus

  \begin{align}
    \ln(n!) - n\ln(n) + n - \frac 12 \ln(n) = \frac{11}{12} + \frac{1}{12n} + \frac 13 \int_1^n \frac{\bp[3]{x}}{x^3} \d{x}
  \end{align}

  $\bps[3]$ is on $[0,1)$ as part of a polynomial bounded. Because $\bps[3]$ is $1$-periodic, this function is bounded on the whole domain $\R$. Thus $\int_1^\infty \frac{\bp[3]{x}}{x^3} \d{x}$ exists. Now we define $\seq{b_n}$ via

  \begin{align}
    b_n = \frac{n!}{n^n e^{-n} \sqrt{n}}
  \end{align}

  \noindent This sequence converges because

  \begin{align}
    \lim_{n\to\infty} \ln(b_n) & = \lim_{n\to\infty} \Big( \ln(n!) -n \ln(n) + n - \frac 12\ln(n) \Big) \nl
    & = \lim_{n\to\infty} \left( \frac{11}{12} + \frac{1}{12n} + \frac 13\int_1^n \frac{\bp[3]{x}}{x^3} \d{x}\right) \nl
    & = \frac{11}{12} + \frac 13\int_1^{\infty} \frac{\bp[3]{x}}{x^3} \d{x}
  \end{align}

  \noindent Let $b=\lim_{n\to\infty} b_n$. To calculate this limit we investigate $\tfrac{b_n^2}{b_{2n}}$:

  \begin{align}
    \lim_{n\to\infty} \frac{b_n^2}{b_{2n}} &= \lim_{n\to\infty} \frac{(n!)^2}{n^{2n}e^{-2n} n} \cdot \frac{(2n)^{2n} e^{-2n} \sqrt{2n}}{(2n)!} \nl
    &= \lim_{n\to\infty} \sqrt{2} \frac{2^{2n}}{\sqrt{n}\binom{2n}{n}} \nl
    & \begin{comment}
    \lim_{n\to\infty} \frac{2^{2n}}{\sqrt{n}\binom{2n}{n}} = \sqrt{\pi} \text{, see \eqref{wallis}}
    \end{comment} \nonumber \nl
    & = \sqrt{2\pi}
  \end{align}

  \noindent On the other hand we have

  \begin{align}
    \lim_{n\to\infty} \frac{b_n^2}{b_{2n}} = \frac{b^2}{b} = b
  \end{align}

  \noindent Thus $\lim_{n\to\infty} b_n = \sqrt{2\pi}$. From this we can follow

  \begin{align}
    \lim_{n\to\infty} \frac{n!}{\sqrt{2\pi n}n^n e^{-n}} = \lim_{n\to\infty} \frac{b_n}{\sqrt{2\pi}} = 1
  \end{align}

  \noindent which proves $n! \asim \sqrt{2\pi n}\left(\tfrac ne\right)^{-n}$.
\end{proof}

\subsection{The local version of de Moivre-Laplace theorem}

\section{Application}

Graunt + sex ratio, Computerprogramme
