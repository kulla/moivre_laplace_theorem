\chapter{Conclusion}

In this thesis I proposed a new proof for the de Moivre-Laplace theorem, which could represent a valid alternative to the standard proof concerning error estimation and concerning comprehensibility in class or in textbooks. 

However, compared with the classic method, using Stirling’s formula, my alternative proof could not yield better approximations or smaller error bounds. This was already prevalent when testing it against the local version of the de Moivre-Laplace theorem.

However the new proof has some clear advantages over the standard one:

\begin{enumerate}
  \item My new proof only uses the following property of the binomial distribution
    \begin{align}
      \bb{\x[k+1]} \in \bb{\x[k]} - \x[k]\bb{\x[k]} h + \bigo{\bb{\x[k]} \pol{\x[k]} h^2}
    \end{align}
    Therefore, the same proof applies to each discrete probability distribution fulfilling this property.

  \item My alternative proof connects the theory of ordinary differential equations in an interesting way with probability theory: The one-step method corresponds to the given function and the solution of the ODE is the approximation. Thus, the role of approximated function and approximation is exchanged.

  \item Eventhough, it is a very subjective decision to rank one proof more understandable than another \cite{tampis:understandability}, the new proof arguably is easy to understand and intuitive. Therefore, it could be used in courses or textbooks for undergraduates. An investigation on this issue is out of scope for this thesis.
\end{enumerate}

Eventhough my alternative proof does not result in better approximations or smaller error bounds, the presented proof is a valuable alternative for teaching and understanding the de Moivre-Laplace theorem and an addition to the classic method. 
