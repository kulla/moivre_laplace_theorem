\chapter{Conclusion}

In the local version of de Moivre-Laplace we have seen that the error estimation provided by the alternative proof is worse than in the standard proof. Thus we have to give a negative answer to our introductory question whether the alternative proof yields a better approximation or a smaller bound for the error.

However the alternative proof has some advantages over the standard proof with Stirling's formula:

\begin{enumerate}
  \item From the binomial distribution only the property
    \begin{align}
      \bb{\x[k+1]} = \bb{\x[k]} - \x[k]\bb{\x[k]} h + \bigo{\bb{\x[k]} \pol{\x[k]} h^2}
    \end{align}
    is used. Therefore the same proof can be applied to each discrete probability distribution which fulfills the same property.

  \item The alternative proof connects the theory of ordinary differential equations in an interesting way with probability theory: The one-step method corresponds to the given function and the solution of the ODE is the approximation. Thus the role of approximated function and the approximation is exchanged.

  \item The alternative proof is easy to understand and intuitive. Therefore it can be used in a course or a textbook for undergraduates. However it is hard to show that one proof is more understandable than the other \cite{tampis:understandability} and an investigation on this issue is out of scope for this thesis. 
\end{enumerate}

To summarize our results: The presented proof by approximating the slopes through an ODE cannot be used to get a better approximation but is a good alternative in teaching mathematics.
