\chapter{Conclusion}

We have seen that already for the local version of the de Moivre-Laplace theorem the error estimation provided by the alternative proof is worse than in the standard proof. Thus we have to give a negative answer to our introductory question whether the alternative proof yields a better approximation or a smaller error bound. However the new proof has some advantages over the standard proof with Stirling's formula:

\begin{enumerate}
  \item From the binomial distribution only the property
    \begin{align}
      \bb{\x[k+1]} = \bb{\x[k]} - \x[k]\bb{\x[k]} h + \bigo{\bb{\x[k]} \pol{\x[k]} h^2}
    \end{align}
    is used. Therefore, the same proof can be applied to each discrete probability distribution that fulfills the same property.

  \item The alternative proof connects the theory of ordinary differential equations with probability theory in an interesting way: The one-step method corresponds to the given function and the solution of the ODE is the approximation. Thus, the role of approximated function and approximation is exchanged.

  \item The new proof is easy to understand and intuitive. Therefore, it can be used in a course or a textbook for undergraduates. However, it is difficult to show that one proof is more understandable than another \cite{tampis:understandability} and further investigation into this issue is out of scope for this thesis.
\end{enumerate}

To summarize the results: The presented proof by approximating the slopes cannot be used to get a better approximation, but is a good alternative for teaching the de Moivre-Laplace theorem.
