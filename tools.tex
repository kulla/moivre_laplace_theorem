\chapter{Tools for this thesis}

In this chapter I will introduce and explain the tools necessary for the upcoming proofs. To keep this chapter short I will prove all following theorems in the appendix.

\section{The big Psi notation}

\subsection{Definition of the big Psi notation}
An important part of this thesis is dedicated to estimate the quality of the normal approximation for the binomial distribution. Therefore we will investigate the error of the cumulative distribution function between the normal distribution and the binomial distribution.

To describe the asymptotic behavior of an error, the big O notation can be used. It declares the convergence speed with which the error converges at least to zero. Given an error $\epsilon_n$ for the $n$-th step of an approximation the big O notation $\epsilon_n \in \bigo{a_n}$ means \todo{cite}

\begin{align} \label{bigo_def1}
  \seq{\epsilon_n} \in \bigo{a_n} :\iff \exists C > 0: \abs{\epsilon_n} \le C \cdot \abs{a_n}
\end{align}

\noindent This definition is equivalent to \todo{cite}

\begin{align} \label{bigo_def_limsup}
  \seq{\epsilon_n} \in \bigo{a_n} :\iff \exists C_\infty > 0: \limsup_{n\to\infty} \abs{\frac{\epsilon_n}{a_n}} \le C_\infty
\end{align}

So $\epsilon_n \in \bigo{\frac 1n}$ means that $\epsilon_n$ approaches zero at least with the convergence speed as of $\seq{\frac 1n}$. Thereby the advantage of the big O notation are its simple arithmetic rules. For example we have

\begin{align}
  \bigo{\frac 1n} + \bigo{\frac 1{n^2}} \subseteq \bigo{\frac 1n}
\end{align}

This and other arithmetic rules simplify error estimations. Unfortunately the big O notation does only provide an estimate of the convergence speed but no estimation of the error's value since the constants $C$ and $C_\infty$ in \eqref{bigo_def1} and \eqref{bigo_def_limsup} are not known \todo{cite/footnote}.

Therefore I want to introduce the big Psi notation with which next to the convergence speed also an estimate for the asymptotic error can be stated. The big Psi notation is defined as:

\begin{definition}[Big Psi notation]
  Let $\seq{a_n}$ be a sequence with $a_n > 0$ for all $n\in\N$. $\bigpsi{a_n}$ is the set of all sequences $\seq{\epsilon_n}$ fulfilling $\limsup_{n\to\infty} \abs{\frac{\epsilon_n}{a_n}} \le 1$. Thus

  \begin{align}
    \seq{\epsilon_n} \in \bigpsi{a_n} :\iff \limsup_{n\to\infty} \abs{\frac{\epsilon_n}{a_n}} \le 1
  \end{align}

  \todo{notation with $\seq{\epsilon_n}$ / cite question on MSE}
\end{definition}

Note that the definition of the big Psi notation is the same as the limes superior definition \eqref{bigo_def_limsup} of the big O notation whereby $C_\infty$ has the concrete value $1$. Thus we directly see

\begin{align}
  \seq{\epsilon_n} \in \bigpsi{a_n} \implies \seq{\epsilon_n} \in \bigo{a_n}
\end{align}

\noindent Due to the properties of the limes superior we may also define the big Psi notation as\footnote{This definition has the advantage that it also allows $a_n=0$ for any $n\in\N$.} \todo{proof?}

\begin{align}
  \seq{\epsilon_n} \in \bigpsi{a_n} \iff \forall \epsilon > 0\, \exists N \in \N\, \forall n\ge N: \abs{\epsilon_n} \le (1+\epsilon)\abs{a_n}
\end{align}

So $\seq{\epsilon_n}\in\bigpsi{a_n}$ iff for each $\epsilon > 0$ almost all $\abs{\epsilon_n}$ are less then or equal $(1+\epsilon)\abs{a_n}$. Because an estimation is only stated for almost all $n\in\N$ and not for all $n\in\N$ the big Psi notation provides an upper bound for the asymptotic error.

Note that $\epsilon_n\in\bigpsi{a_n}$ does not imply $\abs{\epsilon_n}\le\abs{a_n}$ for any $n\in\N$. Take for example $\epsilon_n = \frac 1n + \frac 1{n^2}$. We have $\epsilon_n \in \bigpsi{\frac 1n}$ although $\epsilon_n > \frac 1n$ for all $n\in\N$. \todo{proof?}

\subsection{Arithmetic rules}

The big Psi notation shares with the big O notation a set of easy arithmetic rules. For example we also have

\begin{align} \label{bigpsi_rule_example}
  \bigpsi{\frac an}+\bigpsi{\frac b{n^2}} \subseteq \bigpsi{\frac an}
\end{align}

That's the reason why we will use the big Psi notation in this thesis. If we would estimate the error $\epsilon_n$ for each $n\in\N$ and not its limit behavior, we could not neglect the term $\bigpsi{\frac b{n^2}}$ in the above equation \eqref{bigpsi_rule_example}. Thus the error term would become more complicated.

We also cannot apply simplifications which are only true for almost all $n\in\N$ (and false for finitely many $n\in\N$). This is the reason why I will only consider the asymptotic error in this thesis.

The following theorem contains all arithmetic rules for the big Psi notation which we will use in this thesis. You will find a proof in the appendix.

\begin{theorem}
  The big Psi notation has the following arithmetic rules:

\end{theorem}

\section{Interval arithmetic}
