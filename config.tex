\title{De Moivre–Laplace theorem}
\author{Stephan Kulla}

\allowdisplaybreaks

\theoremstyle{plain}
\newtheorem{lemma}{Lemma}
\newtheorem{proposition}{Proposition}
\newtheorem{theorem}{Theorem}
\newtheorem{definition}{Definition}

\theoremstyle{remark}
\newtheorem{remark}{Remark}

\newcommand*{\Cmax}{0.4748}
\newcommand*{\Cmin}{0.40973}

\newcommand*{\refneeded}{\todo{Referenz!}}

\newcommand*{\includefig}[3]{\begin{figure}[h] \begin{center} \includegraphics[width=0.5\textwidth,draft=false]{figures/{#1}} \caption[#3]{#2} \end{center} \end{figure}}
\newcommand*{\includewrapfig}[3]{\begin{wrapfigure}{r}{0.33\textwidth} \vspace{-20pt} \begin{center} \includegraphics[width=0.3\textwidth,draft=false]{figures/{#1}} \caption[#3]{#2} \end{center} \vspace{-30pt} \end{wrapfigure}}

\newcommand*{\seq}[2][n]{\left({#2}\right)_{{#1}\in\N}}
\newcommand*{\abs}[1]{\left|{#1}\right|}
\newcommand*{\norm}[1]{\left\|{#1}\right\|}
\newcommand*{\snorm}[1]{\left\|{#1}\right\|_\infty}
\newcommand*{\E}[1]{\mathbb E\left[{#1}\right]}
\renewcommand*{\P}[1]{\mathcal P\left({#1}\right)}
\newcommand*{\e}[1]{e^{#1}}

\newcommand*{\f}[1]{f\left({#1}\right)}
\newcommand*{\h}[1]{h\left({#1}\right)}
\newcommand*{\fp}[1]{f'\left({#1}\right)}
\newcommand*{\ft}[1]{\tilde f\left({#1}\right)}

\newcommand*{\cfs}[2][-\infty]{\mathds{1}_{\left[{#1},{#2}\right]}}
\newcommand*{\cf}[3][-\infty]{\mathds{1}_{\left[{#1},{#2}\right]}\left({#3}\right)}

\newcommand*{\R}{\mathbb R}
\newcommand*{\Q}{\mathbb Q}
\newcommand*{\N}{\mathbb N}
\newcommand*{\Z}{\mathbb Z}
\newcommand*{\C}{\mathbb C}
\renewcommand*{\d}{\,\mathrm d}

\newcommand*{\nl}{\\[0.3em]}
\newcommand*{\bnl}{\\[1em]}

\newenvironment{comment}{\color{OliveGreen} \ \left\downarrow\ }{\right. \color{Black} \nonumber}

