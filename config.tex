\title{De Moivre–Laplace theorem}
\author{Stephan Kulla}

\allowdisplaybreaks
\renewcommand{\theenumi}{\alph{enumi}}

\theoremstyle{plain}
\newtheorem{lemma}{Lemma}[chapter]
\newtheorem{proposition}{Proposition}[chapter]
\newtheorem{theorem}{Theorem}[chapter]
\newtheorem{definition}{Definition}[chapter]

\theoremstyle{remark}
\newtheorem{remark}{Remark}[chapter]
\newtheorem{example}{Example}[chapter]

\renewcommand*{\epsilon}{\varepsilon}

\newcommand*{\Cmax}{0.4748}
\newcommand*{\Cmin}{0.40973}

\newcommand*{\refneeded}{\todo{Referenz!}}

\newcommand*{\includefig}[4][0.5\textwidth]{\begin{figure}[h] \begin{center} \includegraphics[width={#1}]{figures/{#2}} \caption[#4]{#3} \end{center} \end{figure}}

\newcommand*{\includeplot}[3][0.5\textwidth]{\begin{figure}[h] \label{#2} \begin{center} \includegraphics[width={#1}]{plots/{#2}} \caption{#3} \end{center} \end{figure}}

\newcommand*{\includewrapfig}[4][0.33\textwidth]{\begin{wrapfigure}{R}{#1} \vspace{-10pt} \begin{center} \includegraphics[width=#1 - 0.02\textwidth]{figures/{#2}} \caption[#4]{#3} \end{center} \vspace{-15pt} \end{wrapfigure}}

\newcommand*{\interval}[2]{\left[{#1},{#2}\right]}

\newcommand*{\seq}[2][n]{\left({#2}\right)_{{#1}\in\N}}
\newcommand*{\abs}[1]{\left|{#1}\right|}
\newcommand*{\norm}[1]{\left\|{#1}\right\|}
\newcommand*{\snorm}[1]{\left\|{#1}\right\|_\infty}
\newcommand*{\E}[1]{\mathbb E\left[{#1}\right]}
\renewcommand*{\P}[1]{\mathcal P\left({#1}\right)}
\newcommand*{\e}[1]{e^{#1}}
\newcommand*{\floor}[1]{\left\lfloor {#1} \right\rfloor}
\newcommand*{\bigo}[1]{O\left( {#1} \right)}
\newcommand*{\bigabs}[1]{\Delta\left( {#1} \right)}
\newcommand*{\littleo}[1]{o\left( {#1} \right)}
\newcommand*{\bigpsi}[1]{\Psi\left( {#1} \right)}
\newcommand*{\fsup}[1]{\sup\left( {#1} \right)}
\newcommand*{\fln}[1]{\ln\left( {#1} \right)}
\newcommand*{\fexp}[1]{\exp\left( {#1} \right)}
\newcommand*{\fcosh}[1]{\cosh\left( {#1} \right)}
\newcommand*{\fphi}[1]{\phi\left( {#1} \right)}
\newcommand*{\fPhi}[1]{\Phi\left( {#1} \right)}
\newcommand*{\br}[1]{\left( {#1} \right)}
\newcommand*{\fe}[1]{e^{#1}}
\newcommand*{\pol}[1]{\mathfrak P\left( {#1} \right)}

\newcommand*{\xup}[1]{\left\lceil {#1} \right\rceil_n}
\newcommand*{\xdown}[1]{\left\lfloor {#1} \right\rfloor_n}
\newcommand*{\xrnd}[1]{\operatorname{rnd}_n\left( {#1} \right)}
\newcommand*{\rnd}[1]{\operatorname{rnd}\left( {#1} \right)}

\newcommand*{\f}[1]{f\left({#1}\right)}
\newcommand*{\h}[1]{h\left({#1}\right)}
\newcommand*{\fp}[1]{f'\left({#1}\right)}
\newcommand*{\ft}[1]{\tilde f\left({#1}\right)}
\newcommand*{\pot}[1]{\mathcal P\left({#1}\right)}

\newcommand*{\cfs}[2][-\infty]{\mathds{1}_{\left[{#1},{#2}\right]}}
\newcommand*{\cf}[3][-\infty]{ \mathds{1}_{\left[{#1},{#2}\right]}\left({#3}\right)}

\newcommand*{\bn}[1][n]{\mathfrak{B}_{#1}} % bernoulli number
\newcommand*{\bps}[1][n]{\beta_{#1}} % symbol bernoulli polynomial
\newcommand*{\bp}[2][n]{\bps[#1]\left({#2}\right)} % bernoulli polynomial

\newcommand*{\an}[2][\epsilon]{\left[{#2} : {#1}\right]}
\newcommand*{\ean}[2][\epsilon]{\left\{{#2} : {#1}\right\}}

\newcommand*{\Bs}[1][n]{B_{#1}}
\newcommand*{\bs}[1][n]{b_{#1}}
\renewcommand*{\b}[2][n]{\bs[#1]\left({#2}\right)}
\newcommand*{\BBs}[1][n]{\tilde B_{#1}}
\newcommand*{\bbs}[1][n]{\tilde b_{#1}}
\newcommand*{\bb}[2][n]{\bbs[#1]\left({#2}\right)}
\newcommand*{\x}[1][k]{x_{#1}}

\newcommand*{\R}{\mathbb R}
\newcommand*{\Rplus}{\mathbb R^{+}}
\newcommand*{\Rplusnull}{\mathbb R^{+}_0}
\newcommand*{\Q}{\mathbb Q}
\newcommand*{\N}{\mathbb N}
\newcommand*{\Z}{\mathbb Z}
\newcommand*{\C}{\mathbb C}
\renewcommand*{\d}{\,\mathrm d}

\newcommand*{\nl}{\\[0.3em]}
\newcommand*{\bnl}{\\[1em]}

\newenvironment{comment}{\color{OliveGreen} \ \left\downarrow\ }{\right. \color{Black} \nonumber}

\newcommand*{\todo}[1]{{\color{RedOrange} \footnotesize{TODO: #1}}}
