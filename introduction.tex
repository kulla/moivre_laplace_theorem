\chapter{Introduction}

\section{The binomial distribution}

Probability theory emerges from the investigation of games of chance~\cite[p. 4]{hald1}. Since the antiquity people played dice games and starting from the 14th century card games became more and more popular~\cite[pp. 33-34]{hald1}. The government uses lotteries to finance their expenditures and there were a lot of private lotteries as well~\cite[p. 34]{hald1}\todo{check citations}.

In the 16th century Cardano started to discuss several problems about games of chance~\cite[pp. 33-41]{hald1} and this discussion was continued by mathematicians like Fermat, Pascal and Huygens in the 17th century~\cite[pp. 42-80]{hald1}. For example the odds and the expected winnings of different games were calculated. Among those discussed problem is the problem of division which states in its most general form \todo{ref}:

There are two players $A$ and $B$ which both need $s$ points to win. Player $A$ has the chance $p\in(0,1)$ to win a point in each round and player $B$ has the chance $q=1-p$. After some rounds with $A$ having $x$ and $B$ having $y$ points the game must be interrupted. How shall the stakes be fairly divided between $A$ and $B$? \todo{form}

This problem in the symmetrical case $p=\tfrac 12$ was first solved in a correspondence between Pascal and Fermat in 1654~\cite[pp. 54-63]{hald1} \todo{jahr richtig?}. Christiaan Huygens also presented this problem in several example problem in his treatise \emph{De ratiociniis in ludo alea} (On Reasoning in Games of Chance) of 1657~\cite[pp. 54-63]{hald1}. 


