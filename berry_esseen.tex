\chapter{The Berry-Esseen theorem}

In this chapter I present you the Berry-Essen theorem which not only proves the central limit but also gives an upper bound for the maximal error of the normal approximation. In the proof of this theorem I will mainly follow Nourdin and Peccati\cite[p. 71 ff.]{nourdin}, who use the Method of Stein.

\section{CLT and Berry-Esseen bounds}

From the central limit theorem we know, that each standardized sum $S_n = \tfrac 1{\sqrt n\sigma} \left(\sum_{k=1}^n x_k - n\mu\right)$ of i.i.d real valued random variables $\seq{X_n}$ with finite expected value $\mu=\E{X_1}$ and finite variance $\sigma^2=\E{X_1^2}-\E{X_1}^2$ converges in distribution to the standard normal distribution \refneeded. So we have $\lim_{n\to\infty} \P{S_n\le x} = \Phi(x)$ for all $x\in\R$. Thus we can use the normal distribution to approximate the cumulative distribution function of $S_n$.

Unfortunately not each proof of the central limit theorem provides an estimate of the error $\abs{\P{S_n\le x}-\Phi(x)}$ \refneeded. Here comes the Berry-Esseen theorem into play. It provides an upper bound for the maximal error between $\P{S_n \le x}$ and $\Phi(x)$. Therefore the Berry-Esseen theorem needs the finiteness of the third moment for each $X_n$ as an additional premise compared to the central limit theorem. The Berry-Esseen theorem states\cite[p. 71]{nourdin}:

\begin{theorem}[CLT and Berry-Esseen bounds]
  Let $\seq{X_n}$ be a sequence of standardized i.i.d real valued random variables with finite third moment. Thus we have $\E{X_1}=0$, $\E{X_1^2}=1$ and $\E{|X_1|^3} < \infty$. Let $S_n$ be the associated sequence of normalized partial sums, i.e

  \begin{align}
    S_n = \frac{1}{\sqrt n} \sum_{k=1}^n X_k
  \end{align}

  The partial sums $S_n$ converge in law to the standard normal distribution. Moreover there is a $C > 0$ not depending on $n$ nor $\seq{X_n}$ such that

  \begin{align}
    \sup_{x\in\R} \abs{\P{S_n\le x}-\Phi(x)} \le \frac{C \E{|X_1|^3}}{\sqrt n}
  \end{align}
\end{theorem}

\begin{remark}
  The currently best known estimate for the best choice of $C$ is $C \ge \Cmin$\cite{esseen1956} and $C \le \Cmax$\cite{shevtsova2011}.
\end{remark}

\begin{remark}[Berry-Esseen theorem for the binomial distribution]
  If for example $\seq{X_n}$ are Bernoulli trials with the probability of success $p \in (0,1)$ the third moment of all $X_n$ are finite. With $q=1-p$ we get \refneeded

\begin{align}
  \E{X_1}=\frac{p^2+q^2}{\sqrt{pq}}
\end{align}

Thus the Berry-Esseen theorem yields the following estimate for the normalized binomial distribution

\begin{align}
  \sup_{x\in\R} \abs{\P{S_n\le x}-\Phi(x)} \le \frac{C(p^2+q^2)}{\sqrt{npq}}
\end{align}

\noindent Thereby the best choice of $C$ fulfills $C \le \Cmax$.
\end{remark}

\section{The method of Stein}

\subsection{The basic idea of the method of Stein}

We need to estimate $\sup_{x\in\R} \abs{\P{S_n\le x}-\Phi(x)}$ to prove the Berry-Esseen theorem. To do so we first rewrite this supremum norm:

\begin{align}
   & \sup_{x\in\R} \abs{\P{S_n\le x}-\Phi(x)} \nl
   &
   \begin{comment}
     \text{Let } N \text{ have a standard normal distribution}
   \end{comment} \nl
  =& \sup_{x\in\R} \abs{\int_{-\infty}^x \d{S_n} - \int_{-\infty}^x \d{N}} \nl
  =& \sup_{x\in\R} \abs{\int_{\R} \cfs{x} \d{S_n} - \int_{\R} \cfs{x} \d{N}} \nl
  =& \sup_{x\in\R} \abs{\E{\cf{x}{S_n}} - \E{\cf{x}{N}}} \nl
   &
   \begin{comment}
     \text{Let }\mathcal H=\left\{\cfs{x} : x\in\R\right\}
   \end{comment} \nl
  =& \sup_{h\in\mathcal H} \abs{\E{h(S_n)}-\E{h(N)}}
\end{align}

Now we can use the method of Stein to transform the difference $\E{h(S_n)}-\E{h(N)}$ in a way that $N$ does not occur any more. Here we take the solution $f$ of the following so called \emph{Stein's equation}:

\begin{align}
  f'(x)=xf(x) + h(x) - \E{h(N)}
\end{align}

Let's assume that this ordinary differential equation has a solution $f$. We get for any random variable $Y$:

\begin{align}
  &            & f'(x) - xf(x) & = h(x) - \E{h(N)} \nl
  &  \implies\ & \int_{\R} (f'(t)-t f(t)) \d{Y}(t) &= \int_{\R} (h(t)-\E{h(N)}) \d{Y}(t) \nl 
  &  \implies\ & \E{f'(Y)-Y f(Y)} &= \E{h(Y)}-\E{h(N)}
\end{align}

So instead of $\abs{\E{h(S_n)}-\E{h(N)}}$ we can estimate $\abs{\E{f'(S_n)-S_n f(S_n)}}$ which does not contain $N$ anymore. Thereby we will use some properties of $f$ which follow from the Stein's equation, namely $\snorm{f} \le \sqrt{\tfrac \pi2}$ and $\snorm{f'} \le 2$.

Before I continue with the proof of the Berry-Esseen theorem I want to concrete what I have mentioned in this section.

\subsection{Stein's equation and its solutions}

For this and the following section we assume that $N$ is a random variable with a standard normal distribution. We have

\begin{definition}[Stein's equation]
  Let $h:\R \to \R$ be a Borel function such that $\E{h(N)}$ exists and $\abs{\E{h(N)}} < \infty$. The \emph{Stein's equation} associated to $h$ is the following ordinary differential equation

  \begin{align}
    f'(x) = x f(x) + h(x) - \E{h(N)}
  \end{align}
\end{definition}

The following proposition shows that the Stein's equation always has a solution and provides an explicit form for it:

\begin{proposition}[Solutions for the Stein's equation]
  Let $f_h$ be defined by

  \begin{align}
    f_h(x) = \e{\frac{x^2}{2}} \int_{-\infty}^x \Big[h(y)-\E{h(N)}\Big] \e{-\frac{y^2}{2}}\d{y}
  \end{align}

  \noindent Every solution to the Stein's equation has the form

  \begin{align}
    f(x) = c \e{\frac{x^2}{2}} + f_h(x)
  \end{align}

  \noindent with $c \in \R$.
\end{proposition}

\begin{proof}
   Because $\E{h(N)}$ exists, the function $y\mapsto h(y)\e{-\frac{y^2}{2}}$ is integrable so that also for each $x\in\R$ the integral ${\int_{-\infty}^x \Big[h(y)-\E{h(N)}\Big]\e{-\frac{y^2}{2}}\d{y}}$ exists. Thus $f_h$ is well defined. From the Stein's equation follows

  \begin{align}
    &      & f'(x)-xf(x) & = h(x) - \E{h(N)} \nl
    & \iff & \left[ f'(x) - xf(x) \right] \e{-\frac{x^2}{2}} &= \Big[h(x) - \E{h(N)}\Big] \e{-\frac{x^2}{2}} \nl
    & \iff & \partial_x \left( \e{-\frac{x^2}2} f(x) \right) &= \Big[h(x) - \E{h(N)}\Big] \e{-\frac{x^2}{2}} \nl
    & \iff & \e{-\frac{x^2}{2}} f(x) & = \int_{-\infty}^x \Big[ h(y) - \E{h(N)} \Big] \e{-\frac{y^2}{2}} \d{y} + c \nl
    & \iff & f(x) & = f_h(x) + c \e{\frac{x^2}{2}}
  \end{align}
\end{proof}

In the proof of the Berry-Esseen theorem we will need a estimate of $\snorm{f_h}$ and $\snorm{f_h'}$. Therefore we need to restrict the values of $h$ in the interval $h$. After this restriction we get

\begin{proposition}[Stein's bounds]
  Let $h:\R \to [0,1]$ be a Borel-function such that $\E{h(N)}$ exists with $\E{h(N)}<\infty$. The solution $f_h$ of Stein's equation given in the above proposition fulfills

  \begin{align}
    \snorm{f_h} \le \sqrt{\frac \pi2} \text{ and } \snorm{f_h'} \le 2
  \end{align}
\end{proposition}

\begin{proof}
  We start with proving $\snorm{f_h}\le \sqrt{\frac{\pi}{2}}$. First get the estimate

  \begin{align} \label{eq:estimate1}
    \abs{f_h(x)} & \le \abs{\e{-\frac{x^2}{2}} \int_{-\infty}^x \Big[ h(y) - \E{h(N)} \Big] \e{-\frac{y^2}{2}} \d{y} } \nl 
    &\le \e{\frac{x^2}{2}} \int_{-\infty}^x \abs{h(y)-\E{h(N)}} \e{-\frac{y^2}{2}}\d{y} \nl
    &
    \begin{comment}
      h(y), \E{h(N)} \in [0,1] \implies \abs{h(y)-\E{h(N)}} \le 1
    \end{comment} \nl
    & \le \e{\frac{x^2}{2}} \int_{-\infty}^x \e{-\frac{y^2}{2}} \d{y}
  \end{align}
  
  \noindent We also have

  \begin{align}
    \int_{-\infty}^x \Big[h(y)-\E{h(N)}\Big] \e{-\frac{y^2}{2}} \d{y} = -\int_{x}^\infty \Big[h(y)-\E{h(N)}\Big] \e{-\frac{y^2}{2}} \d{y}
  \end{align}

  \noindent because $\int_{\R} \Big[h(y)-\E{h(N)}\Big] \e{-\frac{y^2}{2}} \d{y}=0$:

  \begin{align} \label{eq:estimate2}
    \int_{\R} \Big[ h(y)-\E{h(N)} \Big] \e{-\frac{y^2}{2}} \d{y} & = \int_{\R} h(y) \e{-\frac{y^2}{2}} \d{y} - \int_{\R} \E{h(N)} \e{-\frac{y^2}2} \d{y} \nl
    &= \sqrt{2\pi} \int_{\R} h(y) \frac{1}{2\pi} \e{-\frac{y^2}{2}} \d{y} - \sqrt{2\pi} \E{h(N)} \int_{\R} \frac{1}{\sqrt{2\pi}} \e{-\frac{y^2}{2}} \d{y} \nl
    &= \sqrt{2\pi} \int_{\R} h(y) \d{N(y)} - \sqrt{2\pi} \E{h(N)} \int_{\R} \d{N(y)} \nl
    &= \sqrt{2\pi}\E{h(N)} - \sqrt{2\pi}\E{h(N)} \nl
    &= 0
  \end{align}

  \noindent Thus we also could estimate $\abs{f_h(x)}$ via

  \begin{align}
    \abs{f_h(x)} & \le \abs{\e{-\frac{x^2}{2}} \int_{-\infty}^x \Big[ h(y) - \E{h(N)} \Big] \e{-\frac{y^2}{2}} \d{y} } \nl 
    & \le \abs{-\e{-\frac{x^2}{2}} \int_{x}^\infty \Big[ h(y) - \E{h(N)} \Big] \e{-\frac{y^2}{2}} \d{y} } \nl 
    &\le \e{\frac{x^2}{2}} \int_{x}^\infty \abs{h(y)-\E{h(N)}} \e{-\frac{y^2}{2}}\d{y} \nl
    &\le \e{\frac{x^2}{2}} \int_{x}^\infty \e{-\frac{y^2}{2}} \d{y}
  \end{align}
 
  \noindent From \eqref{eq:estimate1} and \eqref{eq:estimate2} follows

  \begin{align}
    \abs{f_h(x)} & \le \min\left\{ \e{\frac{x^2}{2}} \int_{-\infty}^x \e{-\frac{y^2}{2}} \d{y}, \e{\frac{x^2}{2}} \int_x^\infty \e{-\frac{y^2}{2}} \d{y} \right\} \nl
    & \le \e{\frac{x^2}{2}} \min\left\{ \int_{-\infty}^x \e{-\frac{y^2}{2}} \d{y}, \int_x^\infty \e{-\frac{y^2}{2}} \d{y} \right\} \nl
    &
    \begin{comment}
      y \mapsto \e{-\frac{y^2}{2}} \text{ is even}
    \end{comment} \nl
    & = \underbrace{\e{\frac{x^2}{2}} \int_{\abs{x}}^\infty \e{-\frac{y^2}{2}} \d{y}}_{=: s(x)} \nl
  \end{align}

  \noindent To find the maximum of $s(x)$ we first compute $s'(x)$ for $x > 0$:

  \begin{align}
    s'(x) &= \partial_x \left( \e{\frac{x^2}{2}} \int_x^\infty \e{-\frac{y^2}{2}} \d{y} \right) \nl
    &= x\e{\frac{x^2}{2}} \int_x^\infty \e{-\frac{y^2}{2}} \d{y} - 1 \nl
    &= \e{\frac{x^2}{2}} \int_x^\infty x\e{-\frac{y^2}{2}} \d{y} - 1 \nl
    &
    \begin{comment}
      x \le y \text{ on } [x,\infty)
    \end{comment} \nl
    &\le \e{\frac{x^2}{2}} \int_x^\infty y\e{-\frac{y^2}{2}} \d{y} - 1 \nl
    &= \e{\frac{x^2}{2}} \left[-\e{-\frac{y^2}{2}}\right]_x^\infty \d{y} - 1 \nl
    &= \e{\frac{x^2}{2}} \e{-\frac{x^2}{2}} - 1 \nl
    &= 0
  \end{align}

  Thus $s'(x) \le 0$ for $x \ge 0$ so that $s(x)$ is monotone decreasing on $[x,\infty)$. Because $s(x)$ is even (i.e. $s(-x)=s(x)$) we have $s'(x) \ge 0$ for $x < 0$ which implies that $s(x)$ is monotone increasing on $\R^{-}_0$. Altogether $s(x)$ has its global maximum at $x=0$ which is

  \begin{align}
    s(0) = \int_0^\infty \e{-\frac{y^2}{2}} \d{y} = \frac{1}{2} \int_{-\infty}^\infty \e{-\frac{y^2}{2}} \d{y} = \frac{1}{2} \sqrt{2\pi} = \sqrt{\frac{\pi}{2}}
  \end{align}

  \noindent Finally we have

  \begin{align}
    \abs{f_h(x)} \le s(x) \le \sqrt{\frac \pi 2}
  \end{align}

  \noindent To prove the second estimate $\snorm{f_h'} \le 2$ we can start with Stein's equation:

  \begin{align}
    \abs{f_h'(x)} & = \abs{x f_h(x) + h(x) - \E{h(N)}} \nl
    & = \abs{x \e{\frac{x^2}{2}} \int_{-\infty}^x \Big[ h(y) - \E{h(N)} \Big] \e{-\frac{y^2}{2}} \d{y}} \nl
    & \le \abs{x} \e{\frac{x^2}{2}} \int_{-\infty}^x \abs{h(y)-\E{h(N)}} \e{-\frac{y^2}{2}} \d{y}  + \abs{h(y)-\E{h(y)}} \nl
    &
    \begin{comment}
      \abs{h(y)-\E{h(N)}} \le 1
    \end{comment} \nl
    & \le \abs{x} \e{\frac{x^2}{2}} \int_{-\infty}^x \e{-\frac{y^2}{2}} \d{y}  + 1 \nl
  \end{align}

  \noindent Again we can use the following equation in the estimation of $\abs{f_h'(x)}$

  \begin{align}
    \int_{-\infty}^x \Big[h(y)-\E{h(N)}\Big] \e{-\frac{y^2}{2}} \d{y} = -\int_{x}^\infty \Big[h(y)-\E{h(N)}\Big] \e{-\frac{y^2}{2}} \d{y}
  \end{align}

  \noindent No we get

  \begin{align}
    \abs{f_h'(x)} \le \abs{x} \e{\frac{x^2}{2}} \int_x^\infty \e{-\frac{y^2}{2}} \d{y}  + 1 \nl
  \end{align}

  \noindent Altogether we have

  \begin{align}
    \abs{f_h'(x)} & \le \abs{x} \e{\frac{x^2}{2}} \min\left\{ \int_{-\infty}^x \e{-\frac{y^2}{2}} \d{y},  \int_x^\infty \e{-\frac{y^2}{2}} \d{y} \right\} + 1 \nl
    & = \abs{x} \e{\frac{x^2}{2}} \int_{\abs{x}}^\infty \e{-\frac{y^2}{2}} \d{y} + 1 \nl
    & = \e{\frac{x^2}{2}} \int_{\abs{x}}^\infty \abs{x} \e{-\frac{y^2}{2}} \d{y} + 1 \nl
    &
    \begin{comment}
      \abs{x} \le y \text{ on interval } \left[\abs{x},\infty\right)
    \end{comment} \nl
    & = \e{\frac{x^2}{2}} \int_{\abs{x}}^\infty y \e{-\frac{y^2}{2}} \d{y} + 1 \nl
    & = \e{\frac{x^2}{2}} \e{-\frac{x^2}{2}} + 1 \nl
    & = 2 \nl
  \end{align}

  \noindent This proves $\snorm{f_h'} \le 2$.
\end{proof}

\section{Proof of the Berry-Esseen theorem}

\subsection{First steps of the proof}

Let's recall the Berry-Esseen theorem: For the standardized partial sum $S_n$ of i.i.d random variables $\seq{X_n}$ with finite third moment $\E{\abs{X_n}^3}<\infty$ there is a $C > 0$ such that

\begin{align}
  \sup_{x\in\R} \abs{\P{S_n\le x}-\Phi(x)} \le \frac{C \E{|X_1|^3}}{\sqrt n}
\end{align}

\noindent We first define $C_n$ via

\begin{align}
  C_n = \frac{\sqrt{n}}{\E{\abs{X_1}^3}} \sup_{x\in\R} \abs{\P{S_n \le x}-\Phi(x)}
\end{align}

\noindent Thus $C_n$ is the smallest real number with

\begin{align}
  \sup_{x\in\R}\abs{\P{S_n\le x} - \Phi(x)} \le \frac{C_n\E{\abs{X_1}^3}}{\sqrt{n}}
\end{align}

To prove the Berry-Essen theorem we have to show, that $\sup_{n\in\N} C_n < \infty$. Let $\mathcal H$ be the set of all $\cfs{x}$ for $x\in\R$. We already have shown in the section about Stein's method that

\begin{align}
  \sup_{x\in\R}\abs{\P{S_n\le x} - \Phi(x)} & = \sup_{h\in\mathcal H} \abs{\E{h(S_n)}-\E{h(N)}} \nl
  &= \sup_{h\in\R} \abs{\E{f_h'(S_n)-S_n f_h(S_n)}}
\end{align}

Unfortunately the functions of $\mathcal H$ are not continuous and thus $f_h$ is no $C^1$-function for any $h\in \mathcal H$. That's why we introduce the functions $h_{\epsilon,x}$ which are defined by

\includefig{continuous_step_function}{The function $h_{\epsilon,x}$}{The function $h_{epsilon,x}$}

\noindent For any random variable $Y$ it is

\begin{align} \label{berryproof:inequality1}
  \E{h_{x-\epsilon,\epsilon}(Y)} \le \P{Y \le x} \le \E{h_{x+\epsilon,\epsilon}(Y)}
\end{align}

Because the density function $\phi(t)$ of $N$ with the standard normal distribution is bounded by $\tfrac{1}{\sqrt{2\pi}}$ we have:

\begin{align}
  \E{h_{x+\epsilon,\epsilon}(N)}-\frac{4\epsilon}{\sqrt{2\pi}} \le \E{h_{x-\epsilon,\epsilon}(N)} \text{ and } \E{h_{x+\epsilon,\epsilon}(N)} \le \E{h_{x-\epsilon,\epsilon}(N)} + \frac{4\epsilon}{\sqrt{2\pi}}
\end{align}

\noindent Together with \eqref{berryproof:inequality1} we get

\begin{align} \label{berryproof:inequality2}
  \E{h_{x+\epsilon,\epsilon}(N)}-\frac{4\epsilon}{\sqrt{2\pi}} \le \P{N\le x} \le \E{h_{x-\epsilon,\epsilon}(N)} + \frac{4\epsilon}{\sqrt{2\pi}}
\end{align}

\noindent When we combine \eqref{berryproof:inequality1} and \eqref{berryproof:inequality2} we get the estimate

\begin{align}
  \E{h_{x-\epsilon,\epsilon}(S_n)} - \E{h_{x-\epsilon,\epsilon}(N)} - \frac{4\epsilon}{\sqrt{2\pi}} & \le \P{S_n \le x} - \P{N \le x} \nl
  &\le \E{h_{x+\epsilon,\epsilon}(S_n)} - \E{h_{x+\epsilon,\epsilon}(N)} + \frac{4\epsilon}{\sqrt{2\pi}}
\end{align}

\noindent Thus

\begin{align}
  \abs{\P{S_n \le x}-\P{N \le x}} \le \max\left\{
    \begin{array}{l}
      \E{h_{x-\epsilon,\epsilon}(S_n)} - \E{h_{x-\epsilon,\epsilon}(N)}, \nl
      \E{h_{x+\epsilon,\epsilon}(S_n)} - \E{h_{x+\epsilon,\epsilon}(N)}
    \end{array}
  \right\} + \frac{4\epsilon}{\sqrt{2\pi}}
\end{align}

\noindent So we get the estimate

\begin{align} \label{berryproof:before_stein}
  \sup_{x\in\R} \abs{\P{S_n \le x} - \P{N \le x}} \le \sup_{x\in\R} \Big\{ \abs{\E{h_{x,\epsilon}(S_n)} - \E{h_{x,\epsilon}(N)}} \Big\} + \frac{4\epsilon}{\sqrt{2\pi}}
\end{align}

For estimating $\abs{\E{h_{x,\epsilon}(S_n)} - \E{h_{x,\epsilon}(N)}}$ we can now use the method of Stein. Because $h_{x,\epsilon}$ is continuous its solution to the Stein's equation is a $C^1$-function.

\subsection{Application of the method of Stein}

In the following I set $h=h_{x,\epsilon}$. Let $f=f_h$ be the solution to the Stein's equation for $h=h_{x,\epsilon}$. Besides let $n \ge 2$ and let $\tilde f(x)=xf(x)$. I also define $S_n^k$ via

\begin{align}
  S_n^k = \frac 1{\sqrt{n}} \sum_{j\neq k} Y_j = S_n - \frac 1{\sqrt n} Y_k
\end{align}

\noindent In the section about the method of Stein we have seen

\begin{align}
  \E{h(S_n)}-\E{h(N)}=\E{f'(S_n)-S_n f(S_n)}
\end{align}

\noindent The later term can be rewritten as

\begin{align}
  \E{h(S_n)}-\E{h(N)} &= \E{ \fp{S_n} -S_nf(S_h)} \nl
  & \begin{comment}
    S_n = \frac 1{\sqrt n} \sum_{k=1}^n Y_k
  \end{comment} \nl
  &= \E{f'(S_n)-\sum_{k=1}^n \frac{Y_k}{\sqrt n} f(S_n)} \nl
  &= \sum_{k=1}^n \E{f'(S_n)\frac 1n-\frac{Y_k}{\sqrt n} f(S_n)} \nl
  & \begin{comment}
    \E{ \f{S_n^k} Y_k } = \E{ \f{S_n^k} }\E{Y_k} = 0
  \end{comment} \nl
  &= \sum_{k=1}^n \E{ f'(S_n) \frac 1n- \frac{Y_k}{\sqrt n} f(S_n) + \frac{Y_k}{\sqrt n} \f{S_n^k }} \nl
  &= \sum_{k=1}^n \E{ f'(S_n) \frac 1n - \frac{Y_k}{\sqrt n} \left( f(S_n)-\f{S_n^k} \right)} \nl
  &
  \begin{comment}
    \begin{aligned}
  f(y)-f(x) &= \int_x^y f'(s) \d{s} \nl
  &= (y-x) \int_0^1 f'(x+(y-x)s) \d{s} \nl
  &
  \begin{comment}
    \text{\parbox{0.4\textwidth}{
      Choose $\Theta$ uniformly distributed in $[0,1]$ and independent of all $Y_1,\ldots,Y_k$
    }}
  \end{comment} \nl
  &= (y-x)\ \E{ \fp{x+(y-x)} \Theta }
    \end{aligned}
  \end{comment} \nl
  &= \sum_{k=1}^n \E{ f'(S_n) \frac 1n - \frac{Y_k^2}{n} \E{ \fp{ S_n^k + \Theta\frac{Y_k}{\sqrt n} }}} \nl
  &
  \begin{comment}
    \begin{aligned}
      \E{ \frac{Y_k^2}{n} \E{ \fp{ S_n^k + \Theta\frac{Y_k}{\sqrt n} }}} &= \frac{\E{Y_k^2}}{n}  \E{ \ft{S_n^k+\Theta\frac{Y_k}{\sqrt n} }} \nl
&= \frac{1}{n} E\left[f'\left(S_n^k+\Theta\frac{Y_k}{\sqrt n}\right)\right]
    \end{aligned}
  \end{comment} \nl
  &= \frac 1n \sum_{k=1}^n \E{ f'(S_n) - \fp{S_n^k+\Theta\frac{Y_k}{\sqrt n}}} \nl
  &\begin{comment}
    f'(x)=\tilde f(x) + h(x)-E[h(N)]
  \end{comment} \nl
  &= \frac 1n \sum_{k=1}^n \E{
    \begin{multlined}
      \tilde f(S_n) + h(S_n) - E[h(N)] \nl
      -\ft{S_n^k +\Theta \frac{Y_k}{\sqrt n} } - \h{ S_n^k +\Theta \frac{Y_k}{\sqrt n} } + E[h(N)]
    \end{multlined}
  } \nl
  &= \frac 1n \sum_{k=1}^n \E{
  \begin{multlined}
      \tilde f(S_n) + h(S_n) \nl
      -\ft{ S_n^k +\Theta \frac{Y_k}{\sqrt n}} - \h{S_n^k +\Theta \frac{Y_k}{\sqrt n}}
  \end{multlined}
  } \nl
  &= \frac 1n \sum_{k=1}^n \E{
    \begin{multlined}
      \ft{S_n} - \ft{S_n^k} \nl
      - \ft{ S_n^k +\Theta \frac{Y_k}{\sqrt n} } + \ft{ S_n^k } \nl
      + \h{S_n} - \h{ S_n^k } \nl
      - \h{ S_n^k +\Theta \frac{Y_k}{\sqrt n} } + \h{ S_n^k }
    \end{multlined}
  }
\end{align}

\noindent Thus

\begin{align}
  \abs{\E{h(S_n)}-\E{h(N)}}
  &= \abs{\frac 1n \sum_{k=1}^n \E{
    \begin{multlined}
      \ft{S_n} - \ft{S_n^k} \nl
      - \ft{ S_n^k +\Theta \frac{Y_k}{\sqrt n} } + \ft{ S_n^k } \nl
      + \h{S_n} - \h{ S_n^k } \nl
      - \h{ S_n^k +\Theta \frac{Y_k}{\sqrt n} } + \h{ S_n^k }
    \end{multlined}
  }} \nl
  &\le \frac 1n \sum_{k=1}^n \left(
  \begin{multlined}
      \E{\abs{\ft{S_n} - \ft{S_n^k}}} \nl
      +\E{\abs{\ft{ S_n^k +\Theta \frac{Y_k}{\sqrt n} } - \ft{ S_n^k }}} \nl
      +\E{\abs{\h{S_n} - \h{ S_n^k }}} \nl
      +\E{\abs{\h{ S_n^k +\Theta \frac{Y_k}{\sqrt n} } - \h{ S_n^k }}}
    \end{multlined}
  \right)
\end{align}

\noindent So we have four terms which we will estimate respectively

\subsection{Estimation of all four terms}

\subsubsection{Estimation of the first term}

To estimate the first term we will use the fact, that $\snorm{f} \le \sqrt{\tfrac \pi2}$ and $\snorm{f'} \le 2$:

\begin{align}
  &\E{\abs{ \ft{S_n} - \ft{S_n^k} }} \nl
  &
  \begin{comment}
    \begin{aligned}
      |\tilde f(x) - \tilde f(y)| & = |xf(x)-yf(y)| \nl
      & =|xf(x)-yf(x)+yf(x)-yf(y)| \nl
      & =|f(x)(x-y) + y(f(x)-f(y))| \nl
      & \le |f(x)||x-y|+|y||f(x)-f(y)| \nl
      & \le \snorm{f} |x-y|+ |y| \snorm{f'} |x-y| \nl
      & \le \sqrt{\frac \pi2} |x-y|+2 |y| |x-y| \nl
      & = \left( \sqrt{\frac \pi2} + 2|y| \right) |x-y|
    \end{aligned}
  \end{comment} \nl
  &\le \E{ \left( \sqrt{\frac \pi2}+2 \abs{S_n^k} \right) \abs{S_n-S_n^k} } \nl
  &\le \E{ \left( \sqrt{\frac \pi2}+2 \abs{S_n^k} \right) \frac{|Y_k|}{\sqrt n} } \nl
  &\le \frac 1{\sqrt n} \left( \sqrt{\frac \pi2}+2 \E{\abs{S_n^k}} \right) \E{\abs{Y_k}} \nl
  &
  \begin{comment}
    \begin{aligned}
      \E{\abs{Y_k}}   & \le \sqrt{\E{Y_k^2}} = 1 \nl
      \E{\abs{S_n^k}} & \le \sqrt{\E{{S_n^k}^2}} \le \sqrt{\E{\frac{n}{n-1}{S_n^k}^2}} = 1
    \end{aligned}
  \end{comment} \nl
  &\le \frac 1{\sqrt n} \left( \sqrt{\frac \pi2}+2 \right)
\end{align}

\subsubsection{Estimation of the second term}

The second term can be estimated analogy to the first term:

\begin{align}
  \E{\abs{ \ft{ S_n^k + \Theta \frac{Y_k}{\sqrt n} } - \ft{ S_n^k } }} & \le \frac{ \E{\abs{Y_k}} }{\sqrt n} \left( \E{\Theta} \sqrt{\frac \pi2} + 2 \E{\Theta} \E{\abs{S_n^k}} \right) \nl
  &
  \begin{comment}
    \E{\Theta} = \frac 12 \text{ and } \E{\abs{S_n^k}} \le 1
  \end{comment} \nl
  & \le \frac{ \E{\abs{Y_k}} }{\sqrt n} \left(\frac 12\sqrt{\frac \pi2}+1\right) \nl
  &\begin{comment}
      \E{\abs{Y_k}} \le \sqrt{\E{Y_k^2}} = 1
  \end{comment} \nl
  & \le \frac{1}{\sqrt n} \left(\frac 12\sqrt{\frac \pi2}+1\right)
\end{align}

\subsubsection{Estimation of the third term}

\begin{align}
  & \E{\abs{ \h{S_n} - \h{ S_n^k }}} \nl
  &
  \begin{comment}
    \begin{aligned}
      h(y)-h(x) & = \int_y^x h'(s) \d{s} \nl
      &= (y-x) \int_0^1 h'(x+s(y-x)) \d{s} \nl
      &
      \begin{comment}
        \text{\parbox{0.45\textwidth}{
          Choose $\tilde \Theta$ uniformly distributed in $[0,1]$ and independent of all $Y_1,\ldots,Y_n$ and $\Theta$.
        }}
      \end{comment} \nl
      &= -\frac{y-x}{2\epsilon} \E{ \cf[z-\epsilon]{z+\epsilon}{x + \tilde \Theta (y-x)} }
    \end{aligned}
  \end{comment} \nl
  &= \frac{1}{2\epsilon} \E{ \abs{ S_n-S_n^k} \E{ \cf[z-\epsilon]{z+\epsilon}{S_n^k + \tilde \Theta \left( S_n-S_n^k \right)} }} \nl
  &= \frac{ \E{\abs{Y_k}} }{2\sqrt{n}\epsilon} \E{ \cf[z-\epsilon]{z+\epsilon}{S_n^k + \tilde \Theta \frac{Y_k}{\sqrt n} }} \nl
  &
  \begin{comment}
    \E{\abs{Y_k}} \le 1
  \end{comment} \nl
  &\le \frac{1}{2\sqrt{n}\epsilon} \E{ \cf[z-\epsilon]{z+\epsilon}{S_n^k + \tilde \Theta \frac{Y_k}{\sqrt n} }} \nl
  &= \frac{1}{2\sqrt{n}\epsilon} \P{z-\epsilon \le S_n^k + \tilde \Theta \frac{Y_k}{\sqrt n} \le z+\epsilon } \nl
  &
  \begin{comment}
    \tilde \Theta\frac{Y_k}{\sqrt n} \in \R
  \end{comment} \nl
  &\le \frac{1}{2\sqrt{n}\epsilon} \sup_{c\in\R} \P{z-c-\epsilon \le S_n^k \le z-c+\epsilon}
\end{align}

To estimate $\P{ a\le S_n^k\le b }$ we set $\tilde S_n^k = \tfrac{1}{\sqrt{n-1}} \sum_{j\neq k} Y_j$. $\tilde S_n^k$ is the standardized sum of all $Y_1,\ldots,Y_k$ without $Y_k$. We have $S_n^k = \sqrt{1-\tfrac 1n} \tilde S_n^k$. Thus

\begin{align}
  \P{a\le S_n^k\le b} &= \P{a \le \sqrt{1-\tfrac 1n} \tilde S_n^k \le b} \nl
  &= \P{\frac{a}{\sqrt{1-\tfrac 1n}} \le \tilde S_n^k \le \frac{b}{\sqrt{1-\tfrac 1n}}} \nl
  &
  \begin{multlined}
    = \P{\frac{a}{\sqrt{1-\tfrac 1n}} \le \tilde S_n^k \le \frac{b}{\sqrt{1-\tfrac 1n}}}- \P{\frac{a}{\sqrt{1-\tfrac 1n}} \le N \le \frac{b}{\sqrt{1-\tfrac 1n}}} \nl
    +\P{\frac{a}{\sqrt{1-\tfrac 1n}} \le N \le \frac{b}{\sqrt{1-\tfrac 1n}}}
  \end{multlined} \nl
  &
  \begin{comment}
    \P{\frac{a}{\sqrt{1-\tfrac 1n}} \le N \le \frac{b}{\sqrt{1-\tfrac 1n}}} \le \frac{b-a}{\sqrt{2\pi}\sqrt{1-\tfrac 1n}}
  \end{comment} \nl
  &
  \begin{multlined}
    \le \P{\frac{a}{\sqrt{1-\tfrac 1n}} \le \tilde S_n^k \le \frac{b}{\sqrt{1-\tfrac 1n}}}- \P{\frac{a}{\sqrt{1-\tfrac 1n}} \le N \le \frac{b}{\sqrt{1-\tfrac 1n}}} \nl
    +\frac{b-a}{\sqrt{2\pi}\sqrt{1-\tfrac 1n}}
  \end{multlined} \nl
  &
  \begin{comment}
    \begin{aligned}
      & \P{x \le \tilde S_n^k \le y} - \P{x\le N \le y} \nl
      &
      \begin{multlined}
        = \P{\tilde S_n^k \le y} - \P{\tilde S_n^k < x} \nl
        - \P{N \le y} + \P{N < x}
      \end{multlined} \nl
      &
      \begin{multlined}
        = \P{\tilde S_n^k \le y} - \P{N \le y} \nl
        + \P{N < x} - \P{\tilde S_n^k < x}
      \end{multlined} \nl
      &\le 2 \sup_{z\in\R} \abs{\P{\tilde S_n^k \le z}-\Phi(z)} \nl
      &= \frac{2 C_{n-1} \E{\abs{Y_1}^3}}{\sqrt{n-1}}
     \end{aligned}
  \end{comment} \nl
  & \le \frac{2 C_{n-1} \E{\abs{Y_1}^3}}{\sqrt{n-1}} + \frac{b-a}{\sqrt{2\pi}\sqrt{1-\tfrac 1n}}
\end{align}

\noindent Thus

\begin{align}
  \E{\abs{h(S_n) - h(N)}} & \le \frac{1}{2\sqrt{n}\epsilon} \sup_{c\in\R} \P{ z-c-\epsilon \le S_n^k \le z-c+\epsilon} \nl
  &\le \frac{2C_{n-1} \E{\abs{Y_1}^3}}{2\sqrt n\sqrt{n-1}\epsilon}+\frac{2\epsilon}{2\epsilon\sqrt{2\pi}\sqrt{n-1}} \nl
  &= \frac{C_{n-1} \E{\abs{Y_1}^3}}{\sqrt n\sqrt{n-1}\epsilon}+\frac{1}{\sqrt{2\pi}\sqrt{n-1}} \nl
\end{align}

\subsubsection{Estimation of the fourth term}

Also the fourth term can be estimated analogy to the third term:

\begin{align}
  \E{\abs{ \h{ S_n^k+\Theta\frac{Y_k^2}{n} } - \h{ S_n^k } }}
  &= \frac{1}{2\sqrt n \epsilon} \E{ \abs{Y_k} \abs{\Theta} \E{ \cf[z-\epsilon]{z+\epsilon}{S_n^k + \tilde \Theta\Theta\frac{Y_k}{\sqrt n}} } } \nl
  &
  \begin{comment}
    \E{\abs{Y_k}} \le 1 \text{ and } \E{\Theta} = \frac 12
  \end{comment} \nl
  &= \frac{1}{4\sqrt n \epsilon} \E{ \cf[z-\epsilon]{z+\epsilon}{S_n^k + \tilde \Theta\Theta\frac{Y_k}{\sqrt n}} } \nl
  & \le \frac{1}{4\sqrt n \epsilon} \sup_{c\in\R} \P{z-c-\epsilon \le S_n^k \le z-c+\epsilon} \nl
  &\le \frac{1}{2\sqrt{2\pi}\sqrt{n-1}} + \frac{C_{n-1} \E{\abs{Y_1}^3} }{2\sqrt  n\sqrt{n-1}\epsilon}
\end{align}

\subsection{Derivation of recursive relation for $C_n$}

\begin{align}
  \abs{\E{h(S_n)}-\E{h(N)}}
  &\le \frac 1n \sum_{k=1}^n \left(
  \begin{multlined}
      \E{\abs{\ft{S_n} - \ft{S_n^k}}} \nl
      +\E{\abs{\ft{ S_n^k +\Theta \frac{Y_k}{\sqrt n} } - \ft{ S_n^k }}} \nl
      +\E{\abs{\h{S_n} - \h{ S_n^k }}} \nl
      +\E{\abs{\h{ S_n^k +\Theta \frac{Y_k}{\sqrt n} } - \h{ S_n^k }}}
    \end{multlined}
  \right) \bnl
  &\le \frac 1n \sum_{k=1}^n \left(
    \begin{multlined}
      \left(\sqrt{\frac\pi2}+2\right) \frac1{\sqrt n} +\left(\frac 12\sqrt{\frac \pi2}+1\right)\frac{1}{\sqrt n}\nl
      +\frac{1}{\sqrt{2\pi}\sqrt{n-1}}+\frac{C_{n-1} \E{\abs{Y_1}^3} }{\sqrt{n}\sqrt{n-1}\epsilon}\nl
      +\frac{1}{2\sqrt{2\pi}\sqrt{n-1}}+\frac{C_{n-1} \E{\abs{Y_1}^3} }{2\sqrt{n}\sqrt{n-1}\epsilon}
    \end{multlined}
  \right) \bnl
  &
  \begin{multlined}
    = \left(\sqrt{\frac\pi2}+2\right) \frac1{\sqrt n} +\left(\frac 12\sqrt{\frac \pi2}+1\right)\frac{1}{\sqrt n}\nl
    +\frac{1}{\sqrt{2\pi}\sqrt{n-1}}+\frac{C_{n-1} \E{\abs{Y_1}^3} }{\sqrt{n}\sqrt{n-1}\epsilon}\nl
    +\frac{1}{2\sqrt{2\pi}\sqrt{n-1}}+\frac{C_{n-1} \E{\abs{Y_1}^3} }{2\sqrt{n}\sqrt{n-1}\epsilon}
  \end{multlined} \bnl
  & = \frac{1}{\sqrt{n}} \left(
  \begin{multlined}
    \sqrt{\frac\pi2} + 2 + \frac 12 \sqrt{\frac \pi2} + 1 +\frac{3\sqrt n}{2\sqrt{2\pi}\sqrt{n-1}} \nl
    + \frac{3C_{n-1} \E{\abs{Y_1}^3} }{2\sqrt{n-1}\epsilon}
  \end{multlined}
  \right) \bnl
  &
  \begin{comment}
    \E{\abs{Y_1}^3} \ge \E{Y_1^2}^{\tfrac 32} = 1
  \end{comment} \bnl
  & = \frac{ \E{\abs{Y_1}^3}  }{\sqrt{n}} \left(
  \begin{multlined}
    \sqrt{\frac\pi2} + 2 + \frac 12 \sqrt{\frac \pi2} + 1 +\frac{3\sqrt n}{2\sqrt{2\pi}\sqrt{n-1}} \nl
    + \frac{3C_{n-1} \E{\abs{Y_1}^3} }{2\sqrt{n-1}\epsilon}
  \end{multlined}
  \right) \bnl
  &
  \begin{comment}
    n \ge 2 \implies n-1 \ge \frac n2 \implies \sqrt{n-1} \ge \frac{\sqrt{n}}{\sqrt 2} 
  \end{comment} \bnl
  & = \frac{ \E{\abs{Y_1}^3}  }{\sqrt{n}} \left(
  \begin{multlined}
    \sqrt{\frac\pi2} + 2 + \frac 12 \sqrt{\frac \pi2} + 1 +\frac{3\sqrt 2}{2\sqrt{2\pi}} \nl
    + \frac{3 \sqrt{2} C_{n-1} \E{\abs{Y_1}^3} }{2\sqrt{n}\epsilon}
  \end{multlined}
  \right) \bnl
  &
  \begin{comment}
    \frac{\sqrt 2}2 \le 1
  \end{comment} \bnl
  & = \frac{ \E{\abs{Y_1}^3}  }{\sqrt{n}} \left( 5.726\ldots + \frac{3 C_{n-1} \E{\abs{Y_1}^3} }{\sqrt{n}\epsilon}
  \right) \bnl
  &\le \frac{6 \E{\abs{Y_1}^3} }{\sqrt n} + \frac{3 C_{n-1} \E{\abs{Y_1}^3}^2}{n\epsilon}
\end{align}

\noindent This inequality can be used in \eqref{berryproof:before_stein}:

\begin{align}
  \sup_{x\in\R} \abs{\P{S_n \le x} - \P{N \le x}} & \le \sup_{x\in\R} \Big\{ \abs{\E{h_{x,\epsilon}(S_n)} - \E{h_{x,\epsilon}(N)}} \Big\} + \frac{4\epsilon}{\sqrt{2\pi}} \bnl
  &\le \frac{6 \E{\abs{Y_1}^3} }{\sqrt n} + \frac{3 C_{n-1} \E{\abs{Y_1}^3}^2}{n\epsilon}+ \frac{4\epsilon}{\sqrt{2\pi}}
\end{align}

\noindent This inequality holds for all $\epsilon > 0$. We choose $\epsilon = \sqrt{\frac{C_{n-1}}{n}} \E{\abs{Y_1}^3}$. So we get

\begin{align}
  \sup_{x\in\R} \abs{\P{S_n \le x} - \P{N \le x}} & \le \frac{6 \E{\abs{Y_1}^3} }{\sqrt n} + \frac{3 \sqrt{C_{n-1}} \E{\abs{Y_1}^3}^2}{ \sqrt n \epsilon } + \frac{4\sqrt{C_{n-1}} \E{\abs{Y_1}^3} }{\sqrt{2\pi}\sqrt{n}} \nl
  &= \frac{ \E{\abs{Y_1}^3} }{\sqrt n} \left( 6+\left(3+\frac{4}{\sqrt{2\pi}}\right)\sqrt{C_{n-1}} \right)
\end{align}

\noindent We have defined $C_n$ such that it is the smallest number fulfilling 

\begin{align}
  \sup_{x\in\R} \abs{\P{S_n \le x} - \P{N \le x}} \le \frac{ \E{\abs{Y_1}^3} }{\sqrt n} C_n
\end{align}

\noindent Therefore we can conclude for $n \ge 2$

\begin{align}
  C_n \le 6 + \left(3 + \frac{4}{\sqrt{2\pi}}\right) \sqrt{C_{n-1}}
\end{align}

\subsection{Final induction}

No we proof by induction that $C_n < 33$. First note that

\begin{align}
  C_n & = \frac{\sqrt{n}}{\E{\abs{X_1}^3}} \sup_{x\in\R} \abs{\P{S_n \le x}-\Phi(x)} \nl
  &
  \begin{comment}
    \E{\abs{Y_1}^3} \ge 1
  \end{comment} \nl
  & \le \sqrt{n} \sup_{x\in\R} \abs{\P{S_n \le x}-\Phi(x)} \nl
  &
  \begin{comment}
    \begin{multlined}
      \P{S_n \le x} \in [0,1] \land \Phi(x) \in [0,1] \nl
      \implies \abs{\P{S_n \le x}-\Phi(x)} \le 1
    \end{multlined}
  \end{comment} \nl
  & \le \sqrt{n}
\end{align}

So we have $C_1 \le \sqrt{1} < 33$ and

\begin{align}
  C_{n+1} & \le 6+\left(3 + \frac{4}{\sqrt{2\pi}}\right) \sqrt{C_n} \nl
  & \le 6+\left(3 + \frac{4}{\sqrt{2\pi}}\right) \sqrt{33} \nl
  & = 32.40\ldots < 33
\end{align}

This proves $\sup_{n\in\N} C_n < \infty$ from which the Berry-Essen theorem follows.
