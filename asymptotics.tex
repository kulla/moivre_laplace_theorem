\chapter{Asymptotic notations} \label{chapter:tools}

In this chapter I will introduce and explain the asymptotic notations like $\bigo{\cdot}$ and $\littleo{\cdot}$ we will use in this thesis.

\section{The big O notation}

To describe the asymptotic behavior of an error, the big O notation can be utilized. It declares the convergence speed with which the error converges at least to zero and means~\cite[p. 444]{graham}\cite[p.~100]{aigner}

\begin{definition}[Big O notation]
  Let $\seq{a_n}$ be a sequence with positive members. $\bigo{a_n}$ is the set of all sequences defined via
  \begin{align} \label{bigo_def1}
    \seq{\epsilon_n} \in \bigo{a_n} :\iff \exists C > 0\, \forall n\in \N: \abs{\epsilon_n} \le C \cdot \abs{a_n}
  \end{align}
\end{definition}

\noindent This definition is equivalent to~\cite[p.~383]{hachenberger}\cite{wiki:bigo}
\begin{align} \label{bigo_def_limsup}
  \seq{\epsilon_n} \in \bigo{a_n} :\iff \exists C_\infty > 0 \colon\limsup_{n\to\infty} \abs{\frac{\epsilon_n}{a_n}} \le C_\infty
\end{align}

So $\epsilon_n \in \bigo{\frac 1n}$ means that $\epsilon_n$ approaches zero at least with the convergence speed of $\seq{\frac 1n}$. Thereby the advantage of the big O notation is its simple arithmetic rules. For example we have
\begin{align}
  \bigo{\frac 1n} + \bigo{\frac 1{n^2}} \subseteq \bigo{\frac 1n}
\end{align}
These arithmetic rules (which we will discuss in the end of this section) simplify error estimations.

\section{The big Delta and the big Psi notation}
Unfortunately the big O notation does only provide an estimate of the convergence speed but no estimation of the error's value since the constants $C$ and $C_\infty$ in \eqref{bigo_def1} and \eqref{bigo_def_limsup} are not known\footnote{cf.~\cite[p.~444]{graham},~\cite{hurkyl_bigo} and \cite{templatetypedef_bigo}}. Therefore I want to introduce the big Delta and the big Psi notation with which besides to the convergence speed also an estimate for the absolute error and the asymptotic error can be stated. Both notations are defined as\footnote{Donald Knuth uses the term ``A notation'' for the big Delta notation \cite{anotation}. Please compare also \cite{tampis_bigpsi} and~\cite{tampis_bigabs} respectively for the usage of the big Psi notation and big Delta notation in mathematics.}:

\begin{definition}[Big Delta notation for the absolute error]
  Let $\seq{a_n}$ be a sequence. $\bigabs{a_n}$ is the set of all sequences $\seq{\epsilon_n}$ with $\abs{\epsilon_n} \le \abs{a_n}$ for all $n\in\N$:
  \begin{align}
    \seq{\epsilon_n} \in \bigabs{a_n} \iff \forall n\in\N: \abs{\epsilon_n} \le \abs{a_n}
  \end{align}
\end{definition}

\begin{definition}[Big Psi notation for the asymptotic error]
  Let $\seq{a_n}$ be a sequence with $a_n > 0$ for all $n\in\N$. $\bigpsi{a_n}$ is the set of all sequences $\seq{\epsilon_n}$ fulfilling $\limsup_{n\to\infty} \abs{\frac{\epsilon_n}{a_n}} \le 1$. Thus
  \begin{align}
    \seq{\epsilon_n} \in \bigpsi{a_n} :\iff \limsup_{n\to\infty} \abs{\frac{\epsilon_n}{a_n}} \le 1
  \end{align}
\end{definition}

Note that the definition of the big Psi notation is the same as the limes superior definition \eqref{bigo_def_limsup} of the big O notation whereby $C_\infty$ has the concrete value $1$ and the big Delta notation is the same as \eqref{bigo_def1} with $C=1$. Thus we directly see that both notations are specializations of the big O notation:
\begin{align}
  \seq{\epsilon_n} \in \bigabs{a_n} & \implies \seq{\epsilon_n} \in \bigo{a_n} \nl
  \seq{\epsilon_n} \in \bigpsi{a_n} & \implies \seq{\epsilon_n} \in \bigo{a_n}
\end{align}

\section{The little o notation}

We will also use the little o notation which is defined as~\cite[pp.~99,~103]{aigner}\cite[p.~385]{hachenberger}\cite{wiki:bigo}

\begin{definition}[Little o notation]
  Let $\seq{a_n}$ be a sequence with positive members\footnote{To allow $a_n=0$ it is possible to define the little o notation via~\cite[pp.~448]{graham}\cite{wiki:bigo}
  \begin{align}
    \seq{b_n}\in\littleo{a_n} :\iff \forall \epsilon > 0\, \exists N\in\N\, \forall n \ge N: \abs{b_n} \le \epsilon \abs{a_n}
  \end{align}
   In this work we will only deal with sequences $\seq{a_n}$ with $a_n > 0$ for all $n\in\N$.
}. $\littleo{a_n}$ is the set of all sequences $\seq{\epsilon_n}$ with $\lim_{n\to\infty} \abs{\frac{\epsilon_n}{a_n}} = 0$. Thus
  \begin{align}
    \seq{\epsilon_n}\in\littleo{a_n} \iff \lim_{n\to\infty} \abs{\frac{\epsilon_n}{a_n}} = 0
  \end{align}
\end{definition}

\section{Calculations with sets} \label{sec:calculations_sets}

With the above notations we can simplify error estimations. Instead of stating a concrete error $\epsilon_n$ of an approximation at the $n$-th step, we use one of the above notations in the form $S(a_n)$ with $S\in\{O, \Delta, \Psi, o\}$. By using $S(a_n)$ in an expression we mean that there is a $\seq{\epsilon_n}\in S(a_n)$ fulfilling the statement. Because we will have to deal with expressions containing sets of sequences, we need rules for interpreting such expressions. These rules are \cite{bigo_rules1}\cite[pp. 17-18]{kulla}:

\begin{enumerate}
  \item Let $f:D\subseteq\R^n\to\R$ be an $n$-ary operation of real numbers and let $A_1,A_2,\ldots,A_n$ be $n$ subsets of $\R$ such that $A_1\times A_2\times \dots\times A_n\subseteq D$. The result  $f\left(A_1,\ldots,A_n\right)$ of the operation with all $n$ sets is defined as the image of $f$ \footnote{The extended version of $f$ on sets calls Ramon E. Moore in his book about interval analysis the ``unit extension''~\cite[p. 18]{moore}}
\begin{align}
  f\left(A_1,\ldots A_n\right)=\left\{f(x_1,\ldots,x_n):x_1\in A_1,x_2\in A_2,\ldots,x_n\in A_n\right\}
\end{align}
  For example
  \begin{align}
    \bigo{a_n} + \bigabs{b_n} = \{ \seq{\epsilon_n+\delta_n} : \seq{\epsilon_n}\in \bigo{a_n} \land \seq{\delta_n}\in\bigabs{b_n}\}
  \end{align}

  \item Now consider a function $\tilde f:D\subseteq\R^n\to \mathcal P(\R)$ which maps $n$ numbers to a set of real numbers. The result of $\tilde f\left(A_1,A_2,\ldots,A_n\right)$ for $A_1\times\dots\times A_n\subseteq D$ shall be
\begin{align}
  \tilde f\left(A_1,A_2,\ldots,A_n\right) = \bigcup_{x_1\in A_1, \ldots, x_n \in A_n} f(x_1,\ldots,x_n)
\end{align}
Thus $\tilde f(A_1,\ldots,A_n)$ is the union of all possible images of $\tilde f$. For example
\begin{align}
  \littleo{\bigo{a_n}} = \bigcup_{\seq{\epsilon_n}\in\bigo{a_n}} \littleo{\epsilon_n}
\end{align}

  \item If in the above rules one of the $A_k$ is a sequence $\epsilon_n$ (and not a set), we have to interpret $A_k$ as the singleton $\{\seq{\epsilon_n}\}$. Exceptions are expressions such as $\bigo{\epsilon_n}$ which are defined as in the above sections.
\end{enumerate}

It is also possible to use the rules stated by Henning Makholm\cite{bigo_rules2} (which he only formulated for the big O notation):

\begin{quote}
  ``Whenever you see an equation $t=u$ where $t$ or $u$ contain one or more $\bigo{\ldots}$, you're supposed to mentally do the following expansion:
  \begin{itemize}
    \item Replace \emph{each textual occurrence} of $O(\cdots)$ with $\phi(n)$ where $\phi$ is a \emph{fresh} function letter that ranges over $\mathcal O(\cdots)$. Different [occurrences] of $O(\cdots)$ in the formula should get separate function letters, even if the "$\cdots$" are the same.
    \item Every fresh $\phi$ that arises from an $O(\cdots)$ on the \emph{left} side of the equals sign should be \emph{universally} quantified.
    \item Every fresh $\phi$ that arises from an $O(\cdots)$ on the \emph{right} side of the equals sign should be \emph{existentially} quantified.
    \item The universal quantifiers on $\phi$s come before the existential ones.
    \item The variable $n$ that tends to a limit should be universally quantified \emph{after} all of the quantifiers on $\phi$s.''
  \end{itemize}
\end{quote}

\begin{remark}
  Be aware that I will use variables like $h=\frac{1}{\sqrt{npq}}$ where I drop the dependence on $n$ in the name of the variable. When I state something like $\littleo{h}$ I will always mean $\littleo{h_n}$, e.g.
  \begin{align}
    \seq{\epsilon_n} \in \littleo{h} \iff \lim_{n\to\infty} \abs{\frac{\epsilon_n}{h}} = \lim_{n\to\infty} \abs{\epsilon_n \sqrt{npq}} = 0
  \end{align}
\end{remark}

\section{Arithmetic rules}

All the presented notations share a common property which is in the case of the big O notation \cite{tampis:general}:
\begin{align}
  \seq{\epsilon_n} \in \bigo{a_n} \iff \seq{\abs{\frac{\epsilon_n}{a_n}}} \in \bigo{1}
\end{align}
This property of $\bigo{\cdot}$ can be condensed to the equation $\bigo{a_n} = a_n \cdot \bigo{1}$ because
\begin{align}
  \seq{\epsilon_n} \in \bigo{a_n} & \iff \seq{\abs{\frac{\epsilon_n}{a_n}}} \in a_n \cdot \bigo{1} \bnl
  & \iff \exists \seq{\delta_n} \in \bigo{1} : \abs{\frac{\epsilon_n}{a_n}} = \delta_n \bnl
  & \iff \exists \seq{\delta_n} \in \bigo{1}: \abs{\epsilon_n} = \abs{a_n} \delta_n \bnl
  &
  \begin{comment}
    \seq{a_n} \in \bigo{b_n} \implies \forall \seq{c_n} \in \{0,1\}^{\N} : \seq{(-1)^{c_n} a_n} \in \bigo{b_n}
  \end{comment} \bnl
  & \iff \exists \seq{\delta_n} \in \bigo{1} : \epsilon_n = a_n \cdot \delta_n \bnl
  & \iff \seq{\epsilon_n} \in a_n \cdot \bigo{1} \bnl
\end{align}

So $\bigo{a_n}$ is fully determined by knowing $\bigo{1}$ (cf. \cite{tampis:char}). For $\bigabs{\cdot}$ and $\littleo{\cdot}$ we get the analogous equation. Because $A(1)$ for $A\in\{O, o, \Psi, \Delta\}$ determines $A(a_n)$ we can deduce all arithmetic rules for asymptotic notation from the arithmetic rules for sets of the form $A(1)$. For example there are the rules \cite{tampis:rules}

\begin{enumerate}
  \item $(1)_{n\in\mathbb N} \in A(1) \implies (a_n)_{n\in\mathbb N} \in A(a_n)$
  \item $A(1) \subseteq B(1) \implies A(a_n) \subseteq B(a_n)$
  \item $A(1)\cdot B(1) \subseteq C(1) \implies A(a_n)\cdot B(b_n) \subseteq C(a_n\cdot b_n)$ \label{rule:example}
  \item $A(1)\cdot B(1) \subseteq C(1) \implies A(B(a_n)) \subseteq C(a_n)$ \label{rule:toproof}
  \item $A(1)+ B(1) \subseteq C(1) \implies A(a_n) + B(a_n) \subseteq C(a_n)$ \label{rule:demo}
\end{enumerate}

These rules can be mostly proven by using the equation $A(a_n) = a_n \cdot A(1)$. For example when we assume the premise $A(1) + B(1)\subseteq C(1)$ we can demonstrate rule \ref{rule:demo} via

\begin{align}
  A(a_n) + B(a_n) = a_n (A(1)+B(1)) \subseteq a_n C(1) = C(a_n)
\end{align}
Only the rule \ref{rule:toproof} is not obvious. However we find by assuming $A(1) \cdot B(1) \subseteq C(1)$ that
\begin{align}
  \seq{\epsilon_n} \in A(\delta_n) \land \seq{\delta_n} \in B(a_n) & \implies \seq{\abs{\frac{\epsilon_n}{\delta_n}}} \in A(1) \land \seq{\abs{\frac{\delta_n}{a_n}}} \in B(1) \bnl
  &
  \begin{comment}
    A(1) \cdot B(1) \subseteq C(1)
  \end{comment} \bnl
  & \implies \seq{\abs{\frac{\epsilon_n}{\delta_n}} \cdot \abs{\frac{\delta_n}{a_n}}} = \seq{\abs{\frac{\epsilon_n}{a_n}}} \in C(1) \bnl
  & \implies \seq{\epsilon_n} \in C(a_n)
\end{align}
and thus $A(B(a_n))\subseteq C(a_n)$. The arithmetic rules for sets $A(1)$ are mostly easy to prove or are already known theorems. For example $\littleo{1}\cdot\bigabs{1} \subseteq \littleo{1}$ is the well known proposition
\begin{align}
  \lim_{n\to\infty} \epsilon_n = 0 \land \sup_{n\in\N} \abs{\delta_n} \le 1 \implies \lim_{n\to\infty} \epsilon_n \delta_n = 0
\end{align}

From rule \ref{rule:example} and \ref{rule:toproof} we can now conclude that $\littleo{a_n} \cdot \bigabs{b_n} \subseteq \littleo{a_n \cdot b_n}$ and $\littleo{\bigabs{a_n}} \subseteq \littleo{a_n}$. We see that most of the arithmetic rules for asymptotic notations are corollaries from already known theorems about sequences and their limit behavior.

\section{Taylor series}

\begin{theorem}
  Let $f:D\to\R$ be a two times continuously differentiable function with an open domain $D$. Let $x_0\in D$. Then

  \begin{align}
    f(x_0+h) = f(x_0) + f'(x_0) h + \bigo{h^2}
  \end{align}
\end{theorem}

\begin{proof}
  By using the Lagrange form of the remainder we get the following Taylor expansion \cite[p. 284]{koenigsberger}
  \begin{align}
    f(x_0+h)=f(x_0)+f'(x_0)h + \frac 12 f''(\tilde x) h^2
  \end{align}
  with $\tilde x$ being in the interval bounded by $x_0$ and $x_0+h$. We fix a connected and compact neighborhood $U$ of $x_0$. $f''$ being continuous implies that $f''$ is bounded on $U$. Thus we find a $C > 0$ such that for all $h$ with $x_0+h\in U$ we find $\abs{f''(\tilde x)} \le C$. Hence for all $h$ with $x_0+h\in U$:
  \begin{align}
    f(x_0+h)& =f(x_0)+f'(x_0)h + \frac 12 f''(\tilde x) h^2 \nl
    & \in f(x_0)+f'(x_0)h + \bigabs{\frac 12 C h^2} \nl
    & \subseteq f(x_0)+f'(x_0)h + \bigo{h^2}
  \end{align}
\end{proof}

\section{Asymptotic behavior of Riemann sums}

Because we will have to deal with Riemann sums we need the following theorem:

\begin{theorem} \label{thm:asymptotics_riemann}
  Let $h > 0$ and $\x[k]$ be defined via $\x[k] = \x[0] + kh$ for $k\in\Z$ and any $\x[0]\in\R$. Let $f:\R\to\R$ with $\snorm{f} < \infty$ and $\snorm{f'} < \infty$. Let $a,b\in\R$. We find

  \begin{enumerate}
    \item $\int_x^{x+h} f(t) \d{t} \in f(x)h + \bigabs{\frac 12 \snorm{f'} h^2}$ and $\int_{x-h}^x f(t) \d{t} \in f(x) h + \bigabs{\frac 12 \snorm{f'} h^2}$
    \item $\f{\x} h \in \int_{\x-\frac h2}^{\x + \frac h2} \f{t}\d{t} + \bigabs{\frac 14 \snorm{f'} h^2}$
    \item For $a \le b$ we have

      \begin{align}
        \sum_{a \le \x \le b} \f{\x} h \in \int_a^b \f{t}\d{t} + \bigabs{\snorm{f} h + \frac 14 \snorm{f'}(b-a)h + \frac 14 \snorm{f'}h^2}
      \end{align}
  \end{enumerate}
\end{theorem}

\begin{proof}~
  \begin{enumerate}
    \item We find
      \begin{align}
        \abs{\int_x^{x+h} \f{t} \d{t} - f(x)h} &= \abs{\int_x^{x+h} \br{\f{t}-\f{x}} \d{t} } \bnl
        &
        \begin{comment}
          \forall t\,\exists \epsilon_t: \f{t}-\f{x}=\fp{\epsilon_t}(t-x)
        \end{comment} \bnl
        & \le \int_{x}^{x+h} \abs{\fp{\epsilon_t}} \abs{t-x} \d{t} \bnl
        & \le \snorm{f'} \int_{0}^{h} \abs{t} \d{t} \bnl
        & = \frac 12 \snorm{f'} h^2
      \end{align}
      Analogously we can demonstrate that $\int_{x-h}^x f(t) \d{t} \in f(x) h + \bigabs{\frac 12 \snorm{f'} h^2}$.

    \item We have
      \begin{align}
        \f{\x} h & = \f{\x} \frac h2 + \f{\x} \frac h2 \bnl
        &\in \int_{\x}^{\x+\frac h2} \f{t}\d{t} + \bigabs{\frac 18 \snorm{f'} h^2} + \int_{\x-\frac h2}^{\x} \f{t}\d{t} + \bigabs{\frac 18 \snorm{f'} h^2} \bnl
        &\subseteq \int_{\x-\frac h2}^{\x+\frac h2} \f{t}\d{t} + \bigabs{\frac 14 \snorm{f'} h^2}
      \end{align}

    \item Let $\tilde a = \inf \{ \x : \x \ge a \}$ and $\tilde b = \sup \{ \x : \x \le b \}$. The difference of $a$ and $\tilde a$ is less than or equal $h$ and the distance of $a$ and $\tilde a - \frac h2$ can be estimated by $\frac h2$. Similarly we find $\abs{\tilde b-b} \le h$ and $\abs{\br{\tilde b+\frac h2}-b} \le \frac h2$. We have

      \begin{align}
        \sum_{a \le \x \le b} \f{\x} h & \in \sum_{a\le \x\le b} \br{\int_{\x-\frac h2}^{\x+\frac h2} \f{t} \d{t} + \bigabs{\frac 14 \snorm{f'} h^2}} \bnl
        & \subseteq \int_{\tilde a - \frac h2}^{\tilde b + \frac h2} \f{t} \d{t} + \bigabs{\frac 14 \snorm{f'}\br{\tilde b + h - \tilde a} h}\bnl
        & \subseteq \int_{a+\bigabs{\frac h2}}^{b+\bigabs{\frac h2}} \f{t} \d{t} + \bigabs{\frac 14\snorm{f'}\br{b-a+\bigabs{h}} h}\bnl
        & \subseteq \int_{a+\bigabs{\frac h2}}^{b+\bigabs{\frac h2}} \f{t} \d{t} + \bigabs{ \frac 14 \snorm{f'} (b-a) h + \frac 14 \snorm{f'} h^2} \bnl
        &
        \begin{comment}
          \begin{aligned}
            \int_a^{b+\bigabs{\epsilon}} \f{t} \d{t} & \subseteq \int_a^b \f{t} \d{t} + \int_b^{b+\bigabs{\epsilon}} \f{t} \d{t} \bnl
            & \subseteq \int_a^b \f{t} \d{t} + \snorm{f} \int_b^{b+\bigabs{\epsilon}} \d{t} \bnl
            & \subseteq \int_a^b \f{t} \d{t} + \snorm{f} \bigabs{\epsilon}
          \end{aligned}
        \end{comment} \bnl
        & \subseteq \int_a^b \f{t} \d{t} + \bigabs{\snorm{f}h + \frac 14 \snorm{f'} (b-a) h + \frac 14 \snorm{f'} h^2}
      \end{align}
  \end{enumerate}
\end{proof}

