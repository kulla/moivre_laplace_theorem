%%%%%%%%%%%%%%%%%%%%%%%%%%%%%%%%%%%%%%%%%
%%            LMU-Vorlage              %%
%%                                     %%
%%         zur Erstellung einer        %%
%%   Dissertation mit pdflatex/latex   %%
%%                                     %%
%%  (2002) Robert Dahlke               %%
%%         & Sigmund Stintzing         %%
%%%%%%%%%%%%%%%%%%%%%%%%%%%%%%%%%%%%%%%%%

\documentclass[a4paper]{book}


%%%%%%%%%%%%%%%%%%%%%%%%%%%%
%%   Zusaetzliche Pakete  %%
%%%%%%%%%%%%%%%%%%%%%%%%%%%%

\usepackage[a4paper]{geometry}
\usepackage[utf8]{inputenc}
\usepackage[T1]{fontenc}
\usepackage{fancyhdr}
\usepackage{graphicx}
\usepackage{caption}
\usepackage[toc,page]{appendix}
\usepackage[bookmarks=true,draft=false]{hyperref}
\usepackage[dvipsnames]{xcolor}
\usepackage{wrapfig}
\usepackage{amsmath}
\usepackage{amssymb}
\usepackage{amsthm}
\usepackage{dsfont}
\usepackage{mathtools}
\usepackage{relsize}
\usepackage{algorithm}
\usepackage{algpseudocode}
\usepackage{colonequals}
\usepackage{units}
\usepackage{mymacros}

\usepackage{listings}

\usepackage[nottoc]{tocbibind}

\usepackage[backend=bibtex]{biblatex}
\addbibresource{references.bib}

\usepackage{mymacros}

\usepackage[toc,xindy,nonumberlist]{glossaries}
\makeglossaries
\loadglsentries{glossary}


%%%%%%%%%%%%%%%%%%%%%%%%%%%
%% Definition of Heading %%
%%%%%%%%%%%%%%%%%%%%%%%%%%%

\pagestyle{fancyplain}
\renewcommand{\chaptermark}[1]%
         {\markboth{\thechapter.\ #1}{}}
\renewcommand{\sectionmark}[1]%
         {\markright{\thesection\ #1}}
\lhead[\fancyplain{}{\bfseries\thepage}]%
    {\fancyplain{}{\bfseries\rightmark}}
\rhead[\fancyplain{}{\bfseries\leftmark}]%
    {\fancyplain{}{\bfseries\thepage}}
\cfoot{}


%%%%%%%%%%%%%%%%%%%%%%%%%%%%%%%%%%%%%%%%%%%%%%%%%%%%%
%%  Definition des Deckblattes und der Titelseite  %%
%%%%%%%%%%%%%%%%%%%%%%%%%%%%%%%%%%%%%%%%%%%%%%%%%%%%%

\newcommand{\LMUTitle}[9]{
  \thispagestyle{empty}
  \vspace*{\stretch{1}}
  {\parindent0cm
   \rule{\linewidth}{.7ex}}
  \begin{flushright}

    \vspace*{\stretch{1}}
    \sffamily\bfseries\Huge
    #1\\
    \vspace*{\stretch{1}}
    \sffamily\bfseries\large
    #2
    \vspace*{\stretch{1}}
  \end{flushright}
  \rule{\linewidth}{.7ex}
  \vspace*{\stretch{5}}
  \begin{center}
    \includegraphics[width=2in]{siegel}
  \end{center}
  \vspace*{\stretch{1}}
  \begin{center}\sffamily\LARGE{#5}\end{center}
  \newpage
  \thispagestyle{empty}

  \cleardoublepage
  \thispagestyle{empty}

  \vspace*{\stretch{1}}
  {\parindent0cm
  \rule{\linewidth}{.7ex}}
  \begin{flushright}
    \vspace*{\stretch{1}}
    \sffamily\bfseries\Huge
    #1\\
    \vspace*{\stretch{1}}
    \sffamily\bfseries\large
    #2
    \vspace*{\stretch{1}}
  \end{flushright}
  \rule{\linewidth}{.7ex}

  \vspace*{\stretch{3}}
  \begin{center}
    \Large Masterarbeit\\
    \Large an der #4\\
    \Large der Ludwig--Maximilians--Universität\\
    \Large M\"unchen\\
    \vspace*{\stretch{1}}
    \Large vorgelegt von\\
    \Large #2\\
    \Large aus #3\\
    \vspace*{\stretch{2}}
    \Large M\"unchen, den #6
  \end{center}

  \newpage
  \thispagestyle{empty}

  \vspace*{\stretch{1}}

  \begin{flushleft}
    \large Erstgutachter:  #7 \\[1mm]
    \large Diese Masterarbeit steht unter einer \ccby-Lizenz \\
  \end{flushleft}

  \cleardoublepage
}

%%%%%%%%%%%%%%%%%%%%%%%%%%%%
%%  Beginn des Dokuments  %%
%%%%%%%%%%%%%%%%%%%%%%%%%%%%

\begin{document}


  \frontmatter


  \LMUTitle
      {The de Moivre-Laplace theorem}                  % Titel der Arbeit
      {Stephan Kulla}                                  % Vor- und Nachname des Autors
      {Potsdam}                                        % Geburtsort des Autors
      {Mathematikfakultät}                             % Name der Fakultaet
      {München 2015}                                   % Ort und Jahr der Erstellung
      {30.11.2015}                                     % Tag der Abgabe
      {Prof. Dr. Peter Pickl}                          % Name des Erstgutachters

  %\markboth{Summary}{Summary}
  %\include{summary}
  %\cleardoublepage

  \tableofcontents
  \markboth{Contents}{Contents} 
  \cleardoublepage

  \glsaddall
  \printglossary[title={List of symbols},style=altlist]
  \cleardoublepage

 
  %\markboth{Remarks for this thesis}{Remarks for this thesis}
  %\include{remarks}

  \mainmatter\setcounter{page}{1}
  \chapter{Introduction}

In this thesis I present a new proof for the theorem by de Moivre and Laplace. This theorem states that the standardized binomial distribution given by the probability mass function
\begin{align}
  \bb{\x} = \binom nk p^k q^{n-k}
\end{align}
with $\x[k] = \frac{k-np}{\sqrt{npq}}$ for $p\in (0,1)$, $q=1-p$, $k,n\in\N$, and $k\le n$ can be approximated by the standardized normal distribution whose density function is $\fphi{x} = \frac{1}{\sqrt {2\pi}} \fexp{-\frac{x^2}2}$.

The new proof approximates the slope between neighboring $\bb{\x}$ and derives from it an ordinary differential equation (ODE) whose solution is the normal distribution. Hence it will give an interesting application of the theory of ODEs (especially one-step methods) to probability theory.

There are two possible advantages of the new proof which we will investigate in this thesis:

\begin{enumerate}
  \item We will compare the error estimate of the standard proof using Stirling's formula with the error bounds provided by the new proof. Thus we can see whether the new proof yield better approximations or better error estimations.

  \item The new proof may be easier to understand than the standard proof and thus may be used in a course or a textbook.
\end{enumerate}

  \include{asymptotics}
  \chapter{The binomial distribution}

\section{The motivational problem for the binomial distribution}

Probability theory emerged from and was highly influenced by investigations of games of chance~\cite[p. 4]{hald1}. Since the antiquity dice games were played and starting from the 14th century card games became more and more popular~\cite[pp. 33-34]{hald1}. Governments used lotteries to finance their expenditures and a lot of private lotteries were conducted as well~\cite[p. 34]{hald1}.

\includewrapfig{Christiaan_Huygens}{Christiaan Huygens}{File \href{https://commons.wikimedia.org/wiki/File:Christiaan_Huygens.jpg}{``Christiaan Huygens.jpg''} from Wikimedia Commons uploaded by \href{https://commons.wikimedia.org/w/index.php?title=User:Lord_Horatio_Nelson~commonswiki}{Lord Horatio Nelson$\sim$commonswiki} and licensed under Public domain}

From the economic and recreational importance arose a demand in calculating the odds of a game or the value of the expected winnings~\cite[p. 34]{hald1}. In 1654 Fermat and Pascal solved in a correspondence the problem of division~\cite[pp. 42-64]{hald1} which marks the foundation of probability theory~\cite[p. 4]{hald1}\footnote{In the 16th century Cardano already discussed several problems about games of chance in his work \emph{Liber de ludo aleae} (Book on Games of Chance)~\todo{cite, name des Buchs}\cite[pp. 33-41]{hald1}. However his book was first published posthumously 1663 and thus did not influenced Fermat, Pascal or Huygens~\cite[p. vii]{bernoulli}.}. Christiaan Huygens, who (according to himself) heard from the letters by Pascal and Fermat but did not know their methods~\cite[p. 67]{hald1}, wrote a short treatise \emph{De Ratiociniis in Ludo Aleae} (On Reckoning in Games of Chance) which was published 1657~\cite[p. vii]{bernoulli}\footnote{Huygens wrote his treatise in Dutch. His mathematics teacher Frans van Schooten translated it in Latin and published it at the end of his book \emph{Exercitationes Mathematicae}~\cite[pp. 65-68]{hald1}.}. Among the different problems, Huygens solved in his work, was the following~\cite[p. 163]{bernoulli}:

\begin{quotation}
  To find with how many dice one may undertake to throw two sixes on the first try.
\end{quotation}

\includewrapfig[0.5\textwidth]{tavern1658}{Men gambling in a tavern (image section of a Flemish painting from 1658)}{Painting \href{https://commons.wikimedia.org/wiki/File:Tavern_Scene-1658-David_Teniers_II.jpg}{``Tavern Scene''} by \href{https://en.wikipedia.org/wiki/David_Teniers_the_Younger}{David Teniers the Younger} from 1658. Cropped version of picture which was uploaded at Wikimedia Commons and which was taken by National Gallery of Art, Washington, D.C., USA. Picture is licensed under Public Domain.}

The problem is to calculate the number of dices one need to throw such that the probability of throwing two sixes is at least $\tfrac 12$. Imagine for example that you are in a tavern of the 17th century drinking beer. A merchant comes to your table and offers you the following deal: ``I'll give you $8$ dices which you can throw at once. If you will get at least two sixes you will get a gold coin. In case you only throw one six or none you have to pay me a gold coin.'' Shall you participate in his game? Is the game fair or not? Who has the higher chance to win? This example demonstrates very well the motivation of Huygens and his colleagues to engage in such problems. 

To solve this problem one first need to calculate the probability two throw exactly $m$ sixes with $n$ dices. If you are already familiar with the binomial distribution you see how this distribution can be used in the solution of the problem. This was later done by Jacob Bernoulli who reprinted Huygens' work in the first part of his book \emph{Ars Conjectandi} (The Art of Conjecturing) where he also added additional comments~\cite[p. 63]{bernoulli}. Before we will have a look at Bernoulli's solution with a derivation of the binomial distribution I want to show how Huygens dealt with the problem to have a comparison of both attempts. Huygens first noted~\cite[p. 163]{bernoulli}:

\begin{quotation}
  Now this is just the same as asking in how many throws a person may undertake to throw one die in order to get two sixes.
\end{quotation}

So it does not make any difference whether $n$ dices are thrown at once or whether there is one die which is thrown $n$ times. Jacob Bernoulli gave a good explanation for this circumstance in his reprint of Huygens' work~\cite[p. 163]{bernoulli}:

\begin{quotation}
  If, for example, one throw of ten dice is allowed, then it is certainly evident that it makes no difference whether those ten dice are thrown onto the gaming board altogether at one time or successively one after another. And if it is done successively, then it is equally clear that it makes no difference whether the ten dice thrown are ten different dice or one and the same die retrieved from the board and thrown ten times.
\end{quotation}

After noting that it does not matter how the dices are thrown, Huygens showed how to calculate the expected winning recursively when one wins the stake $a$ for throwing at least two sixes in a certain number of throws~\cite[p. 163]{bernoulli}:

\begin{quotation}
  If someone undertook to do this in two throws, there would fall to him $a/36$, by what was shown before. If he is given three throws, then, if his first throw is a six, he will still have two throws, both of which must be sixes, and we said this is worth the same as $a/36$. But if his first throw is not a six, then he needs only to get one six on the remaining two throws. By proposition X this is worth of having $11a/36$. But there is certainly one case in which he throws a six th first time and five cases in which something else happens. So in the beginning there is one case for $11a/36$ and five cases for $a/36$. By proposition III\todo{typo untersuchen}, this is worth as much as $16a/127$ or $2a/27$. In repeatedly considering one more throw, we find that there is an advantage in undertaking to throw two sixes in ten throws with one die or in one throw with ten dice.
\end{quotation}

The argumentation of Huygens is the following: First he looked at the case when two dices a thrown. In proposition XI he already had calculated the expected value for this case to be $\tfrac 1{36}a$. The he investigated the case of three throws. There are five cases in which the first throw is not a six and the remaining expected value in these cases is $\tfrac 1{36}a$ because two sixes must be thrown in the remaining two throws. In one case the first throw is a six. Huygens already had shown in proposition X that the expected value of throwing at least one six in two throws is $\tfrac{11}{36}a$. So in total there a $5$ cases with a remaining expected value of $\tfrac 1{36}a$ and $1$ case with the remaining value of $\tfrac{11}{36} a$. The total expected value for three throws was now calculated as

\begin{align}
  \frac{1}{5+1} \left(5 \cdot \frac 1{36} a + 1 \cdot \frac{11}{36} a \right) = \frac{1}{6} \cdot \frac{5+11}{36} a = \frac{16}{216} a = \frac{2}{27} a
\end{align}

\noindent Thereby Huygens used his proposition III which states~\cite[p. 135]{bernoulli}:

\begin{quotation}
  If the number of cases in which $a$ falls to me is $p$ and the number of cases in which $b$ falls to me is $q$, and if all the cases can happen equally easily, then my expectation will be worth $(pa+qb)/(p+q)$.\footnote{The final formula can be rewritten as $\tfrac{p}{p+q} a+\tfrac q{p+q} b$ which is the definition of the expected value for the Bernoulli trial, a distribution with two possible outcomes. \todo{verbessern}}
\end{quotation}

Without writing it explicitly down in his book, Huygens now calculates recursively the odds of having two sixes in $n$ throws. His calculation might have looked like the following: Let $O_n$ be the expected winning of having at least one six in $n$ throws. In proposition X Huygens already had demonstrated by looking at several examples that $O_n = \tfrac 1{1+5} \left( 1 \cdot a + 5 \cdot O_{n-1} \right)=\tfrac 16 \left( a + 5 \cdot O_{n-1} \right)$. If the first throw is a six, then the player gets the whole stack $a$. In the remaining five cases the player still has the opportunity to throw a six in the remaining $n-1$ throws and thus have an expected winning of $O_{n-1}$ to win. The calculated values by Huygens are:

\begin{align}
  \begin{array}{rll}
    O_1 & = \frac 16 a \nl
    O_2 & = \frac 16 ( a + 5 \cdot \frac 16 a) & = \frac{11}{36} a \nl
    O_3 & = \frac 16 ( a + 5 \cdot \frac{11}{36} a) & = \frac{91}{216} a \nl
    O_4 & = \frac 16 ( a + 5 \cdot \frac{91}{216} a) & = \frac{671}{1296} a \nl
    & \vdots \nl
    O_8 & = \frac 16 ( a + 5 \cdot \frac{201811}{279936} a) & = \frac{1288991}{1679616} a \nl
    O_9 & = \frac 16 ( a + 5 \cdot \frac{1288991}{1679616} a) & = \frac{8124571}{10077696} a \nl
    & \vdots
  \end{array}
\end{align}

Let $T_n$ be the expected winning for throwing at least two sixes in $n$ throws. By the same argument Huygens gave for the special case $n=3$ we deduce $T_n = \tfrac{1}{6} ( 5 \cdot T_{n-1}  + O_{n-1})$. Thus Huygens calculated

\begin{align}
  \begin{array}{rlll}
    T_2 & = \frac 1{36} a \nl
    T_3 & = \frac 16 (5 \cdot \frac 1{36} a + \frac{11}{36} a) &= \frac 2{27} a \nl
    T_4 & = \frac 16 (5 \cdot \frac 2{27} a + \frac{91}{216} a) &= \frac {19}{144} a \nl
    & \vdots \nl
    T_9 & = \frac 16 (5 \cdot \frac {663991}{1679616} a + \frac{1288991}{1679616} a) &= \frac {2304473}{5038848} a & \approx 0.457 a \nl
    T_{10} & = \frac 16 (5 \cdot \frac{2304473}{5038848} a + \frac{8124571}{10077696} a) &= \frac{10389767}{20155392} a & \approx 0.515 a \nl
    & \vdots 
  \end{array}
\end{align}

So starting with $10$ dices the odds for the player to win is higher than $\tfrac 12$. Besides the probability for us to win in the above supposed game with eight dices is $\tfrac{663991}{1679616}$ which is roughly $0.40$.

\section{Bernoulli's derivation of the binomial distribution}

\includewrapfig{Jakob_Bernoulli}{Jacob Bernoulli}{File \href{https://commons.wikimedia.org/wiki/File:Jakob_Bernoulli.jpg}{``Jakob Bernoulli.jpg''} from Wikimedia Commons uploaded by user \href{https://commons.wikimedia.org/wiki/User:Materialscientist}{Materialscientist} and licensed under Public Domain}

As we have seen in the previous section Christiaan Huygens could well solve his proposed problem. One also can extract an algorithm from his method with which similar problems can be solved. But he only investigated a special case although a more general formulation of his method is possible. For example one may be interested in a game with dices which have more or less than six faces. Also the number of favorable faces on a die may be more then one and the minimal value of favorable results in all throws may differ from two. To the previous and similar problem Jacob Bernoulli wrote~\cite[p. 157]{bernoulli}:

\begin{quotation}
  If the Author [Christiaan Huygens, S.K.]\todo{richtig?} had substituted letters for numbers, he could have expressed this and the preceding proposition as one problem and investigated its general solution with equal ease [...]\todo{richtig?}
\end{quotation}

Also Huygens' recursive method needs a lot of calculational effort to find the final solution. Thus one question arises: Is there a more direct and general solution for Huygens' problem?

\includefig{Ars_Conjectandi}{Cover of \emph{Ars Conjectandi}}{Cover of \emph{Ars Conjectandi} by Jacob Bernoulli 1713. File \href{https://commons.wikimedia.org/wiki/File:Arsconj.gif}{``Arsconj.gif''} from Wikimedia Commons uploaded by user \href{https://commons.wikimedia.org/wiki/User:Nousernamesleft}{Nousernamesleft} and licensed under Public Domain.}

Jacob Bernoulli addressed this question with the binomial distribution as an answer in his book \emph{Ars Conjectandi} (The Art of Conjecturing) from 1713~\cite[pp. 220-256]{hald1}\footnote{The book was published eight years after Jacob Bernoulli's death by his nephew Nicholas Bernoulli in 1703~\cite[pp. 223-224]{hald1}}. In the first part of this book Bernoulli\footnote{The family of Bernoullis is huge with a lot of influential mathematicians. In the following I always refer to Jacob Bernoulli when I only mention ``Bernoulli''. A good overview of Bernoulli's family tree gives~\cite[pp. 1-4]{bernoulli}} reprinted Huygens' \emph{De Ratiociniis in Ludo Aleae} with additional notes and alternative solutions~\cite[p. 63]{bernoulli}.

In his comments to the above problem Bernoulli first derived the binomial distribution by incomplete induction with an argumentation similar to Huygens' solution. Afterwards he derived the binomial distribution with a combinatorial argumentation~\cite[pp. 165-167]{bernoulli}:

\begin{quotation}
  [Consider] $n$ dice A, B, C, D, etc., each constructed with $a$ faces, $b$ of which match the purpose of the player, and the other $c$ of which do not. And ask in how many cases it may happen that what was undertaken is accomplished on none of the dice, on just one of the dice, or on only 2, 3, or 4 of them, and so on; finally, in how many cases the person undertaking the game fails in his purpose and his opponent wins. But it was shown in the Note to the preceding Proposition that there are $C^n$ cases in which what was undertaken is accomplished on none of the dice. And it may be concluded in a similar way that there are $b$ cases, or $bb$, or $b^3$, etc., in which one of the dice, for instance A, or two, A and B, or three, A, B, and C, and so on, meet the purpose of the person undertaking the game. Similarly, there are $c^{n-1}$ cases, or $c^{n-2}$ cases, and so on, in which $n-1$ dice, or $n-2$ dice, or $n-3$ dice, and so on, fail to fulfill his hopes. Therefore, since each of these cases can be joined with any of the preceding, multiplication of the latter by the former produces $bc^{n-1}$ cases, or $bbc{n-2}$ cases, or $b^3c^{n-3}$ cases, and so on. And since the die or dice that favor the person undertaking the game might be either A or B or C, and so on, if there is one, or, if there are two, either A and B, or A and C, or B and C, and so on, or, if there are three, either A, B, and C, or A, B and D, and so on, these number of cases must again be multiplied by the number of units, pairs, triples, and so son that can be drawn from all $n$ dice. But by the theory of combinations developed in Part II, this number is $n$, or $n(n-1)/(1\cdot 2)$, or $n(n-1)(n-2)/(1\cdot 2\cdot 3)$, etc. So from this multiplication we obtain $nbc^{n-1}$, or

  
  $\frac{n(n-1)}{1\cdot 2}bbc^{n-2}$,

  or $\frac{n(n-1)(n-2)}{1\cdot 2\cdot 3}b^3c^{n-3}$ etc., on up to

  $\frac{n(n-1)(n-2)\ldots(n-m+2)}{1\cdot 2\cdot 3\cdot 4\ldots(m-1)} b^{m-1} c^{n-m+1}$

  for the numbers of cases in which what was undertaken is accomplished with exactly one die, or two dice, or three dice and so on up to $m-1$ dice. Since, as we mentioned, the opponent wins in all these cases and since there are altogether $a^n$ cases with $n$ dice, his lot is, \textbf{[42]} by Corollary 1 of Proposition III,

  $c^n+nbc^{n-1}+\frac{n(n-1)}{1\cdot 2}bbc^{n-2} + \frac{n(n-1)(n-2)}{1\cdot 2\cdot 3}b^3c^{n-3} \ldots$

  $+\frac{n(n-1)(n-2)\ldots(n-m+2)}{1\cdot 2\cdot 3\cdot 4\ldots(m-1)} b^{m-1}c^{n-m+1} : a^n$,\footnote{The notation $x_1+x_2+\ldots+x_n:y$ means $\tfrac 1y (x_1+x_2+\ldots+x_n)$. \todo{cite}}

  as above.
\end{quotation}

Bernoulli considered a die with $a$ faces which has $b$ faces being favorable for the player and $c=a-b$ faces being unfavorable. The player has to throw at least $m$ times one of the $b$ favorable faces in a total of $n$ throws in order to win. What is the probability of winning / loosing?

Bernoulli started with calculating the number of outcomes in which the player throws exactly $k$ times one of the favorable faces. Fix $k$ of the total $n$ throws. All of these $k$ throws shall be one of the $b$ favorable results and the other $n-k$ shall be one of the $c$ unfavorable results. Because each of these $k$ throws can show one of $b$ possible faces the total number of different outcomes in these $k$ throws with only favorable results is $b^k$. Analogously is the number of possible outcomes in the other $n-k$ throws with only unfavorable results equal to $c^{n-k}$. Therefore there are in total $b^kc^{n-k}$ different outcomes in which the $k$ fixed throws show a favorable face and the other $n-k$ throws show a unfavorable face.

In favor of throwing exactly $k$ favorable numbers, the order in which those $k$ numbers are thrown does not matter. As known by combinatorics there are $\binom nk=\frac{n!}{k!(n-k)!}=\frac{n\cdot(n-1)\ldots(n-k+1)}{1\cdot2\cdot3\ldots k}$ different orders in which exactly $k$ favorable throws occur. For each of these orders the total number of different outcomes, in which the $k$ favorable and $n-k$ unfavorable results occur in the specified order, is $b^kc^{n-k}$ as we have seen in the previous paragraph. Thus the total numbers of outcomes with exactly $k$ favorable throws is $\binom nk b^k c^{n-k}$.

Because the total number of outcomes is $a^n$ and because all outcomes are equally likely, the probability for the player to have exactly $k$ favorable throws is $\frac {1}{a^n} \binom nk b^k c^{n-k}$.

Because the player looses when he throws at most $m-1$ favorable numbers, the probability for the player to loose is

\begin{align}
  \P{\text{player looses}} = \frac{1}{a^n} \sum_{k=0}^{m-1} \binom nk b^k c^{n-k}
\end{align}

This is the same formula Bernoulli derived in the end of the above quoted section. He ended his discussion of Huygens' problem with two corollaries which he named ``Rule for knowing the lot of a player to whom several throws of a die have given and who is held to archieving something precisely on some certain throws and not on others''~\cite[p. 170]{bernoulli}:

\begin{quotation}
  \emph{Corollary 1.} If there are the same numbers of cases in all games [...] the expectation found to be [...] $b^mc^{n-m}/a^n$, taking $n$ for the number of all the throws, and $m$ for the number in which what is to be achieved ought to be achieved.
  
  \emph{Corollary 2.} If the same numbers of cases rule in all games and if, in addition, the number of games or throws is determined within which something is to be achieved, but such that the desired result can occur on any throws and not on certain determined ones [...] then it is clear that the quantity of the expectation found in the previous corollary should be multiplied by as many times as [...] $m$ things may be chosen from $n$ diverse things. This, however, may occur, by the Theory of Combinatorics to be treated in Part II, in $n(n-1)(n-2)(n-3)\ldots(n-m+1)/(1\cdot 2\cdot 3\cdot 4\ldots m)$ ways [...]. Thus the expectation of the player will be worth

  \begin{align*}
    \frac{n(n-1)(n-2)(n-3)\ldots(n-m+1)}{(1\cdot 2\cdot 3\cdot 4\ldots m)}\cdot \frac{b^mc^{n-m}}{a^n}
  \end{align*}
\end{quotation}

When we introduce $p=\tfrac ba$ which is the probability that one get a favorable result in one throw and analogously $q=\tfrac ca=1-p$ for the probability of an unfavorable result in a throw, we get:

\begin{align}
  \P{\text{$m$ in $n$ throws are favorable}} & = \frac{n(n-1)(n-2)(n-3)\ldots(n-m+1)}{(1\cdot 2\cdot 3\cdot 4\ldots m)}\cdot \frac{b^mc^{n-m}}{a^n} \nl
  &= \binom nm \frac{b^m}{a^m} \frac{c^{n-m}}{a^{n-m}} \nl
  &= \binom nm p^m q^{n-m}
\end{align}

So we finally get a formula for calculating the odds of having $m$ favorable cases in a total of $n$ (independent) cases when in each case the probability of having a favorable case is $p$\footnote{In Bernoulli's work $p$ is $\tfrac ba$ and thus Bernoulli only considered rational probabilities in his derivation. This was common in the beginning of probability theory \todo{cite}. It may have its origin, that in most games with a finite number of rounds only rational probabilities occur.}. As we will see in the next section, this formula gets the probability mass of the binomial distribution.

\section{Definition of the binomial distribution}

\section{Applications of the binomial distribution}


  \chapter{The De Moivre-Laplace theorem}

\section{Formulation of De Moivre-Laplace theorem}

In the last chapter we have seen with the application of the binomial distribution in the investigation of the sex ratio that calculations with the binomial distribution are arduous. In order to compute $\binom nk p^kq^{n-k}$ there are $n+2\min\{k,n-k\}-1$ operations necessary.

This motivates approximations to the binomial distribution to minimize the computational efforts. With those approximations also equations like $\P{\Bs \le z}=\alpha$ are easier to solver for $z$ or $n$ ($\Bs$ shall be binomially distributed). Explicit solutions for those equations are hard to calculate~\cite[p. 469]{hald1}.

The first simple approximation to the binomial distribution was found by Abraham de Moivre with the help of James Stirling~\cite[p. 469]{hald1}. His proof was later extended by Pierre Simon Laplace~\cite[pp. 495 ff.]{hald1}. Today this approximation is known as the ``De Moivre-Laplace theorem''~\cite[pp. 64-67]{irle}. \todo{besseres Zitat}

\section{The historical proof of the De Moivre-Laplace theorem}

\section{Applications of De Moivre-Laplace theorem}

Graunt + sex ratio, Computerprogramme

  \chapter{Standard proof of the De Moivre-Laplace theorem}

In this chapter I want to prove the theorem by de Moivre and Laplace whereby I want to follow the basic ideas of their original proof\footnote{You can find similar proofs without error estimations in \cite[pp. 64-67]{irle}, \cite[pp. 131-134]{georgii} and \cite{wiki:demoivrelaplace}}. So we will first derive Stirling's formula of the factorial with which we will find an approximation of the binomial's probability mass function. This approximation will be used to deduce de Moivre's and Laplace's theorem.

\section{Stirling's formula}

Without proof the following theorem\footnote{cf.~\cite[pp. 505, 63]{heuser}} will be used in this section:

\begin{theorem}[Wallis' product]
  It is:

  \begin{align}
    \lim_{n\to\infty} \frac{2^2\cdot4^2\cdot6^2\dots(2n)^2}{1^2\cdot3^2\cdot5^2\dots(2n-1)^2}\cdot \frac{1}{2n} = \lim_{n\to\infty} \frac{1}{2n} \cdot \frac{2^{4n}}{\binom{2n}{n}^2} = \frac{\pi}{2}
  \end{align}
\end{theorem}

\noindent A proof of this equality can be found in~\cite[pp. 504-505]{heuser}. From this equation directly follows: 

\begin{align} \label{wallis}
  \lim_{n\to\infty} \frac{2^{2n}}{\sqrt n \binom{2n}{n}} = \lim_{n\to\infty} \sqrt{\frac 1n \cdot \frac{2^{4n}}{\binom{2n}{n}^2}} = \sqrt \pi
\end{align}

\noindent Besides we will also need the Euler–Maclaurin formula~\cite[p. 226]{koenigsberger}:

\begin{theorem}[Euler–Maclaurin formula]
  Let $f:[1,n]\to\R$ be a $2n+1$-times continuously differentiable function. Let $\bn$ be the $n$th Bernoulli number and let $\bp{x}$ be the $n$th periodic Bernoulli polynomial. It is

  \begin{multline} \label{euler-maclaurin-formula}
    \sum_{k=1}^n f(k) = \int_1^n f(x)\d{x} + \frac{f(1)+f(n)}{2} + \sum_{k=1}^n \frac{\bn[2k]}{(2k)!} \Big[f^{(2k-1)}(n) - f^{(2k-1)}(1)\Big] \nl
     + \int_1^n \frac{\bp[2n+1]{x}}{(2n+1)!} f^{(2n+1)}(x) \d{x}
  \end{multline}
\end{theorem}

The periodic Bernoulli polynomials $\bps$ are defined on $[0,1)$ by the following properties~\cite[p. 291]{koenigsberger}:

\begin{enumerate}
  \item $\bp[0]{x} = 1$
  \item $\bps[n+1]'(x) = (n+1)\cdot \bp{x}$
  \item $\int_0^1 \bp{x} \d{x} = 0$
\end{enumerate}

The definition of $\bps$ on $[0,1)$ is then periodically continued to the whole domain $\R$ \todo{how? / cite}. The Bernoulli numbers $\bn$ fulfill $\bn=\bp{0}$~\cite[p.~290]{koenigsberger}. All odd Bernoulli numbers $\bn[2n-1]$ for $n>1$ are zero and for the even Bernoulli numbers we get~\cite[p.~289]{koenigsberger}:

\begin{align}
  \bn[2] = \frac 16;\quad \bn[4]=-\frac{1}{30};\quad \bn[6]=\frac{1}{42};\quad \ldots
\end{align}

You can find a proof of the Euler-Maclaurin formula in~\cite[pp.~225-226]{koenigsberger} and~\cite[pp.~506-509]{heuser}. With the above two theorems we can derive Stirling's formula~(cf.~\cite[p.~228]{koenigsberger}):

\begin{theorem}[Stirling's formula]
  The factorial $n!$ fulfills:

  \begin{align}
    n! \in \ean[\frac 1n]{\sqrt{2\pi n} \left(\frac ne\right)^n}
  \end{align}
\end{theorem}

\begin{proof}
  In the proof we will follow~\cite[pp. 227-228]{koenigsberger}. First we apply the Euler-Maclaurin formula~\eqref{euler-maclaurin-formula} to $\ln(n!)=\sum_{k=1}^n \ln(k)$:

  \begin{align}
    \ln(n!) & = \sum_{k=1}^n \ln(k) \nl
    & = \int_1^n \ln(x) \d{x} + \frac{\ln(1)+\ln(n)}{2} + \frac{\bn[2]}{2!} \Big[\ln'(n)-\ln'(1)\Big] + \int_1^n \frac{\bp[3]{x}}{3!} \ln^{(3)}(x) \d{x} \nl
    &= \int_1^n \ln(x) \d{x} + \frac 12 \ln(n) + \frac{1}{12} \left(\frac 1n - 1\right) + \frac 13 \int_1^n \frac{\bp[3]{x}}{x^3} \d{x} \nl
    & \begin{comment}
      \int_1^n \ln(x) \d{x} = \Big[x\ln(x)-x\Big]_1^n = n\ln(n)-n+1
    \end{comment} \nonumber \nl
    &= n\ln(n)-n+\frac 12 \ln(n) + \frac{1}{12n} + \frac{11}{12} + \frac{1}{3} \int_1^n \frac{\bp[3]{x}}{x^3}\d{x} 
  \end{align}

  \noindent Thus

  \begin{align}
    \ln(n!) - n\ln(n) + n - \frac 12 \ln(n) = \frac{11}{12} + \frac{1}{12n} + \frac 13 \int_1^n \frac{\bp[3]{x}}{x^3} \d{x}
  \end{align}

  $\bps[3]$ is on $[0,1)$ as part of a polynomial bounded. Because $\bps[3]$ is $1$-periodic, $\bps[3]$ is bounded on the whole domain $\R$. Thus $\int_1^\infty \frac{\bp[3]{x}}{x^3} \d{x}$ exists because $\int_1^\infty \frac 1{x^3} d{x} < \infty$. Now we define $\seq{b_n}$ via

  \begin{align}
    b_n = \frac{n!}{n^n e^{-n} \sqrt{n}}
  \end{align}

  \noindent This sequence converges because

  \begin{align}
    \lim_{n\to\infty} \ln(b_n) & = \lim_{n\to\infty} \Big( \ln(n!) -n \ln(n) + n - \frac 12\ln(n) \Big) \nl
    & = \lim_{n\to\infty} \left( \frac{11}{12} + \frac{1}{12n} + \frac 13\int_1^n \frac{\bp[3]{x}}{x^3} \d{x}\right) \nl
    & = \frac{11}{12} + \frac 13\int_1^{\infty} \frac{\bp[3]{x}}{x^3} \d{x}
  \end{align}

  \noindent Let $b=\lim_{n\to\infty} b_n$. To calculate this limit we investigate $\tfrac{b_n^2}{b_{2n}}$:

  \begin{align}
    \lim_{n\to\infty} \frac{b_n^2}{b_{2n}} &= \lim_{n\to\infty} \frac{(n!)^2}{n^{2n}e^{-2n} n} \cdot \frac{(2n)^{2n} e^{-2n} \sqrt{2n}}{(2n)!} \nl
    &= \lim_{n\to\infty} \sqrt{2} \frac{2^{2n}}{\sqrt{n}\binom{2n}{n}} \nl
    & \begin{comment}
    \lim_{n\to\infty} \frac{2^{2n}}{\sqrt{n}\binom{2n}{n}} = \sqrt{\pi} \text{, see \eqref{wallis}}
    \end{comment} \nonumber \nl
    & = \sqrt{2\pi}
  \end{align}

  \noindent On the other hand we have

  \begin{align}
    \lim_{n\to\infty} \frac{b_n^2}{b_{2n}} = \frac{b^2}{b} = b
  \end{align}

  \noindent Thus $\lim_{n\to\infty} b_n = \sqrt{2\pi}$. From this we can follow

  \begin{align}
    \lim_{n\to\infty} \frac{n!}{\sqrt{2\pi n}n^n e^{-n}} = \lim_{n\to\infty} \frac{b_n}{\sqrt{2\pi}} = 1
  \end{align}

  This proves that $n!$ can be approximated by $\sqrt{2\pi n}n^n e^{-n}$ for large $n$. To estimate the error of this approximation we first calculate $\ln(n!)-\ln\left(\sqrt{2\pi n}n^n e^{-n}\right)$:

  \begin{align}
    \ln\left(\frac{n!}{\sqrt{2\pi n}n^n e^{-n}}\right) & = \ln\left(\frac{b_n}{\sqrt{2\pi}}\right) \nl
    &= \ln(b_n)-\ln\left(\sqrt{2\pi}\right) \nl
    &= \ln(b_n)-\lim_{n\to\infty} \ln(b_n) \nl
    &\begin{comment}
      \begin{aligned}
        \ln(b_n) &= \ln(n!)-n\ln(n)+n-\frac 12\ln(n) \nl
        &= \frac{11}{12} + \frac{1}{12n} + \frac 13 \int_1^n \frac{\bp[3]{x}}{x^3} \d{x}
      \end{aligned}
    \end{comment} \nl
    &= \frac{11}{12} + \frac{1}{12n} + \frac 13 \int_1^n \frac{\bp[3]{x}}{x^3} \d{x} -\left(\frac{11}{12}  + \frac 13 \int_1^\infty \frac{\bp[3]{x}}{x^3} \d{x}\right) \nl
    &= \frac{1}{12n}-\frac 13\int_n^\infty \frac{\bp[3]{x}}{x^3} \d{x}
  \end{align}

  On $[0,1)$ the periodic Bernoulli polynomial has the form $\bp[3]{x} = x^3-\frac 32 x^2 + \frac 12 x$~\cite[p.~290]{koenigsberger}. It has roots at $x=0$ and $x=1$. Its derivative $\bps[3]'(x) = 3x^2-3x+\tfrac 12$ has two distinct roots at $x = \frac 12 \pm \sqrt{\frac 1{12}}$. Thus $\bps[3]$ has local extrema at $x=\frac 12 \pm \sqrt{1{12}}$. Because $\bp[3]{0}=\bp[3]{1}=0$ and $\bps[3]$ is 1-periodic its supremum is

  \begin{align}
    \snorm{\bps[3]} = \max\left\{\abs{\bp[3]{\frac 12 + \sqrt{\frac{1}{12}}}}, \abs{\bp[3]{\frac 12 - \sqrt{\frac{1}{12}}}}\right\} \approx 0.0481 \le 0.05 = \frac{1}{20}
  \end{align}

  \noindent Thus

  \begin{align}
    \abs{\frac 13\int_n^\infty \frac{\bp[3]{x}}{x^3} \d{x}} & \le \frac 13 \int_n^\infty \frac{\abs{\bp[3]{x}}}{x^3} \d{x} \nl
    & \le \frac{\snorm{\bps[3]}}{3} \int_n^\infty \frac 1{x^3} \d{x} \nl
    & \le \frac{1}{60} \left[-\frac{1}{2x^2}\right]_n^\infty \nl
    & = \frac 1{120n^2}
  \end{align}

  \noindent So\footnote{At this point we also may have deduced $\ln\left(\frac{n!}{\sqrt{2\pi n}n^n e^{-n}}\right) \in \an[\frac1{120n^2}]{\frac 1{12n}}$ from which follows

  \begin{align}
    n! \in \ean[\frac 1{120n^2}]{\sqrt{2\pi n} \left(\frac ne\right)^n e^{\frac 1{12n}}}
  \end{align}

  This gives a better approximation with a better error bound. For the following sections we do not need such a good approximation.

  \todo{better design -> clearpage?! Sieht sonst blöd aus mit der Fußnote am Seitenende}
  }

  \begin{align}
    \abs{\ln\left(\frac{n!}{\sqrt{2\pi n}n^n e^{-n}}\right)} & = \abs{\frac 1{12n}-\frac 13 \int_n^\infty \frac{\bp[3]{x}}{x^3} \d{x}} \nl
    & \le \frac 1{12n} + \abs{\frac 13\int_n^\infty \frac{\bp[3]{x}}{x^3} \d{x}} \nl
    & \le \frac 1{12n} + \frac{1}{120n^2} \nl
    & \le \frac 1{2n} + \frac{1}{2n} \nl
    & = \frac{1}{n}
  \end{align}

  \noindent From the above inequality follows

  \begin{align}
    \sqrt{2\pi n}n^n e^{-n} e^{-\frac 1n} \le n! \le \sqrt{2\pi n} n^n e^n
  \end{align}

  \noindent or

  \begin{align}
    n! \in \ean[\frac 1n]{\sqrt{2\pi n} \left(\frac ne\right)^n}
  \end{align}
\end{proof}

\section{The local version of de Moivre-Laplace theorem}

In this section we will show that the probability mass function of the standardized binomial distribution can be approximated by the density function of the normal distribution. This will be the basis for the global version of de Moivre-Laplace theorem.

\begin{theorem}[Local version of de Moivre-Laplace theorem]
  Let $C \in \Rplus$. For $\abs{\x} \le \bigpsi{C}$ we have

  \begin{align}
    \bb{k} = \binom nk p^k q^{n-k} & \in \ean[\bigpsi{\frac{1}{2\sqrt{npq}} \cdot C + \frac{p^2+q^2}{6\sqrt{npq}} \cdot C^3}]{\frac{1}{\sqrt{2\pi npq} \exp\left(-\frac{(k-np)^2}{2npq}\right)}} \nl
    & = \ean[\bigpsi{\frac{1}{2} Ch + \frac{p^2+q^2}{6} \cdot C^3h}]{\frac{1}{\sqrt{2\pi}} \exp\left(-\frac{\x}{2}\right) h}
  \end{align}
\end{theorem}

\begin{proof}
  Note that $\abs{\x}\le \bigpsi{C}$ means that there is a sequence $\seq{\epsilon_n}$ with $\limsup_{n\to\infty} \frac{\epsilon_n}{C} \le 1$ and $\abs{\x} \le \epsilon_n$. First we investigate what $\abs{\x} \le \bigpsi{C}$ means for $k$ and $n-k$:

  \begin{align}
    \begin{array}{rrl}
      & \abs{\x} & \le \bigpsi{C} \bnl
      \implies & \abs{\frac{k-np}{\sqrt{npq}}} & \le \bigpsi{C} \bnl
      \implies & \abs{k-np} & \le \bigpsi{C\sqrt{npq}} \bnl
      \implies & \abs{\frac k{np}-1} & \le \bigpsi{\frac{C\sqrt{q}}{\sqrt{np}}} \bnl
      && \begin{comment}
       \abs{\frac ab-1} \le \epsilon \implies a \in \ean[\epsilon]{b}
      \end{comment} \bnl
      \implies & k & \in \ean[\bigpsi{\frac{C\sqrt{q}}{\sqrt{np}}}]{np} \bnl
      \implies & k & \in \ean[\bigpsi{qCh}]{np}
    \end{array}
  \end{align}

  \noindent For $n-k$ we get

  \begin{align}
    \begin{array}{rrl}
      & \abs{\x} & \le \bigpsi{C} \bnl
      \implies & \abs{\frac{k-np}{\sqrt{npq}}} & \le \bigpsi{C} \bnl
      \implies & \abs{np-k} & \le \bigpsi{C\sqrt{npq}} \bnl
      \implies & \abs{n-k-nq} & \le \bigpsi{C\sqrt{npq}} \bnl
      \implies & \abs{\frac {n-k}{nq}-1} & \le \bigpsi{\frac{C\sqrt{p}}{\sqrt{nq}}} \bnl
      && \begin{comment}
       \abs{\frac ab-1} \le \epsilon \implies a \in \ean[\epsilon]{b}
      \end{comment} \bnl
      \implies & k & \in \ean[\bigpsi{\frac{C\sqrt{p}}{\sqrt{nq}}}]{nq} \bnl
      \implies & k & \in \ean[\bigpsi{pCh}]{nq}
    \end{array}
  \end{align}

  \noindent We have

  \begin{align}
    \bb{\x} &= \binom nk p^k q^{n-k} \nl
    &= \frac{n!}{k!(n-k)!} p^k q^{n-k} \nl
    & \begin{comment} \text{Stirling's formula} \end{comment} \nl
    &\in \frac{\ean[\frac 1n]{\sqrt{2\pi n}n^n e^{-n}}}{\ean[\frac 1k]{\sqrt{2\pi k}k^k e^{-k}}\cdot \ean[\frac 1n]{\sqrt{2\pi (n-k)}(n-k)^{n-k} e^{-(n-k)}}} \cdot p^k q^{n-k} \bnl
    & \subseteq \ean[\frac 1n + \frac 1k + \frac 1{n-k}]{\sqrt{\frac{n}{2\pi k (n-k)}} \frac{n^n}{k^k(n-k)^{(n-k)}}\cdot p^k q^{n-k}} \bnl
    & \subseteq \ean[\frac 1n + \frac 1k + \frac 1{n-k}]{\sqrt{\frac{n}{2\pi k (n-k)}} \left(\frac{np}{k}\right)^k \left(\frac{nq}{n-k}\right)^{n-k}} \bnl
    &
    \begin{comment}
      \begin{aligned}
        \frac 1k + \frac 1{n-k} & = \frac{n}{k(n-k)} \bnl
        & \in \frac{n}{\ean[qCh]{np}\cdot \ean[pCh]{nq}} \bnl
        & = \ean[Ch]{\frac{1}{npq}} \bnl
        & \subseteq \ean[\littleo{1}]{\frac{1}{npq}} \nl
        & \subseteq \bigpsi{\frac{1}{npq}} = \bigpsi{h^2}
      \end{aligned}
    \end{comment} \bnl
    & \subseteq \ean[\frac 1n + \bigpsi{h^2}]{\sqrt{\frac{n}{2\pi k (n-k)}} \left(\frac{np}{k}\right)^k \left(\frac{nq}{n-k}\right)^{n-k}} \bnl
    &
    \begin{comment}
      \frac 1n \in \bigpsi{\frac 1n} = \bigpsi{\frac{pq}{npq}} = \bigpsi{pq h^2}
    \end{comment} \bnl
    & \subseteq \ean[\bigpsi{(1+pq)h^2}]{\sqrt{\frac{n}{2\pi k (n-k)}} \left(\frac{np}{k}\right)^k \left(\frac{nq}{n-k}\right)^{n-k}} \bnl
  \end{align}

  Now we first approximate the two main factors of the remaining product. We start with the root $\sqrt{\frac{n}{2\pi k(n-k)}}$: 

  \begin{align}
    \sqrt{\frac{n}{2\pi k(n-k)}} & \in \sqrt{\frac{n}{2\pi \ean[qCh]{np} \ean[pCh]{nq}}} \nl
    &
    \begin{comment}
      \lambda \cdot \ean[\epsilon]{a} \cdot \ean[\delta]{b} = \ean[\epsilon+\delta]{\lambda ab}
    \end{comment} \nl
    & = \sqrt{\ean[Ch]{\frac{1}{2\pi npq}}} \nl
    &
    \begin{comment}
      \ean[\epsilon]{a}^\alpha = \ean[\alpha \cdot \epsilon]{a^\alpha}
    \end{comment} \nl
    & = \ean[\frac 12 Ch]{\frac 1{\sqrt{2\pi npq}}}
  \end{align}

  \noindent In order to approximate $\left(\frac{np}{k}\right)^k \left(\frac{nq}{n-k}\right)^{n-k}$ we investigate its logarithm

  \begin{align}
    \ln\left(\left(\frac{np}{k}\right)^k \left(\frac{nq}{n-k}\right)^{n-k}\right) & = k \ln\left(\frac{np}{k}\right) + (n-k) \ln\left(\frac{nq}{n-k}\right) \nl
    &=-n \left[ \frac kn \ln\left(\frac{k}{np}\right) + \left(1-\frac kn\right) \ln\left(\frac{1-\frac kn}{q}\right)\right] \nl
    &
    \begin{comment}
      \h{s} := s \ln\left(\frac sp\right) + (1-s) \ln\left(\frac{1-s}{1-p}\right)
    \end{comment} \nl
    &=-n \h{\frac kn}
  \end{align}

  \noindent The function $\h{s}$ has the derivatives

  \begin{align}
    h'(s) &= \ln\left(\frac sp\right) - \ln\left(\frac{1-s}{1-p}\right) \nl
    h''(s) &= \frac 1s + \frac 1{1-s} \nl
    h^{(3)}(s) &= -\frac 1{s^2} + \frac 1{(1-s)^2}
  \end{align}

  \noindent We have $h(p)=h'(p)=0$ and $h''(p) = \frac 1p + \frac 1q = \frac 1{pq}$. Therefore the Taylor series of $\h{s}$ is

  \begin{align}
    h(s) = \frac{1}{2pq} (s-p)^2 + R(s)
  \end{align}

  \noindent with $R(s)=\frac{1}{3!} \left(\frac{1}{(1-\tilde s)^2}-\frac{1}{\tilde s^2}\right) (s-p)^3$ for a $\tilde s$ in the interval bounded by $s$ and $p$. 
\end{proof}

  \chapter{Alternative proof via slope approximation}

In this chapter I present an alternative proof of the de Moivre-Laplace theorem which approximates the slope of the probability mass function and the density function between the standardized binomial and the normal distribution. The idea of this proof comes from Prof. Peter Pickl and was first described in the thesis by ... \todo{cite}.

\section{Intuition behind the proof}

\section{Necessary prior knowledge from ODE}

In order to formalize the above proof sketch into a valid proof we will need some concepts and theorems from the theory about ordinary differential equations. An equation of the form

\begin{align} \label{ode:def}
  y'(x) = f(x,y(x))
\end{align}

with $f:\R^2 \to \R$ will be called an \emph{ordinary differential equation} (of first order)~\cite[p. 465]{stoer}\cite{wiki:ode}\footnote{We may also restrict the domain of $f$ to the set $I\times \Omega$ with $I$ being an interval and $\Omega\subseteq \R$\todo{cite}. For this chapter the more special definition is sufficient.} Instead of the term ``ordinary differential equation'' also its acronym ``ODE'' is often used~\cite[p. 2]{ricardo}\cite{wiki:ode}. Any function $y:\R\to\R$ fulfilling the ODE \eqref{ode:def} is called a \emph{solution of the ordinary differential equation}~\cite[p. 8]{ricardo}\cite{wiki:ode}. For example the density function $\fphi{x} = \frac{1}{\sqrt{2\pi}} \fexp{-\frac{x^2}2}$ of the normal distribution is the solution of the ODE

\begin{align}
  y'(x) = -xy(x)
\end{align}

with the initial value $y(0) = \frac{1}{\sqrt{2\pi}}$ \todo{cite}. In general the ordinary differential equation with initial value

\begin{align}
  y'(x) = -xy(x),\ y(0)=a
\end{align}

is solved by $y_a(x) = a \fexp{-\frac{x^2}2}$. ODEs can be approximated numerically with \emph{one-step methods}~\cite[pp.~471~ff]{stoer}. A one-step method is described by a function $M(x,y,h,f)$ with a step size $h>0$~\cite[p.~473]{stoer}. Given an initial value $y(x_0)=y_0$ the approximations $y_k$ at the points $x_k = x_0 + kh$ ($k\in\N$) can be obtained recursively via the relation

\begin{align}
  y_{k+1} = y_k + h M(x_k, y_k, h, f)
\end{align}

for $k \le 0$~\cite[p. 473]{stoer}. One of the most famous one-step method is the \emph{Euler method} with $M(x,y,h,f) = f(x,y)$~\cite[p. 473]{stoer}. We will use the concept of one-step methods to connect the density function of the normal distribution with the probability mass function of the binomial distribution. To state the theorem of the global error of one-step methods we will need the following definition of the local discretization error (see \cite[pp. 473-474]{stoer}):

\begin{definition}[Local discretization error]
  The \emph{local discretization error} is the error of the one-step method which is produced during a step of approximation. Given an one-step method $M(x,y,h,f)$ its local discretization error at the point $x_k$ is the value

  \begin{align}
    \tau(x_k) = \frac{y(x_{k+1})-y(x_k)}h - M(x_k,y(x_k),h,f)
  \end{align}
\end{definition}

\noindent Therefore $\tau$ fulfills:

\begin{align}
  \underbrace{y(x_{k+1})}_{\text{exact solution}} = \underbrace{y(x_k) + M(x_k,y(x_k),h,f)h}_{\text{approximation of one-step method}} + \underbrace{\tau(x_k)h}_{\text{error}}
\end{align}

\noindent The error of an one-step method can be estimated as following~\cite[pp. 478-479]{stoer}:

\begin{theorem}[Global error of one-step method] \label{thm:ode_global_error}
  Given an ODE with initial value

  \begin{align}
    y'(x)=f(x,y(x)),\ y'(x_0)=y_0
  \end{align}

  which has the exact solution $y(x)$. Let $M(x,y,h,f)$ be a continuous one-step method for this ODE with step size $h > 0$. Let $M(x,y,h,f)$ fulfill the Lipschitz condition\footnote{See \cite[p. 467]{stoer} and \cite{wiki:lipschitz} for more details about Lipschitz continuity.} regarding the variable $y$ with the Lipschitz constant $L > 0$, i.e. for all $x,y_1,y_2\in\R$ we have
  \begin{align}
    \abs{M(x,y_1,h,f)-M(x,y_2,h,f)} \le L \abs{y_1-y_2}
  \end{align}
  Let the absolute value of the local discretization error be bounded by $C h^p$ for a $C > 0$ and $p\in \N$:
  \begin{align}
    \abs{\tau(x_k)} = \abs{\frac{y(x_{k+1})-y(x_k)}h - M(x_k,y(x_k),h,f)} \le Ch^p
  \end{align}
  Then the global error $\abs{y(x_k)-y_k}$ is bounded by $\frac 1L \br{\fe{L\abs{x_k-x_0}}-1} C h^p$:
  \begin{align}
    \abs{y(x_k)-y_k} \le \frac{\fe{L\abs{x_k-x_0}}-1}{L} \cdot Ch^p
  \end{align}
\end{theorem}
\todo{premises all right?!}

We may also have an one-step method backwards by setting $h < 0$. The above definitions and the  above theorem \ref{thm:ode_global_error} are applicable in this case.

\section{Local version of de Moivre-Laplace}

We will prove the local version of the de Moivre-Laplace theorem in the following form

\begin{theorem}
  ...
\end{theorem}

This proof will heavily bases on the theory of one-step methods for approximating ordinary differential equations. Thereby we will revert the perspective: Whereas in the theory of ODEs one-step methods are approximations of unknown solutions of ODEs, we will consider the probability mass function of the binomial distribution as the result of an one-step method. This one-step method corresponds to an ODE and the solution to this ODE will be an approximation to the probability mass function. Thus we flip the role of the exact solution and the approximation. This makes this proof in particular interesting and my be a method which might be applicable in other proofs, too.

\todo{$\xrnd{}$ über $\rnd{}$ definieren.}

\begin{proof}
  \newcommand*{\knp}{\tilde k}
  \newcommand*{\xnp}{\x[\knp]}
  We start with investigating $\bb{\x[k+1]}-\bb{\x[k]}$ (let $\pol{\x}$ be the set of all polynomials in $\x$):

  \begin{align}
    \bb{\x[k+1]} - \bb{\x[k]} &= \binom n{k+1} p^{k+1}q^{n-k-1} - \binom nk p^k q^k \nl
    &= \binom nk p^k q^{n-k} \br{\frac{\binom n{k+1}p}{\binom nk q} -1} \nl
    &= \bb{\x[k]} \br{\frac{k! (n-k)! p}{(k+1)!(n-k-1)! q} -1} \nl
    &= \bb{\x[k]} \br{\frac{(n-k)p}{(k+1)q}-1} \nl
    &
    \begin{comment}
      k = np + \x[k]\sqrt{npq}
    \end{comment} \nl
    &= \bb{\x[k]} \br{\frac{npq-\x[k]p\sqrt{npq}}{npq + \x[k]q\sqrt{npq}+q}-1} \nl
    &= \bb{\x[k]} \br{\frac{1-\x[k]ph}{1+\x[k]qh+qh^2}-1} \nl
    &
    \begin{comment}
      \text{\todo{}} \frac1{1+x} = 1 - x+ \bigo{x^2}
    \end{comment} \nl
    &= \bb{\x[k]} \br{\br{1-\x[k]ph}\br{1+\x[k]qh + \bigo{\pol{x}h^2}}-1} \nl
    &= \bb{\x[k]} \br{-\x[k]h + \bigo{\pol{\x}h^2}} \nl
    &= -\x[k]\bb{\x}h + \bigo{\bb{\x}\pol{\x}h^2}
  \end{align}
  Because \todo{Bernstein inequality -> cite + reference}
  \begin{align}
    \bb{\x} \le \P{\BBs \ge \abs{x_k}} \le 2\fexp{-\frac{\x^2}4}
  \end{align}
  and $\lim_{x\to\pm\infty} p(x)\fexp{-\frac{x^2}4}=0$ for any polynomial $p(x)$, there is for any polynomial $p(\x)\in\pol{\x}$ a constant $A_n$ such that
  \begin{align}
    \forall k\in\Z: \abs{\bb{\x}p(\x)} \le A
  \end{align}
  so that
  \begin{align}
    \bb{\x[k+1]} &= \bb{\x[k]} -\x[k]\bb{\x[k]} h + \bigabs{A h^2} \bnl
    &
    \begin{comment}
      \text{Set } f(x,y) = -xy
    \end{comment} \bnl
    &= \bb{\x[k]} - \br{\f{\x[k], \bb{\x[k]}} + \bigabs{A h}}h \bnl
    &
    \begin{comment}
      \text{Set } M(x,y,h,f) = f(x,y) + \bigabs{Ah}
    \end{comment} \bnl
    &= \bb{\x[k]} + M(\x[k], \bb{\x[k]}, f, h) h
  \end{align}

  Therefore we can think about $\bb{\x}$ as the results of the one-step method $M(x,y,f,h)=f(x,y)+\bigabs{Ah}$ whereby we can choose any $\bb{\x}$ as the initial value. We take $\bb{\xnp}$ for $\knp=\rnd{np}$ so that $\xnp$ is the $\x$ which is closest to $0$. The one-step method now corresponds to the ODE with initial condition
  \newcommand*{\phin}[2][n]{y_{#1}\left({#2}\right)}
  \newcommand*{\dphin}[2][n]{y_{#1}'\left({#2}\right)}
  \begin{align}
    \dphin{x}=f(x,\phin{x})=-x\phin{x},\ \phin{\xnp} = \bb{\xnp}
  \end{align}
  Its solution is\todo{proof?!}
  \begin{align}
    \phin{x} = \bb{\xnp}\fexp{-\frac{x^2}2+\frac{\xnp^2}2}
  \end{align}

  In order to estimate the global error of the one-step method with theorem \ref{thm:ode_global_error} we need to estimate the local discretization error. We get by using the taylor series
  \begin{align}
    \phin{x+h} = \phin{x} -x\phin{x}h+\br{\tilde x^2-1}\phin{\tilde x} h^2
  \end{align}
  with $\tilde x$ being in the interval bounded by $x$ and $x+h$. Because $\lim_{x\to\pm\infty} \br{x^2-1}\phin{x}=0$ ($\phin{x} \in \bigo{\exp{-\frac{x^2}2}}$ for $x\to\pm\infty$) there is a $B > 0$ independent of $x$ with $\abs{\br{x^2-1}\phin{x}} \le B$ for all $x\in\R$. Thus
  \begin{align}
    \frac{\phin{x+h}-\phin{x}}h &= -x\phin{x}+\br{\tilde x^2-1}\phin{x} h \\
    &\in -x\phin{x} + \bigabs{Bh}
  \end{align}
  Now we can estimate the local discretization error
  \begin{align}
    \abs{\tau(\x)} &= \abs{\frac{\phin{\x[k]+h}-\phin{\x}}h -  M(\x,\phin{\x},f,h)} \nl
    &\in \abs{-\x\phin{\x} + \bigabs{Ah} +\x\phin{\x} - \bigabs{Bh}} \nl
    &= \abs{\bigabs{Ah}-\bigabs{Bh}} \nl
    &\subseteq \bigabs{(A+B)h} \nl
    &
    \begin{comment}
      \text{Set } C=A+B
    \end{comment} \nl
    &= \bigabs{Ch}
  \end{align}
  from which follows $\abs{\tau(\x)} \le Ch$.

  Besides approximating $\abs{\tau(x)}$ we also need to prove, that $f(x,y)$ is Lipschitz continuous regarding $y$ with finding its Lipschitz constant. We have
  \begin{align}
    \abs{f(x,y_1)-f(x,y_2)} &= \abs{-xy_1+xy_2} \nl
    &= \abs{x} \cdot \abs{y_1-y_2}
  \end{align}
  So $f:\R^2\to\R$ is not Lipschitz continuous regarding $y$ because for any $y_1\neq y_2$ we have
  \begin{align}
    \lim_{x\to\pm\infty} \frac{\abs{f(x,y_1)-f(x,y_2)}}{\abs{y_1-y_2}} = \infty
  \end{align}
  But we can restrict $f$ to the domain $\interval{-\x}{\x}\times \R$. \todo{theorem anpassen}. On this domain the restriction of $f$ is Lipschitz continuous regarding second argument with Lipschitz constant $\abs{\x}$. Because $\x$ might be zero, we will take $L=\max\{\x,1\}$ as Lipschitz constant. From the theorem \ref{thm:ode_global_error} about the global error of one-step methods we can conclude
  \begin{align}
    \abs{\bb{\x}-\phin{\x}} &\le \frac{\fe{L\abs{\x-\xnp}}-1}L Ch \nl
    &
    \begin{comment}
      \abs{\x-\xnp} \le \abs{\x}+\abs{\xnp} \le \abs{\x}+h
    \end{comment} \nl
    &\le \frac{\fe{L\abs{x}+Lh}-1}L Ch
  \end{align}
  Therefore
  \begin{align}
    \bb{x} = \phin{\x} + \bigabs{\frac{\fe{L\abs{\x}+Lh}-1}L Ch}
  \end{align}
\end{proof}

  \chapter{Conclusion}

In this thesis I proposed a new proof for the de Moivre-Laplace theorem, which could represent a valid alternative to the standard proof concerning error estimation and concerning comprehensibility in class or in textbooks. 

However, compared with the classic method, using Stirling’s formula, my alternative proof could not yield better approximations or smaller error bounds. This was already prevalent when testing it against the local version of the de Moivre-Laplace theorem.

However the new proof has some clear advantages over the standard one:

\begin{enumerate}
  \item My new proof only uses the following property of the binomial distribution
    \begin{align}
      \bb{\x[k+1]} \in \bb{\x[k]} - \x[k]\bb{\x[k]} h + \bigo{\bb{\x[k]} \pol{\x[k]} h^2}
    \end{align}
    Therefore, the same proof applies to each discrete probability distribution fulfilling this property.

  \item My alternative proof connects the theory of ordinary differential equations in an interesting way with probability theory: The one-step method corresponds to the given function and the solution of the ODE is the approximation. Thus, the role of approximated function and approximation is exchanged.

  \item Eventhough, it is a very subjective decision to rank one proof more understandable than another \cite{tampis:understandability}, the new proof arguably is easy to understand and intuitive. Therefore, it could be used in courses or textbooks for undergraduates. An investigation on this issue is out of scope for this thesis.
\end{enumerate}

Eventhough my alternative proof does not result in better approximations or smaller error bounds, the presented proof is a valuable alternative for teaching and understanding the de Moivre-Laplace theorem and an addition to the classic method. 


  \begin{appendices}
    \chapter{The Berry-Esseen theorem}

In this chapter I present you the Berry-Essen theorem which not only proves the central limit but also gives an upper bound for the maximal error of the normal approximation. In the proof of this theorem I will mainly follow Nourdin and Peccati\cite[p. 71 ff.]{nourdin}\todo{Bessere Referenz - Originalquelle?}, who use the Method of Stein.

\section{CLT and Berry-Esseen bounds}

From the central limit theorem we know, that each standardized sum $S_n = \tfrac 1{\sqrt n\sigma} \left(\sum_{k=1}^n x_k - n\mu\right)$ of i.i.d real valued random variables $\seq{X_n}$ with finite expected value $\mu=\E{X_1}$ and finite variance $\sigma^2=\E{X_1^2}-\E{X_1}^2$ converges in distribution to the standard normal distribution \refneeded. So we have $\lim_{n\to\infty} \P{S_n\le x} = \Phi(x)$ for all $x\in\R$. Thus we can use the normal distribution to approximate the cumulative distribution function of $S_n$.

Unfortunately not each proof of the central limit theorem provides an estimate of the error $\abs{\P{S_n\le x}-\Phi(x)}$ \refneeded. Here comes the Berry-Esseen theorem into play. It provides an upper bound for the maximal error between $\P{S_n \le x}$ and $\Phi(x)$ \todo{$\Phi$ muss eingeführt werden}. Therefore the Berry-Esseen theorem needs the finiteness of the third moment for each $X_n$ as an additional premise compared to the central limit theorem. The Berry-Esseen theorem states\cite[p. 71]{nourdin}:

\begin{theorem}[CLT and Berry-Esseen bounds]
  Let $\seq{X_n}$ be a sequence of standardized i.i.d real valued random variables with finite third moment. Thus we have $\E{X_1}=0$, $\E{X_1^2}=1$ and $\E{|X_1|^3} < \infty$. Let $S_n$ be the associated sequence of normalized partial sums, i.e

  \begin{align}
    S_n = \frac{1}{\sqrt n} \sum_{k=1}^n X_k
  \end{align}

  The partial sums $S_n$ converge in law to the standard normal distribution. Moreover there is a $C > 0$ not depending on $n$ nor $\seq{X_n}$ such that

  \begin{align}
    \sup_{x\in\R} \abs{\P{S_n\le x}-\Phi(x)} \le \frac{C \E{|X_1|^3}}{\sqrt n}
  \end{align}
\end{theorem}

\begin{remark}
  The currently best known estimate for the best choice of $C$ is $C \ge \Cmin$\cite{esseen1956} and $C \le \Cmax$\cite{shevtsova2011}.
\end{remark}

\begin{remark}[Berry-Esseen theorem for the binomial distribution]
  If for example $\seq{X_n}$ are binomial trials with the probability of success $p \in (0,1)$ the third moment of all $X_n$ are finite. With $q=1-p$ we get \refneeded

\begin{align}
  \E{X_1}=\frac{p^2+q^2}{\sqrt{pq}}
\end{align}

Thus the Berry-Esseen theorem yields the following estimate for the normalized binomial distribution

\begin{align}
  \sup_{x\in\R} \abs{\P{S_n\le x}-\Phi(x)} \le \frac{C(p^2+q^2)}{\sqrt{npq}}
\end{align}

\noindent Thereby the best choice of $C$ fulfills $C \le \Cmax$.
\end{remark}

\section{The method of Stein}

\subsection{The basic idea of the method of Stein}

We need to estimate $\sup_{x\in\R} \abs{\P{S_n\le x}-\Phi(x)}$ to prove the Berry-Esseen theorem. To do so we first rewrite this supremum norm:

\begin{align}
   & \sup_{x\in\R} \abs{\P{S_n\le x}-\Phi(x)} \nl
   &
   \begin{comment}
     \text{Let } N \text{ have a standard normal distribution}
   \end{comment} \nl
  =& \sup_{x\in\R} \abs{\int_{-\infty}^x \d{S_n} - \int_{-\infty}^x \d{N}} \nl
  =& \sup_{x\in\R} \abs{\int_{\R} \cfs{x} \d{S_n} - \int_{\R} \cfs{x} \d{N}} \nl
  =& \sup_{x\in\R} \abs{\E{\cf{x}{S_n}} - \E{\cf{x}{N}}} \nl
   &
   \begin{comment}
     \text{Let }\mathcal H=\left\{\cfs{x} : x\in\R\right\}
   \end{comment} \nl
  =& \sup_{h\in\mathcal H} \abs{\E{h(S_n)}-\E{h(N)}}
\end{align}

Now we can use the method of Stein to transform the difference $\E{h(S_n)}-\E{h(N)}$ in a way that $N$ does not occur any more. Here we take the solution $f$ of the following so called \emph{Stein's equation}:

\begin{align}
  f'(x)=xf(x) + h(x) - \E{h(N)}
\end{align}

Let's assume that this ordinary differential equation has a solution $f$. We get for any random variable $Y$:

\begin{align}
  &               & f'(x) - xf(x) & = h(x) - \E{h(N)} \nl
  &  \implies\ & \int_{\R} (f'(t)-t f(t)) \d{Y}(t) &= \int_{\R} (h(t)-\E{h(N)}) \d{Y}(t) \nl 
  &  \implies\ & \E{f'(Y)-Y f(Y)} &= \E{h(Y)}-\E{h(N)}
\end{align}

So instead of $\abs{\E{h(S_n)}-\E{h(N)}}$ we can estimate $\abs{\E{f'(S_n)-S_n f(S_n)}}$ which does not contain $N$ anymore. Thereby we will use some properties of $f$ which follow from the Stein's equation, namely $\snorm{f} \le \sqrt{\tfrac \pi2}$ and $\snorm{f'} \le 2$.

Before I continue with the proof of the Berry-Esseen theorem I want to concrete what I have mentioned in this section.

\subsection{Stein's equation and its solutions}

For this and the following section we assume that $N$ is a random variable with a standard normal distribution. We have

\begin{definition}[Stein's equation]
  Let $h:\R \to \R$ be a Borel function such that $\E{h(N)}$ exists and $\abs{\E{h(N)}} < \infty$. The \emph{Stein's equation} associated to $h$ is the following ordinary differential equation

  \begin{align}
    f'(x) = x f(x) + h(x) - \E{h(N)}
  \end{align}
\end{definition}

The following proposition shows that the Stein's equation always has a solution and provides an explicit form for it:

\begin{proposition}[Solutions for the Stein's equation]
  Let $f_h$ be defined by

  \begin{align}
    f_h(x) = \e{\frac{x^2}{2}} \int_{-\infty}^x \Big[h(y)-\E{h(N)}\Big] \e{-\frac{y^2}{2}}\d{y}
  \end{align}

  \noindent Every solution to the Stein's equation has the form

  \begin{align}
    f(x) = c \e{\frac{x^2}{2}} + f_h(x)
  \end{align}

  \noindent with $c \in \R$.
\end{proposition}

\begin{proof}
   Because $\E{h(N)}$ exists \todo{bessere Formulierung}, the function $y\mapsto h(y)\e{-\frac{y^2}{2}}$ is integrable so that also for each $x\in\R$ the integral ${\int_{-\infty}^x \Big[h(y)-\E{h(N)}\Big]\e{-\frac{y^2}{2}}\d{y}}$ exists. Thus $f_h$ is well defined. From the Stein's equation follows

  \begin{align}
    &      & f'(x)-xf(x) & = h(x) - \E{h(N)} \nl
    & \iff & \left[ f'(x) - xf(x) \right] \e{-\frac{x^2}{2}} &= \Big[h(x) - \E{h(N)}\Big] \e{-\frac{x^2}{2}} \nl
    & \iff & \partial_x \left( \e{-\frac{x^2}2} f(x) \right) &= \Big[h(x) - \E{h(N)}\Big] \e{-\frac{x^2}{2}} \nl
    & \iff & \e{-\frac{x^2}{2}} f(x) & = \int_{-\infty}^x \Big[ h(y) - \E{h(N)} \Big] \e{-\frac{y^2}{2}} \d{y} + c \nl
    & \iff & f(x) & = f_h(x) + c \e{\frac{x^2}{2}}
  \end{align}
\end{proof}

In the proof of the Berry-Esseen theorem we will need a estimate of $\snorm{f_h}$ and $\snorm{f_h'}$. Therefore we need to restrict the values of $h$ in the interval $h$. After this restriction we get

\begin{proposition}[Stein's bounds]
  Let $h:\R \to [0,1]$ be a Borel-function such that $\E{h(N)}$ exists with $\E{h(N)}<\infty$. The solution $f_h$ of Stein's equation given in the above proposition fulfills

  \begin{align}
    \snorm{f_h} \le \sqrt{\frac \pi2} \text{ and } \snorm{f_h'} \le 2
  \end{align}
\end{proposition}

\begin{proof}
  We start with proving $\snorm{f_h}\le \sqrt{\frac{\pi}{2}}$. First get the estimate

  \begin{align} \label{eq:estimate1}
    \abs{f_h(x)} & \le \abs{\e{-\frac{x^2}{2}} \int_{-\infty}^x \Big[ h(y) - \E{h(N)} \Big] \e{-\frac{y^2}{2}} \d{y} } \nl 
    &\le \e{\frac{x^2}{2}} \int_{-\infty}^x \abs{h(y)-\E{h(N)}} \e{-\frac{y^2}{2}}\d{y} \nl
    &
    \begin{comment}
      h(y), \E{h(N)} \in [0,1] \implies \abs{h(y)-\E{h(N)}} \le 1
    \end{comment} \nl
    & \le \e{\frac{x^2}{2}} \int_{-\infty}^x \e{-\frac{y^2}{2}} \d{y}
  \end{align}
  
  \noindent We also have

  \begin{align}
    \int_{-\infty}^x \Big[h(y)-\E{h(N)}\Big] \e{-\frac{y^2}{2}} \d{y} = -\int_{x}^\infty \Big[h(y)-\E{h(N)}\Big] \e{-\frac{y^2}{2}} \d{y}
  \end{align}

  \noindent because $\int_{\R} \Big[h(y)-\E{h(N)}\Big] \e{-\frac{y^2}{2}} \d{y}=0$ \todo{$\d{N(y)}$ richtig?}:

  \begin{align} \label{eq:estimate2}
    \int_{\R} \Big[ h(y)-\E{h(N)} \Big] \e{-\frac{y^2}{2}} \d{y} & = \int_{\R} h(y) \e{-\frac{y^2}{2}} \d{y} - \int_{\R} \E{h(N)} \e{-\frac{y^2}2} \d{y} \nl
    &= \sqrt{2\pi} \int_{\R} h(y) \frac{1}{2\pi} \e{-\frac{y^2}{2}} \d{y} - \sqrt{2\pi} \E{h(N)} \int_{\R} \frac{1}{\sqrt{2\pi}} \e{-\frac{y^2}{2}} \d{y} \nl
    &= \sqrt{2\pi} \int_{\R} h(y) \d{N(y)} - \sqrt{2\pi} \E{h(N)} \int_{\R} \d{N(y)} \nl
    &= \sqrt{2\pi}\E{h(N)} - \sqrt{2\pi}\E{h(N)} \nl
    &= 0
  \end{align}

  \noindent Thus we also could estimate $\abs{f_h(x)}$ via

  \begin{align}
    \abs{f_h(x)} & \le \abs{\e{-\frac{x^2}{2}} \int_{-\infty}^x \Big[ h(y) - \E{h(N)} \Big] \e{-\frac{y^2}{2}} \d{y} } \nl 
    & \le \abs{-\e{-\frac{x^2}{2}} \int_{x}^\infty \Big[ h(y) - \E{h(N)} \Big] \e{-\frac{y^2}{2}} \d{y} } \nl 
    &\le \e{\frac{x^2}{2}} \int_{x}^\infty \abs{h(y)-\E{h(N)}} \e{-\frac{y^2}{2}}\d{y} \nl
    &\le \e{\frac{x^2}{2}} \int_{x}^\infty \e{-\frac{y^2}{2}} \d{y}
  \end{align}
 
  \noindent From \eqref{eq:estimate1} and \eqref{eq:estimate2} follows

  \begin{align}
    \abs{f_h(x)} & \le \min\left\{ \e{\frac{x^2}{2}} \int_{-\infty}^x \e{-\frac{y^2}{2}} \d{y}, \e{\frac{x^2}{2}} \int_x^\infty \e{-\frac{y^2}{2}} \d{y} \right\} \nl
    & \le \e{\frac{x^2}{2}} \min\left\{ \int_{-\infty}^x \e{-\frac{y^2}{2}} \d{y}, \int_x^\infty \e{-\frac{y^2}{2}} \d{y} \right\} \nl
    &
    \begin{comment}
      y \mapsto \e{-\frac{y^2}{2}} \text{ is even}
    \end{comment} \nl
    & = \underbrace{\e{\frac{x^2}{2}} \int_{\abs{x}}^\infty \e{-\frac{y^2}{2}} \d{y}}_{=: s(x)} \nl
  \end{align}

  \noindent To find the maximum of $s(x)$ we first compute $s'(x)$ for $x > 0$:

  \begin{align}
    s'(x) &= \partial_x \left( \e{\frac{x^2}{2}} \int_x^\infty \e{-\frac{y^2}{2}} \d{y} \right) \nl
    &= x\e{\frac{x^2}{2}} \int_x^\infty \e{-\frac{y^2}{2}} \d{y} - 1 \nl
    &= \e{\frac{x^2}{2}} \int_x^\infty x\e{-\frac{y^2}{2}} \d{y} - 1 \nl
    &
    \begin{comment}
      x \le y \text{ on } [x,\infty)
    \end{comment} \nl
    &\le \e{\frac{x^2}{2}} \int_x^\infty y\e{-\frac{y^2}{2}} \d{y} - 1 \nl
    &= \e{\frac{x^2}{2}} \left[-\e{-\frac{y^2}{2}}\right]_x^\infty \d{y} - 1 \nl
    &= \e{\frac{x^2}{2}} \e{-\frac{x^2}{2}} - 1 \nl
    &= 0
  \end{align}

  Thus $s'(x) \le 0$ for $x \ge 0$ so that $s(x)$ is monotone decreasing on $[x,\infty)$. Because $s(x)$ is even (i.e. $s(-x)=s(x)$) we have $s'(x) \ge 0$ for $x < 0$ which implies that $s(x)$ is monotone increasing on $\R^{-}_0$. Altogether $s(x)$ has its global maximum at $x=0$ which is

  \begin{align}
    s(0) = \int_0^\infty \e{-\frac{y^2}{2}} \d{y} = \frac{1}{2} \int_{-\infty}^\infty \e{-\frac{y^2}{2}} \d{y} = \frac{1}{2} \sqrt{2\pi} = \sqrt{\frac{\pi}{2}}
  \end{align}

  \noindent Finally we have

  \begin{align}
    \abs{f_h(x)} \le s(x) \le \sqrt{\frac \pi 2}
  \end{align}

  \noindent To prove the second estimate $\snorm{f_h'} \le 2$ we can start with Stein's equation:

  \begin{align}
    \abs{f_h'(x)} & = \abs{x f_h(x) + h(x) - \E{h(N)}} \nl
    & = \abs{x \e{\frac{x^2}{2}} \int_{-\infty}^x \Big[ h(y) - \E{h(N)} \Big] \e{-\frac{y^2}{2}} \d{y}} \nl
    & \le \abs{x} \e{\frac{x^2}{2}} \int_{-\infty}^x \abs{h(y)-\E{h(N)}} \e{-\frac{y^2}{2}} \d{y}  + \abs{h(y)-\E{h(y)}} \nl
    &
    \begin{comment}
      \abs{h(y)-\E{h(N)}} \le 1
    \end{comment} \nl
    & \le \abs{x} \e{\frac{x^2}{2}} \int_{-\infty}^x \e{-\frac{y^2}{2}} \d{y}  + 1 \nl
  \end{align}

  \noindent Again we can use the following equation in the estimation of $\abs{f_h'(x)}$

  \begin{align}
    \int_{-\infty}^x \Big[h(y)-\E{h(N)}\Big] \e{-\frac{y^2}{2}} \d{y} = -\int_{x}^\infty \Big[h(y)-\E{h(N)}\Big] \e{-\frac{y^2}{2}} \d{y}
  \end{align}

  \noindent No we get

  \begin{align}
    \abs{f_h'(x)} \le \abs{x} \e{\frac{x^2}{2}} \int_x^\infty \e{-\frac{y^2}{2}} \d{y}  + 1 \nl
  \end{align}

  \noindent Altogether we have

  \begin{align}
    \abs{f_h'(x)} & \le \abs{x} \e{\frac{x^2}{2}} \min\left\{ \int_{-\infty}^x \e{-\frac{y^2}{2}} \d{y},  \int_x^\infty \e{-\frac{y^2}{2}} \d{y} \right\} + 1 \nl
    & = \abs{x} \e{\frac{x^2}{2}} \int_{\abs{x}}^\infty \e{-\frac{y^2}{2}} \d{y} + 1 \nl
    & = \e{\frac{x^2}{2}} \int_{\abs{x}}^\infty \abs{x} \e{-\frac{y^2}{2}} \d{y} + 1 \nl
    &
    \begin{comment}
      \abs{x} \le y \text{ on interval } \left[\abs{x},\infty\right)
    \end{comment} \nl
    & = \e{\frac{x^2}{2}} \int_{\abs{x}}^\infty y \e{-\frac{y^2}{2}} \d{y} + 1 \nl
    & = \e{\frac{x^2}{2}} \e{-\frac{x^2}{2}} + 1 \nl
    & = 2 \nl
  \end{align}

  \noindent This proves $\snorm{f_h'} \le 2$.
\end{proof}

\section{Proof of the Berry-Esseen theorem}



\newpage

\section{Umschreibung von $E[h_{z,\epsilon}(V_n)]-E[h_{z,\epsilon}(N)]$}

Im folgenden setzen wir

\begin{enumerate}
\item $h:=h_{z,\epsilon}$
\item $f:=f_h$
\item $\tilde f(x):=xf(x)$
\end{enumerate}

\noindent Es ist

\begin{align*}
E[h(V_n)]-E[h(N)] &= E[f'(V_n)-V_nf(V_h)] \\
&= E\left[f'(V_n)-\sum_{k=1}^n \frac{Y_k}{\sqrt n} f(V_n)\right] \\
&= \sum_{k=1}^n E\left[f'(V_n)\frac 1n-\frac{Y_k}{\sqrt n} f(V_n)\right] \\
&\left\downarrow\ E\left[f\left(V_n^k\right)Y_k\right]=E\left[f\left(V_n^k\right)\right]E\left[Y_k\right]=0\right.\\
&= \sum_{k=1}^n E\left[f'(V_n)\frac 1n-\frac{Y_k}{\sqrt n} f(V_n)+\frac{Y_k}{\sqrt n} f(V_n^k)\right] \\
&= \sum_{k=1}^n E\left[f'(V_n)\frac 1n-\frac{Y_k}{\sqrt n} \left(f(V_n)-f(V_n^k)\right)\right] \\[1em]
&\left\downarrow\ \begin{array}{rl}
	f(y)-f(x) &= \int_x^y f'(s) ds \\[0.5em]
	&= (y-x) \int_0^1 f'(x+(y-x)s)ds \\[0.5em]
	&= (y-x) E\left[f'(x+(y-x)\Theta\right]
\end{array} \right.\\[1em]
&= \sum_{k=1}^n E\left[f'(V_n)\frac 1n-\frac{Y_k^2}{n} E\left[f'\left(V_n^k+\Theta\frac{Y_K}{\sqrt n}\right)\right]\right] \\[1em]
&\left\downarrow\ \begin{array}{rl}
E\left[\frac{Y_k^2}{n} E\left[f'\left(V_n^k+\Theta\frac{Y_K}{\sqrt n}\right)\right]\right] &= \frac{E\left[Y_k^2\right]}{n} E\left[f'\left(V_n^k+\Theta\frac{Y_K}{\sqrt n}\right)\right] \\[0.5em]
&= \frac{1}{n} E\left[f'\left(V_n^k+\Theta\frac{Y_K}{\sqrt n}\right)\right]
\end{array}\right.\\[1em]
&= \sum_{k=1}^n \frac 1n E\left[f'(V_n)-f'\left(V_n^k+\Theta\frac{Y_K}{\sqrt n}\right)\right] \\[1em]
&\left\downarrow\ f'(x)=\tilde f(x) + h(x)-E[h(N)] \right. \\
&= \sum_{k=1}^n \frac 1n E \left[\begin{array}{l} \tilde f(V_n) + h(V_n) - E[h(N)] \\[0.3em]
-\tilde f\left(V_n^k +\Theta \frac{Y_k}{\sqrt n}\right) -h\left(V_n^k +\Theta \frac{Y_k}{\sqrt n}\right) + E[h(N)]\end{array}\right]\\[1em]
&= \sum_{k=1}^n \frac 1n E \left[\begin{array}{l} \tilde f(V_n) + h(V_n) \\[0.3em]
-\tilde f\left(V_n^k +\Theta \frac{Y_k}{\sqrt n}\right) -h\left(V_n^k +\Theta \frac{Y_k}{\sqrt n}\right)\end{array}\right]\\[1em]
&= \sum_{k=1}^n \frac 1n E \left[\begin{array}{ll} \tilde f(V_n) &-\tilde f\left(V_n^k\right)\\[0.3em]
-\tilde f\left(V_n^k +\Theta \frac{Y_k}{\sqrt n}\right) &+ \tilde f\left(V_n^k\right)\\[0.3em]
h(V_n) & - h\left(V_n^k\right)\\[0.3em]
 -h\left(V_n^k +\Theta \frac{Y_k}{\sqrt n}\right) & + h\left(V_n^k\right)\end{array}\right]
\end{align*}

\section{Abschätzung der einzelnen Terme}

\subsection{Abschätzung des ersten Terms}

\begin{align*}
\left|E\left[\tilde f(V_n)-\tilde f\left(V_n^k\right)\right]\right| &\le E\left[\left|\tilde f(V_n)-\tilde f\left(V_n^k\right)\right|\right] \\[1em]
& \left\downarrow\ \begin{array}{rl}
|\tilde f(x) - \tilde f(y)| & = |xf(x)-yf(y)|\\[0.3em]
& =|xf(x)-yf(x)+yf(x)-yf(y)| \\[0.3em]
& =|f(x)(x-y) + y(f(x)-f(y))| \\[0.3em]
& \le |f(x)||x-y|+|y||f(x)-f(y)| \\[0.3em]
& \le \|f\|_\infty |x-y|+ |y| \|f'\|_\infty |x-y| \\[0.3em]
& \le \sqrt{\tfrac \pi2}|x-y|+2|y||x-y| \\[0.3em]
& = \left(\sqrt{\tfrac\pi2}+2|y|\right)|x-y|
\end{array} \right.\\[1em]
&\le E\left[\left(\sqrt{\tfrac \pi2}+2\left|V_n^k\right|\right)\left|V_n-V_n^k\right|\right] \\[0.5em]
&\le E\left[\left(\sqrt{\tfrac \pi2}+2\left|V_n^k\right|\right)\frac{|Y_k|}{\sqrt n}\right] \\[0.5em]
&\le \tfrac 1{\sqrt n} \left(\sqrt{\tfrac \pi2}+2E\left[\left|V_n^k\right|\right]\right)E[|Y_k|] \\[0.5em]
&\left\downarrow\ \begin{array}{l}
E[|Y_k|] \le E[Y_k^2]^{\tfrac 12} = 1 \\[0.3em]
E[|V_n^k|] \le E[{V_n^k}^2]^{\tfrac 12} = 1
\end{array} \right.\\[0.5em]
&\le \tfrac 1{\sqrt n} \left(\sqrt{\tfrac \pi2}+2\right) \\[0.5em]
\end{align*}

\subsection{Abschätzung des zweiten Terms}

\begin{align*}
\left|E\left[\tilde f\left(V_n^k + \Theta \frac{Y_k}{\sqrt n}\right)-\tilde f\left(V_n^k\right)\right]\right| & \le \frac{E[|Y_k|]}{\sqrt n}\left(E[\Theta]\sqrt{\tfrac \pi2}+2E[\Theta]E[|V_n^k|]\right) \\
&\left\downarrow\ E[\Theta]=\tfrac 12 \land E[|V_n^k|] \le E[{V_n^k}^2]^{\tfrac 12}=1\right.\\
& \le \frac{E[|Y_k|]}{\sqrt n}\left(\frac 12\sqrt{\frac \pi2}+1\right) \\
&\left\downarrow\ E[|Y_k|] \le E[Y_k^2]^{\tfrac 12}=1\right.\\
& \le \frac{1}{\sqrt n}\left(\frac 12\sqrt{\frac \pi2}+1\right) \\
\end{align*}

\subsection{Abschätzung des dritten Terms}

\begin{align*}
\left|E\left[h(V_n)-h\left(V_n^k\right)\right]\right|&\le E\left[\left|h(V_n)-h\left(V_n^k\right)\right|\right] \\[1em]
&\left\downarrow\ \begin{array}{rl}
h(y)-h(x) & = \int_y^x h'(s) ds \\[0.5em]
&= (y-x) \int_0^1 h'(x+s(y-x))ds \\[0.5em]
&=-\frac{y-x}{2\epsilon} E\left[\mathbf{1}_{[z-\epsilon,z+\epsilon]}(x + \tilde \Theta (y-x))\right]
\end{array} \right. \\[1em]
&= \frac{1}{2\epsilon}E\left[\left|V_n-V_n^k\right|\cdot E\left[\mathbf{1}_{[z-\epsilon,z+\epsilon]}(V_n^k + \tilde \Theta (V_n-V_n^k))\right]\right] \\
&= \frac{1}{2\sqrt{n}\epsilon}E\left[\left|Y_k\right|\right]\cdot E\left[\mathbf{1}_{[z-\epsilon,z+\epsilon]}\left(V_n^k + \tilde \Theta \frac{Y_k}{\sqrt n}\right)\right] \\[1em]
&\left\downarrow\ E[|Y_1|]\le E\left[Y_1^2\right]^{\tfrac 12}=1 \right.\\[1em]
&\le \frac{1}{2\sqrt{n}\epsilon}E\left[\mathbf{1}_{[z-\epsilon,z+\epsilon]}\left(V_n^k + \tilde \Theta \frac{Y_k}{\sqrt n}\right)\right] \\
&= \frac{1}{2\sqrt{n}\epsilon}P\left(z-\epsilon \le V_n^k + \tilde \Theta \frac{Y_k}{\sqrt n} \le z+\epsilon\right) \\[1em]
&\left\downarrow\ \tilde \Theta\frac{Y_k}{\sqrt n} \in \mathbb R \right. \\[1em]
&\le \frac{1}{2\sqrt{n}\epsilon} \sup_{c\in\mathbb R} P\left(z-c-\epsilon \le V_n^k \le z-c+\epsilon\right) \\
\end{align*}

Um $P(a\le V_n^k\le b)$ abschätzen zu können, setzen wir $\tilde V_n^k = \tfrac{1}{\sqrt{n-1}} \sum_{i\neq k} Y_i$ und damit ist $V_n^k = \sqrt{1-\tfrac 1n} \tilde V_n^k$. Es ist

\begin{align*}
P(a\le V_n^k\le b) &= P\left(a \le \sqrt{1-\tfrac 1n} \tilde V_n^k \le b\right) \\
&= P\left(\frac{a}{\sqrt{1-\tfrac 1n}} \le \tilde V_n^k \le \frac{b}{\sqrt{1-\tfrac 1n}}\right) \\
&= P\left(\frac{a}{\sqrt{1-\tfrac 1n}} \le \tilde V_n^k \le \frac{b}{\sqrt{1-\tfrac 1n}}\right)- P\left(\frac{a}{\sqrt{1-\tfrac 1n}} \le N \le \frac{b}{\sqrt{1-\tfrac 1n}}\right)\\
& \quad +P\left(\frac{a}{\sqrt{1-\tfrac 1n}} \le N \le \frac{b}{\sqrt{1-\tfrac 1n}}\right) \\[1em]
& \left\downarrow\ P\left(\frac{a}{\sqrt{1-\tfrac 1n}} \le N \le \frac{b}{\sqrt{1-\tfrac 1n}}\right) \le \frac{b-a}{\sqrt{2\pi}\sqrt{1-\tfrac 1n}}\right. \\[1em]
&\le P\left(\frac{a}{\sqrt{1-\tfrac 1n}} \le \tilde V_n^k \le \frac{b}{\sqrt{1-\tfrac 1n}}\right)- P\left(\frac{a}{\sqrt{1-\tfrac 1n}} \le N \le \frac{b}{\sqrt{1-\tfrac 1n}}\right)\\
& \quad +\frac{b-a}{\sqrt{2\pi}\sqrt{1-\tfrac 1n}} \\[1em]
&\left\downarrow\ \begin{array}{rl}
P(x \le \tilde V_n^k \le y) - P(x\le N \le y) & = P(\tilde V_n^k \le y) - P(\tilde V_n^k < x) \\[0.3em]
&\quad- P(N \le y) + P(N < x) \\[0.3em]
& = P(\tilde V_n^k \le y) - P(N \le y) \\[0.3em]
&\quad+ P(N < x) - P(\tilde V_n^k < x) \\[0.3em]
&\le 2 d_\text{Kol}(\tilde V_n^k, N)= \frac{2 C_{n-1} E\left[|Y_1|^3\right]}{\sqrt{n-1}}
\end{array} \right. \\[1em]
& \le \frac{2 C_{n-1} E\left[|Y_1|^3\right]}{\sqrt{n-1}}+\frac{b-a}{\sqrt{2\pi}\sqrt{1-\tfrac 1n}}
\end{align*}

\noindent Damit

\begin{align*}
\left|E\left[h(V_n)\right] - E\left[h(N)\right]\right| & \le \frac{1}{2\sqrt{n}\epsilon} \sup_{c\in\mathbb R} P\left(z-c-\epsilon \le V_n^k \le z-c+\epsilon\right) \\[1em]
& \le \frac{C_{n-1} E\left[|Y_1|^3\right]}{\sqrt n\sqrt{n-1}\epsilon}+\frac{2\epsilon}{2\epsilon\sqrt{2\pi}\sqrt{n-1}} \\
& \le \frac{C_{n-1} E\left[|Y_1|^3\right]}{\sqrt n\sqrt{n-1}\epsilon}+\frac{1}{\sqrt{2\pi}\sqrt{n-1}} \\
\end{align*}

\subsection{Abschätzung des vierten Terms}

\begin{align*}
	\left|E\left[h\left(V_n^k+\Theta\frac{Y_k^2}n\right)-h\left(V_n^k\right)\right]\right| &=\frac{1}{2\sqrt n \epsilon}\left|E\left[Y_k\Theta\cdot E\left[\mathbf{1}_[z-\epsilon,z+\epsilon]\left(V_n^k + \tilde \Theta\Theta\frac{Y_k}{\sqrt n}\right)\right]\right]\right| \\[1em]
& \le \frac{E[|Y_1|]}{4\sqrt n \epsilon} \sup_{c\in\mathbb R} P\left(z-c-\epsilon \le V_n^k \le z-c+\epsilon\right) \\[1em]
& \le \frac{1}{2\sqrt{2\pi}\sqrt{n-1}} + \frac{C_{n-1}E[|Y_1|^3]}{2\sqrt  n\sqrt{n-1}\epsilon}
\end{align*}

\section{Herleitung einer Rekursionsbeziehung für $C_n$}

\begin{align*}
\left|E[h(V_n)] - E[h(N)]\right| &= \left|\sum_{k=1}^n \frac 1n E \left[\begin{array}{ll} \tilde f(V_n) &-\tilde f\left(V_n^k\right)\\[0.3em]
-\tilde f\left(V_n^k +\Theta \frac{Y_k}{\sqrt n}\right) &+ \tilde f\left(V_n^k\right)\\[0.3em]
h(V_n)& - h\left(V_n^k\right) \\[0.3em]
 -h\left(V_n^k +\Theta \frac{Y_k}{\sqrt n}\right) & + h\left(V_n^k\right) \end{array}\right] \right| \\[1em]
&\le \sum_{k=1}^n \frac 1n \left(\begin{array}{ll} E\left[\left|\tilde f(V_n)\right.\right.&\left.\left.-\tilde f\left(V_n^k\right)\right|\right]\\[0.3em]
+E\left[\left|\tilde f\left(V_n^k +\Theta \frac{Y_k}{\sqrt n}\right) \right.\right.&\left.\left.- \tilde f\left(V_n^k\right)\right|\right]\\[0.3em]
+E\left[\left|h(V_n) \right.\right.&\left.\left. - h\left(V_n^k\right)\right|\right]\\[0.3em]
+E\left[\left|h\left(V_n^k +\Theta \frac{Y_k}{\sqrt n}\right) \right.\right.&\left.\left. - h\left(V_n^k\right) \right|\right]\end{array}\right)  \\[1em]
&\le \sum_{k=1}^n \frac 1n \left(\begin{array}{l}
\left(\sqrt{\tfrac\pi2}+2\right)\frac1{\sqrt n}\\[0.3em]
+\left(\tfrac 12\sqrt{\tfrac \pi2}+1\right)\frac{1}{\sqrt n}\\[0.3em]
+\tfrac{1}{\sqrt{2\pi}\sqrt{n-1}}+\frac{C_{n-1}E[|Y_1|^3]}{\sqrt{n}\sqrt{n-1}\epsilon}\\[0.3em]
+\tfrac{1}{2\sqrt{2\pi}\sqrt{n-1}}+\frac{C_{n-1}E[|Y_1|^3]}{2\sqrt{n}\sqrt{n-1}\epsilon}
\end{array}  \right)\\[1em]
&= \begin{array}{l}
\left(\sqrt{\tfrac\pi2}+2\right)\frac1{\sqrt n}\\[0.3em]
+\left(\tfrac 12\sqrt{\tfrac \pi2}+1\right)\frac{1}{\sqrt n}\\[0.3em]
+\tfrac{1}{\sqrt{2\pi}\sqrt{n-1}}+\frac{C_{n-1}E[|Y_1|^3]}{\sqrt{n}\sqrt{n-1}\epsilon}\\[0.3em]
+\tfrac{1}{2\sqrt{2\pi}\sqrt{n-1}}+\frac{C_{n-1}E[|Y_1|^3]}{2\sqrt{n}\sqrt{n-1}\epsilon}\\[0.3em]
\end{array}  \\[1em]
&= \frac{1}{\sqrt n} \left(\begin{array}{l}\left(\sqrt{\frac\pi2}+2\right)+\frac 1n\left(\frac 12 \sqrt{\frac \pi2}+1\right)\\ +\frac{\sqrt{n}}{\sqrt{2\pi}\sqrt{n-1}} + \frac{\sqrt{n}}{2\sqrt{2\pi}\sqrt{n-1}}\end{array}\right) \\[0.5em]
&\quad + \frac{3C_{n-1}E[|Y_1|^3]}{2\sqrt{n}\sqrt{n-1}\epsilon} \\[1em]
&\left\downarrow\ E[|Y_1|^3] \ge E[Y_1^2]^{\tfrac 32}=1 \land \tfrac 1n \le 1\right.\\[1em]
&\le \frac{E[|Y_1|^3]}{\sqrt n} \left(\begin{array}{l}\sqrt{\frac\pi2}+2+\frac 12 \sqrt{\frac \pi2}+1\\ +\frac{\sqrt{n}}{\sqrt{2\pi}\sqrt{n-1}} + \frac{\sqrt{n}}{2\sqrt{2\pi}\sqrt{n-1}}\end{array}\right) \\[0.5em]
&\quad + \frac{3C_{n-1}E[|Y_1|^3]^2}{2\sqrt{n}\sqrt{n-1}\epsilon} \\[1em]
&\left\downarrow\ n \ge 2 \Rightarrow n-1 \ge \tfrac n2 \Rightarrow \sqrt{n-1} \ge \frac{\sqrt{n}}{\sqrt 2} \right. \\[1em]
&\le \frac{E[|Y_1|^3]}{\sqrt n} \left(\begin{array}{l}\sqrt{\frac\pi2}+2+\frac 12 \sqrt{\frac \pi2}+1\\ +\frac{\sqrt{2}}{\sqrt{2\pi}} + \frac{\sqrt{2}}{2\sqrt{2\pi}}\end{array}\right) \\[0.5em]
&\quad + \frac{3\sqrt{2}C_{n-1}E[|Y_1|^3]^2}{2n\epsilon}\\[1em]
&\left\downarrow\ \frac{\sqrt 2}2 \le 1 \right. \\[1em]
&\le \frac{E[|Y_1|^3]}{\sqrt n} 5{,}726\ldots + \frac{3 C_{n-1}E[|Y_1|^3]^2}{n\epsilon} \\[1em]
&\le \frac{6E[|Y_1|^3]}{\sqrt n} + \frac{3 C_{n-1}E[|Y_1|^3]^2}{n\epsilon}
\end{align*}

\section{Zusammenfassung}

\begin{align*}
d_\text{Kol}(V_n,N) & \le \sup_{z\in\mathbb R}\left|E[h_{z,\epsilon}(V_n)]-E[h_{z,\epsilon}(N)]\right|+\frac{4\epsilon}{\sqrt{2\pi}} \\
&\le 6\frac{E[|Y_1|^3]}{\sqrt n}+\frac{3C_{n-1}E[|Y_1|^3]^2}{n\epsilon}+\frac{4\epsilon}{\sqrt{2\pi}} \\
&\left\downarrow\ \epsilon = \sqrt{\frac{C_{n-1}}{n}}E[|Y_1|^3] \right. \\
&\le 6\frac{E[|Y_1|^3]}{\sqrt n}+\frac{3\sqrt{C_{n-1}}E[|Y_1|^3]}{\sqrt n}+\frac{4\sqrt{C_{n-1}}E[|Y_1|^3]}{\sqrt{2\pi}\sqrt{n}} \\
&\le \frac{E[|Y_1|^3]}{\sqrt n}\left[6+\left(3+\frac{4}{\sqrt{2\pi}}\right)\sqrt{C_{n-1}}\right] \\
\end{align*}

\noindent Damit

\begin{align*}
C_n \le 6 + \left(3 + \frac{4}{\sqrt{2\pi}}\right)\sqrt{C_{n-1}}
\end{align*}

    \chapter{Source code for plots}

\newcommand*{\src}[1]{\section{Figure \ref{#1}}

\lstinputlisting[language=Python]{"plots/#1.py"}}

\src{pmf}
\src{cdf}
\src{ml_pmf}
\src{ml_cdf}
\src{berry_diff}
\src{berry_line}
\src{berry_slope}
\src{edgeworth_diff_pdf}


    \printbibliography
    \markboth{}{}

    \backmatter
    
    \cleardoublepage
\phantomsection \label{acknowledgment}
\addcontentsline{toc}{chapter}{\protect Acknowledgment}

\chapter*{Acknowledgment}

I would like to thank my supervisor \href{http://www.mathematik.uni-muenchen.de/personen/professoren/pickl/index.html}{Prof. Dr. Peter Pickl} for his help and support. I especially want to thank him for allowing me to choose the thesis's topic freely and for continuing to allow me great freedom throughout the entire process of writing the thesis.

I also want to thank my second supervisor \href{http://homepages.physik.uni-muenchen.de/~vondelft/}{Prof. Dr. Jan von Delft} for correcting this thesis and for conducting my oral exam.
    \markboth{Acknowledgment}{Acknowledgment}
    \cleardoublepage

    \listoffigures
    \markboth{List of Figures}{List of Figures}
    \cleardoublepage

    \markboth{Statement of Authorship}{Statement of Authorship}
    \cleardoublepage
\phantomsection \label{statement_of_authorship}
\addcontentsline{toc}{chapter}{\protect Statement of Authorship}

\chapter*{Statement of Authorship}

Hiermit versichere ich, dass ich die vorliegende Masterarbeit mit dem Titel

\begin{center}
\textbf{The de Moivre-Laplace theorem}
\end{center}

\noindent selbständig verfasst und keine anderen als die angegebenen Quellen und Hilfsmittel benutzt habe.

\vspace*{2cm}

\noindent\rule{0.3\textwidth}{0.4pt}

\noindent Stephan Kulla, \newline 
München, den 30.11.2013

    \cleardoublepage
  \end{appendices}
\end{document}
