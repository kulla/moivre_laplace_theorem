\chapter{Proofs for the tools of this thesis}

In this chapter I will prove the theorems I have stated in the chapter~\ref{chapter:tools} where I presented the needed tools for this thesis.

\section{Interval arithmetic}

\begin{theorem}
  Let $a,b,\lambda\in\R$ and $\epsilon,\delta\in\Rplusnull$. We have

  \begin{enumerate}
    \item $\delta \le \epsilon \implies \an[\delta]{a} \subseteq \an[\epsilon]{a}$
    \item $\an[\epsilon]{a}+\an[\delta]{b}=\an[\epsilon+\delta]{a+b}$
    \item $\an[\epsilon]{\an[\delta]{a}} = \an[\epsilon+\delta]{a}$
    \item $b \in \an[|b|]{0}$
    \item $\an[\epsilon]{a+b} \subseteq \an[\epsilon+\abs{b}]{a}$
    \item $\lambda \an[\epsilon]{a} = \an[\abs{\lambda}\cdot \epsilon]{\lambda a}$
  \end{enumerate}
\end{theorem}


\begin{proof} ~
  \begin{enumerate}
    \item

      \begin{align}
        \an[\delta]{a} = \interval{a-\delta}{a+\delta} \subseteq \interval{a-\epsilon,a+\epsilon} = \an[\epsilon]{a}
      \end{align}

    \item Claim $\interval ab + \interval cd \subseteq \interval{a+c}{b+d}$:

      \begin{align}
        x \in \interval ab \land y \in \interval cd & \implies a \le x \le b \land c \le y \le d \nl
        & \implies a+c \le x+y \le b+d \nl
        & \implies x+y \in \interval{a+c}{b+d}
      \end{align}

      Claim $\interval{a+c}{b+d} \subseteq \interval ab + \interval cd$: Let $f$ be the function $f: \interval ab \times \interval cd \to \R$ with $f(x,y) =x+y$ so that $\interval ab + \interval cd$ is the image of $f$. The function $f$ attains the maximum $b+d$ of the interval $\interval{a+c}{b+d}$ for $(x,y) = (b,d)$ and its minimum $a+c$ for $(x,y)=(a,c)$. Because $f$ is continuous, $f$ attains each value between $a+c$ and $b+d$ and therefore each value of $\interval{a+c}{b+d}$ (intermediate value theorem~\cite{wiki:intermediatevaluetheorem}).

    \item

      \begin{align}
        \an[\epsilon]{a} + \an[\delta]{b} & = \interval{a-\epsilon}{a+\epsilon} + \interval{b-\delta}{b+\delta} \nl
        & = \interval{(a+b)-(\epsilon+\delta)}{(a+b)+(\epsilon+\delta)} \nl
        & = \an[\epsilon+\delta]{a+b}
      \end{align}

    \item

      \begin{align}
        \an[\epsilon]{\an[\delta]{a}} & = \bigcup_{x\in \interval{a-\delta}{a+\delta}} \an[\epsilon]{x} = \bigcup_{x\in\interval{a-\delta}{a+\delta}} \bigcup_{y\in\interval{-\epsilon}{\epsilon}} \{x+y\} \nl
        &= \an[\delta]{a} + \an[\epsilon]{0} = \an[\epsilon+\delta]{a}
      \end{align}
 
    \item

      This follows directly from $-\abs{b} \le b \le \abs{b}$ because $b \in \{-\abs{b}, \abs{b}\}$.

    \item

      \begin{align}
        \an[\epsilon]{a+b} \subseteq \an[\epsilon]{a+\an[\abs{b}]{0}} = \an[\epsilon]{\an[\abs{b}]{a}} = \an[\epsilon+\abs{b}]{a}
      \end{align}

    \item
      Let $\lambda \ge 0$ and let $f_\lambda$ be the function $f_\lambda : \an[\epsilon]{a}\to\R:x\mapsto \lambda x$. $f_\lambda$ attains its maximum at $\lambda(a+\epsilon)$ and its minimum at $\lambda(a-\epsilon)$. Because $f_\lambda$ is continuous and $\an[\epsilon]{a}$ is connected, also $f_\lambda(\an[\epsilon]{a})$ is connected and therefore $f_\lambda$ attains all values between $\lambda(a-\epsilon)$ and $\lambda(a+\epsilon)$ (intermediate value theorem~\cite{wiki:intermediatevaluetheorem}). Thus

      \begin{align}
        f_\lambda\left(\an[\epsilon]{a}\right) = \interval{\lambda(a-\epsilon)}{\lambda(a+\epsilon} = \interval{\lambda a - \lambda \epsilon}{\lambda a+\lambda \epsilon} = \an[\lambda \epsilon]{\lambda a} = \an[\abs{\lambda}\cdot \epsilon]{\lambda a}
      \end{align}

      Let $\lambda < 0$ and $f_\lambda$ be as above. The maximum of this function is now $\lambda (a-\epsilon)$ and its minimum is $\lambda (a+\epsilon)$. Due to the intermediate value theorem~\cite{wiki:intermediatevaluetheorem} all values between $\lambda(a+\epsilon)$ and $\lambda(a-\epsilon)$ are attained by the function. So we have

      \begin{align}
        f_\lambda\left(\an[\lambda]{a}\right) = \interval{\lambda(a+\epsilon)}{\lambda(a-\epsilon)} = \interval{\lambda a -\abs{\lambda}\cdot \epsilon}{\lambda a + \abs{\lambda}\cdot \epsilon} = \an[\abs{\lambda}\cdot \epsilon]{\lambda a}
      \end{align}

 \end{enumerate}
\end{proof}

\begin{theorem} \label{thm:multiplicative:rules}
  Let $a,b\in \Rplus$ and $\epsilon,\delta \in\Rplusnull$. We have

  \begin{enumerate}
    \item $\delta \le \epsilon \implies \ean[\delta]{a} \subseteq \ean[\epsilon]{a}$ 
    \item $\ean[\epsilon]{a} \cdot \ean[\delta]{b} = \ean[\epsilon+\delta]{a\cdot b}$
    \item $\ean[\epsilon]{\ean[\delta]{a}} = \ean[\epsilon+\delta]{a}$
    \item $\ean[\epsilon]{a} = \an[\sinh(\epsilon)a]{\cosh(\epsilon)a}$
    \item $\ean[\epsilon]{a} \subseteq \an[\left(e^\epsilon-1\right)a]{a}$
    \item $\abs{\frac ab-1} \le \epsilon \implies a\in\ean[\epsilon]{b}$
  \end{enumerate}
\end{theorem}


\begin{proof} ~
  \begin{enumerate}
    \item
      
      \begin{align}
        \ean[\delta]{a} = \interval{ae^{-\delta}}{ae^\delta} \subseteq \ean[\epsilon]{ae^{-\epsilon}}{ae^\epsilon} = \ean[\epsilon]{a}
      \end{align}

    \item Claim $\ean[\epsilon]{a}\cdot \ean[\delta]{b} \subseteq \ean[\epsilon+\delta]{ab}$:

      \begin{align}
        x \in \ean[\epsilon]{a} \land y \in \ean[\delta]{b} & \implies \left(ae^{-\epsilon} \le x \le ae^{\epsilon}\right) \land \left(be^{-\delta} \le y \le be^\delta\right) \nl
        & \implies abe^{-(\epsilon+\delta)} \le xy \le abe^{\epsilon+\delta} \nl
        & \implies xy \in \ean[\epsilon+\delta]{a\cdot b}
      \end{align}

      Claim $\ean[\epsilon+\delta]{ab} \subseteq \ean[\epsilon]{a} \cdot \ean[\delta]{b}$: Let $f$ be the function $f: \ean[\epsilon]{a} \times \ean[\delta]{b} \to \R$ with $f(x,y) =xy$ so that $\ean[\epsilon]{a} \cdot \ean[\delta]{b}$ is the image of $f$. The function $f$ attains the maximum $abe^{\epsilon+\delta}$ of the interval $\ean[\epsilon+\delta]{ab}$ for $(x,y) = \left(ae^\epsilon,be^\delta\right)$ and its minimum $abe^{-(\epsilon+\delta)}$ for $(x,y)=\left(ae^{-\epsilon},be^{-\delta}\right)$. Because $f$ is continuous, $f$ attains each value between $abe^{-(\epsilon+\delta)}$ and $abe^{\epsilon+\delta}$ and therefore each value of $\ean[\epsilon+\delta]{ab}$ (intermediate value theorem~\cite{wiki:intermediatevaluetheorem}).

    \item

      \begin{align}
        \ean[\epsilon]{\ean[\delta]{a}} & = \bigcup_{x\in \ean[\delta]{a}} \ean[\epsilon]{x} = \bigcup_{x\in\ean[\epsilon]{a}} \bigcup_{y\in\ean[\epsilon]{1}} \{x\cdot y\} \nl
        &= \ean[\delta]{a} \cdot \ean[\epsilon]{1} = \ean[\epsilon+\delta]{a}
      \end{align}

    \item

      \begin{align}
        \ean[\epsilon]{a} + \ean[\epsilon]{b} & = \interval{ae^{-\epsilon}}{ae^\epsilon} + \interval{be^{-\epsilon}}{be^\epsilon} \nl
        & = \interval{ae^{-\epsilon}+be^{-\epsilon}}{ae^\epsilon+be^\epsilon} \nl
        & = \interval{(a+b)e^{-\epsilon}}{(a+b)e^{\epsilon}} \nl
        & = \ean[\epsilon]{a+b}
      \end{align}

    \item
      
      \begin{align}
        x\in\ean[\epsilon]{a} & \iff x \in \interval{ae^{-\epsilon}}{ae^\epsilon} \nl
        & \iff x \in \an[\frac{ae^\epsilon-ae^{-\epsilon}}2]{\frac{ae^\epsilon+ae^{-\epsilon}}2} \nl
        & \iff x \in \an[a\cdot \frac{e^\epsilon-e^{-\epsilon}}2]{a\cdot \frac{e^\epsilon+e^{-\epsilon}}2} \nl
        & \iff x\in \an[a\sinh(\epsilon)]{a\cosh(\epsilon)}
      \end{align}

    \item

      \begin{align}
        \ean[\epsilon]{a} &= \an[\sinh(\epsilon)a]{\cosh(\epsilon)a} \nl
        &= \an[\sinh(\epsilon)a]{a+\cosh(\epsilon)a-a} \nl
        &\subseteq \an[\sinh(\epsilon)a+\abs{\cosh(\epsilon)a-a}]{a} \nl
      &\begin{comment} a > 0 \land \cosh(\epsilon) > 0 \implies \cosh(\epsilon)a > a \end{comment} \nl
        &= \an[\sinh(\epsilon)a+\cosh(\epsilon)a-a]{a} \nl
      &\begin{comment} \sinh(\epsilon) + \cosh(\epsilon) = \frac 12 \left(e^\epsilon-e^{-\epsilon}\right) + \frac 12 \left(e^\epsilon+e^{-\epsilon}\right) = e^\epsilon \end{comment} \nl
        &= \an[e^\epsilon a-a]{a} \nl
        &= \an[\left(e^\epsilon-1\right)a]{a}
      \end{align}

    \item

      \begin{align}
        \exp\left(\an[\epsilon]{a}\right) = \left\{ e^x : x \in \an[\epsilon]{a}\right\} = \interval{e^{a-\epsilon}}{e^{a+\epsilon}} = \interval{e^a\cdot e^{-\epsilon}}{e^a\cdot e^\epsilon} = \ean[\epsilon]{a}
      \end{align}

    \item

      \begin{align}
        \abs{\frac ab - 1} \le \epsilon & \implies 1-\epsilon \le \frac ab \le 1+\epsilon \nl
        & \implies e^{-\epsilon} \le \frac ab \le e^{\epsilon} \nl
        & \implies be^{-\epsilon} \le a \le be^\epsilon \nl
        & \implies a \in \ean[\epsilon]{b}
      \end{align}
  \end{enumerate}

\end{proof}

\section{Big Psi notation}

\begin{theorem} \label{thm:bigpsi:rules}
  Let $\seq{\epsilon_n}$ be a sequence. Let $\seq{a_n}$ and $\seq{b_n}$ be sequences with $a_n,b_n > 0$ for all $n\in\N$. Let $\lambda, \mu \in \Rplus$. The big Psi notation has the following arithmetic rules:

  \begin{enumerate}
    \item $\seq{a_n} \in \bigpsi{a_n}$
    \item $\seq{\epsilon_n} \in \bigo{a_n} \land \seq{a_n} \in \littleo{b_n} \implies \seq{\epsilon_n} \in \littleo{b_n}$
    \item $\bigpsi{a_n} \subseteq \bigo{a_n}$
    \item $\bigpsi{a_n} + \littleo{a_n} \subseteq \bigpsi{a_n}$
    \item $\bigpsi{a_n}\cdot \bigpsi{b_n} \subseteq \bigpsi{a_n b_n}$
    \item $b_n \cdot \bigpsi{a_n} \subseteq \bigpsi{b_n \cdot a_n}$
    \item $\frac{\bigpsi{a_n}}{b_n} \subseteq \bigpsi{\frac{a_n}{b_n}}$
    \item $\bigpsi{\lambda a_n} + \bigpsi{\mu a_n} \subseteq \bigpsi{(\lambda+\mu)a_n}$
    \item $e^{\littleo{1}} \subseteq \bigpsi{1}$
    \item $\seq{a_n} \in o(1) \implies e^{\bigpsi{a_n}}-1 \subseteq \bigpsi{a_n}$
  \end{enumerate}
\end{theorem}


\begin{proof} ~
  \begin{enumerate}
    \item \label{proof:littleo_rule}

      \begin{align}
        & \seq{\epsilon_n}\in\bigo{a_n} \land \seq{a_n} \in \littleo{b_n} \nl
        \implies & \left(\exists C > 0: \limsup_{n\to\infty} \abs{\frac{\epsilon_n}{a_n}} \le C\right) \land \lim_{n\to\infty} \abs{\frac{a_n}{b_n}} = 0 \nl
        \implies & \lim_{n\to\infty} \abs{\frac{\epsilon_n}{b_n}} = \lim_{n\to\infty} \underbrace{\abs{\frac{\epsilon_n}{a_n}}}_{\text{bounded}} \cdot \underbrace{\abs{\frac{a_n}{b_n}}}_{\to 0} = 0 \nl
        \implies & \seq{\epsilon_n} \in \littleo{b_n}
      \end{align}
    \item

      \begin{align}
        \seq{\epsilon_n} \in \bigpsi{a_n} & \implies \limsup_{n\to\infty} \abs{\frac{\epsilon_n}{a_n}} \le 1 \nl
                                          & \implies \exists C_\infty > 0 : \limsup_{n\to\infty} \abs{\frac{\epsilon_n}{a_n}} \le C_\infty \nl
                                          & \implies \seq{e_n} \in \bigo{a_n}
      \end{align}

    \item

      \begin{align}
        & \seq{\epsilon_n} \in \bigpsi{a_n} \land \seq{\delta_n} \in \littleo{a_n} \nl
        \implies& \left(\limsup_{n\to\infty} \abs{\frac{\epsilon_n}{a_n}} \le 1\right) \land \left(\lim_{n\to\infty} \abs{\frac{\delta_n}{a_n}} = 0\right) \nl
        \implies& \limsup_{n\to\infty} \abs{\frac{\epsilon_n+\delta_n}{a_n}} \le \limsup_{n\to\infty} \left(\abs{\frac{\epsilon_n}{a_n}}+\abs{\frac{\delta_n}{a_n}}\right) \le 1 \nl
        \implies& \seq{\epsilon_n+\delta_n} \in \bigpsi{a_n}
      \end{align}

    \item

      \begin{align}
        \seq{\epsilon_n} \in \bigpsi{a_n}  & \implies \limsup_{n\to\infty} \abs{\frac{\epsilon_n}{a_n}} \le 1 \nl
        & \implies \limsup_{n\to\infty} \abs{\frac{\epsilon_n b_n}{a_n b_n}} = \limsup_{n\to\infty} \abs{\frac{\epsilon_n}{a_n}} \le 1 \nl
        & \implies \seq{\epsilon_n b_n} \in \bigpsi{a_n b_n}
      \end{align}

    \item

      \begin{align}
        \frac{\bigpsi{a_n}}{b_n} = \bigpsi{a_n} \cdot \frac{1}{b_n} \subseteq \bigpsi{\frac{a_n}{b_n}}
      \end{align}

    \item

      \begin{align}
        & \seq{\epsilon_n} \in \bigpsi{\lambda a_n} \land \seq{\delta_n} \in \bigpsi{\mu a_n} \nl
        \implies &
        \begin{aligned}
          \limsup_{n\to\infty} \abs{\frac{\epsilon_n+\delta_n}{(\lambda+\mu)a_n}} & \le \limsup_{n\to\infty} \left( \frac{\lambda}{\lambda + \mu} \cdot \underbrace{\abs{\frac{\epsilon_n}{\lambda a_n}}}_{\le 1} + \frac{\mu}{\lambda + \mu} \underbrace{\abs{\frac{\delta_n}{\mu a_n}}}_{\le 1} \right) \nl
          &\le \frac{\lambda}{\lambda+\mu} + \frac{\mu}{\lambda+\mu} = 1
        \end{aligned} \nl
        \implies & \seq{\epsilon_n+\delta_n} \in \bigpsi{(\lambda+\delta)a_n}
      \end{align}
    \item

      \begin{align}
        \seq{\epsilon_n} \in \littleo{1} & \implies \lim_{n\to\infty} \epsilon_n = 0 \nl
        & \implies \lim_{n\to\infty} e^{\epsilon_n} = 1 \nl
        & \implies \limsup_{n\to\infty} \frac{e^{\epsilon_n}}{1} \le 1 \nl
        & \implies \seq{e^{\epsilon_n}} \in \bigpsi{1}
      \end{align}

    \item

      Let $\seq{a_n}\in\littleo{1}$ and $\seq{\epsilon_n}\in\bigpsi{a_n}\subseteq \bigo{a_n}$. Rule \ref{proof:littleo_rule} implies $\seq{\epsilon_n} \in o(1)$ and thus $\lim_{n\to\infty} \epsilon_n = 0$. We have

      \begin{align}
        \limsup_{n\to\infty} \abs{\frac{e^{\epsilon_n}-1}{a_n}} = \limsup_{n\to\infty} \underbrace{\abs{\frac{e^{\epsilon_n}-1}{\epsilon_n}}}_{\to \exp'(0)=1} \cdot \underbrace{\abs{\frac{\epsilon_n}{a_n}}}_{\le 1} \le 1
      \end{align}

      \noindent Thus $\seq{e^{\epsilon_n}} \in \bigpsi{a_n}$.
  \end{enumerate}
\end{proof}
