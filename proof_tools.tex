\chapter{Proofs for the tools of this thesis}

In this chapter I will prove the theorems I have stated in the chapter~\ref{chapter:tools} where I presented the needed tools for this thesis.

\section{Big Psi notation}

\begin{theorem} \label{thm:bigpsi:rules}
  Let $\seq{\epsilon_n}$ and $\seq{a_n}$ be sequences whereby $a_n > 0$ for all $n\in\N$. The big Psi notation has the following arithmetic rules:

  \begin{enumerate}
    \item $\seq{\epsilon_n} \in \bigo{a_n} \land \seq{a_n} \in \littleo{b_n} \implies \seq{\epsilon_n} \in \littleo{b_n}$
    \item $\bigpsi{a_n} \subseteq \bigo{a_n}$
    \item $\bigpsi{a_n} + \littleo{a_n} \subseteq \bigpsi{a_n}$
  \end{enumerate}
\end{theorem}


\begin{proof} ~
  \begin{enumerate}
    \item

      \begin{align}
        & \seq{\epsilon_n}\in\bigo{a_n} \land \seq{a_n} \in \littleo{b_n} \nl
        \implies & \left(\exists C > 0: \limsup_{n\to\infty} \abs{\frac{\epsilon_n}{a_n}} \le C\right) \land \lim_{n\to\infty} \abs{\frac{a_n}{b_n}} = 0 \nl
        \implies & \lim_{n\to\infty} \abs{\frac{\epsilon_n}{b_n}} = \lim_{n\to\infty} \underbrace{\abs{\frac{\epsilon_n}{a_n}}}_{\text{bounded}} \cdot \underbrace{\abs{\frac{a_n}{b_n}}}_{\to 0} = 0 \nl
        \implies & \seq{\epsilon_n} \in \littleo{b_n}
      \end{align}
    \item

      \begin{align}
        \seq{\epsilon_n} \in \bigpsi{a_n} & \implies \limsup_{n\to\infty} \abs{\frac{\epsilon_n}{a_n}} \le 1 \nl
                                          & \implies \exists C_\infty > 0 : \limsup_{n\to\infty} \abs{\frac{\epsilon_n}{a_n}} \le C_\infty \nl
                                          & \implies \seq{e_n} \in \bigo{a_n}
      \end{align}

    \item

      \begin{align}
        & \seq{\epsilon_n} \in \bigpsi{a_n} \land \seq{\delta_n} \in \littleo{a_n} \nl
        \implies& \left(\limsup_{n\to\infty} \abs{\frac{\epsilon_n}{a_n}} \le 1\right) \land \left(\lim_{n\to\infty} \abs{\frac{\delta_n}{a_n}} = 0\right) \nl
        \implies& \limsup_{n\to\infty} \abs{\frac{\epsilon_n+\delta_n}{a_n}} \le \limsup_{n\to\infty} \left(\abs{\frac{\epsilon_n}{a_n}}+\abs{\frac{\delta_n}{a_n}}\right) \le 1 \nl
        \implies& \seq{\epsilon_n+\delta_n} \in \bigpsi{a_n}
      \end{align}
  \end{enumerate}
\end{proof}

\section{Interval arithmetic}

\begin{theorem}
  Let $a,b,\lambda\in\R$ and $\epsilon,\delta\in\Rplus$. We have

  \begin{enumerate}
    \item $\an[\epsilon]{a}+\an[\delta]{b}=\an[\epsilon+\delta]{a+b}$
    \item $\lambda \an[\epsilon]{a} = \an[\abs{\lambda}\cdot \epsilon]{\lambda a}$
  \end{enumerate}
\end{theorem}


\begin{proof} ~
  \begin{enumerate}
    \item

      \begin{align}
        x \in \an[\epsilon]{a} \land y \in \an[\delta]{b} & \implies a-\epsilon \le x \le a+\epsilon \land b-\delta \le y \le b+\delta \nl
        & \implies (a+b)-(\epsilon + \delta) \le x+y \le (a+b)+(\epsilon + \delta) \nl
        & \implies x+y \in \an[\epsilon+\delta]{a+b}
      \end{align}

    \item

      \begin{align}
        \an[\epsilon]{\an[\delta]{a}} & = \bigcup_{x\in \interval{a-\delta}{a+\delta}} \an[\epsilon]{x} = \bigcup_{x\in\interval{a-\delta}{a+\delta}} \bigcup_{y\in\interval{-\epsilon}{\epsilon}} \{x+y\} \nl
        &= \an[\delta]{a} + \an[\epsilon]{0} = \an[\epsilon+\delta]{a}
      \end{align}
 
    \item

      This follows directly from $-\abs{b} \le b \le \abs{b}$ because $b \in \{-\abs{b}, \abs{b}\}$.

    \item

      \begin{align}
        \an[\epsilon]{a+b} \subseteq \an[\epsilon]{a+\an[\abs{b}]{0}} = \an[\epsilon]{\an[\abs{b}]{a}} = \an[\epsilon+\abs{b}]{a}
      \end{align}

    \item
      Let $\lambda \ge 0$ and let $f_\lambda$ be the function $f_\lambda : \an[\epsilon]{a}\to\R:x\mapsto \lambda x$. $f_\lambda$ attains its maximum at $\lambda(a+\epsilon)$ and its minimum at $\lambda(a-\epsilon)$. Because $f_\lambda$ is continuous and $\an[\epsilon]{a}$ is connected, also $f_\lambda(\an[\epsilon]{a})$ is connected and therefore $f_\lambda$ attains all values between $\lambda(a-\epsilon)$ and $\lambda(a+\epsilon)$ (intermediate value theorem~\cite{wiki:intermediatevaluetheorem}). Thus

      \begin{align}
        f_\lambda\left(\an[\epsilon]{a}\right) = \interval{\lambda(a-\epsilon)}{\lambda(a+\epsilon} = \interval{\lambda a - \lambda \epsilon}{\lambda a+\lambda \epsilon} = \an[\lambda \epsilon]{\lambda a} = \an[\abs{\lambda}\cdot \epsilon]{\lambda a}
      \end{align}

      Let $\lambda < 0$ and $f_\lambda$ be as above. The maximum of this function is now $\lambda (a-\epsilon)$ and its minimum is $\lambda (a+\epsilon)$. Due to the intermediate value theorem~\cite{wiki:intermediatevaluetheorem} all values between $\lambda(a+\epsilon)$ and $\lambda(a-\epsilon)$ are attained by the function. So we have

      \begin{align}
        f_\lambda\left(\an[\lambda]{a}\right) = \interval{\lambda(a+\epsilon)}{\lambda(a-\epsilon)} = \interval{\lambda a -\abs{\lambda}\cdot \epsilon}{\lambda a + \abs{\lambda}\cdot \epsilon} = \an[\abs{\lambda}\cdot \epsilon]{\lambda a}
      \end{align}

 \end{enumerate}
\end{proof}

\begin{theorem} \label{thm:multiplicative:rules}
  Let $a,b\in \Rplus$ and $\epsilon,\delta \in\Rplusnull$. We have

  \begin{enumerate}
    \item $\delta \le \epsilon \implies \ean[\delta]{a} \subseteq \ean[\epsilon]{a}$ 
    \item $\ean[\epsilon]{a} \cdot \ean[\delta]{b} = \ean[\epsilon+\delta]{a\cdot b}$
    \item $\ean[\epsilon]{\ean[\delta]{a}} = \ean[\epsilon+\delta]{a}$
    \item $\ean[\epsilon]{a} + \ean[\epsilon]{b} = \ean[\epsilon]{a+b}$
    \item $\ean[\epsilon]{a} = \an[\sinh(\epsilon)a]{\cosh(\epsilon)a}$
    \item $\ean[\epsilon]{a} \subseteq \an[\left(e^\epsilon-1\right)a]{a}$
    \item $\abs{\frac ab-1} \le \epsilon \implies a\in\ean[\epsilon]{b}$
  \end{enumerate}
\end{theorem}


\begin{proof} ~
  \begin{enumerate}
    \item

      \begin{align}
        x \in \ean[\epsilon]{a} \land y \in \ean[\delta]{b} & \implies \left(ae^{-\epsilon} \le x \le ae^{\epsilon}\right) \land \left(be^{-\delta} \le y \le be^\delta\right) \nl
        & \implies abe^{-(\epsilon+\delta)} \le xy \le abe^{\epsilon+\delta} \nl
        & \implies xy \in \ean[\epsilon+\delta]{a\cdot b}
      \end{align}

    \item
      
      \begin{align}
        x\in\ean[\epsilon]{a} & \iff x \in \interval{ae^{-\epsilon}}{ae^\epsilon} \nl
        & \iff x \in \an[\frac{ae^\epsilon-ae^{-\epsilon}}2]{\frac{ae^\epsilon+ae^{-\epsilon}}2} \nl
        & \iff x \in \an[a\cdot \frac{e^\epsilon-e^{-\epsilon}}2]{a\cdot \frac{e^\epsilon+e^{-\epsilon}}2} \nl
        & \iff x\in \an[a\sinh(\epsilon)]{a\cosh(\epsilon)}
      \end{align}

    \item

      \begin{align}
        \ean[\epsilon]{a} &= \an[\sinh(\epsilon)a]{\cosh(\epsilon)a} \nl
        &= \an[\sinh(\epsilon)a]{a+\cosh(\epsilon)a-a} \nl
        &\subseteq \an[\sinh(\epsilon)a+\abs{\cosh(\epsilon)a-a}]{a} \nl
      &\begin{comment} a > 0 \land \cosh(\epsilon) > 0 \implies \cosh(\epsilon)a > a \end{comment} \nl
        &= \an[\sinh(\epsilon)a+\cosh(\epsilon)a-a]{a} \nl
      &\begin{comment} \sinh(\epsilon) + \cosh(\epsilon) = \frac 12 \left(e^\epsilon-e^{-\epsilon}\right) + \frac 12 \left(e^\epsilon+e^{-\epsilon}\right) = e^\epsilon \end{comment} \nl
        &= \an[e^\epsilon a-a]{a} \nl
        &= \an[\left(e^\epsilon-1\right)a]{a}
      \end{align}
  \end{enumerate}

\end{proof}
